\chapter{Einleitung}   \label{ch_1}
% Aggressivität
Prof. Dr. Friedrich Hacker sagte \enquote{Jede Aggression sucht sich zu rechtfertigen. Angefangen hat doch immer der andere}. \parencite{Friedrich_Hacker} Aggressivität zeigt sich auf unterschieliche Art und Weisen. Jeder Mensch hat ein gewisses Niveau von Aggressivität. Einige zeigen es mehr als andere. 

%Aggression an sich eher spontan: 1st time is Gewalt vielleicht mal ausgerutscht und das
%entwickelt sich dann in langanhaltende häusliche Gewalt. 

% häusliche Gewalt
Im Falle von häuslicher Gewalt tut dies der Partner, der man am meisten vertraut und 
bei der man nie damit gerechnet hätte, dass sie es einem gegenüber zeigt. In so einem 
Fall weiß die betroffene Person oftmals nicht, wie sie dieser Situation entkommen kann. 
Das Zuhause sollte für alle ein sich Ort sein, an dem sie sich entspannen können. 
Dennoch gibt es viele Fälle wo dies nicht der Fall ist. Im Jahr 2020 gab es 148.031 
gemeldete Fälle von häuslicher Gewalt \parencite{häusliche_Gewalt}. 
Die Dunkelziffer ist weit höher, da viele Betroffene keine Anzeige erstatten. Bei Männern
ist ein solches Verhalten möglicherweise schambehaftet.

% Victim Blaming
Wie Hacker \parencite{Friedrich_Hacker} 
in seinem Zitat beschrieben hat, sucht der Aggressor sich zu rechtfertigen, er fällt
dem \textit{Victim Blaming} zum Opfer. Das bedeutet, dass eine Person die Verantwortung
dem Geschädigten zuschreibt. 

% Gewaltmythen
Victim Blaming ist ein großer Aspekt von Gewaltmythen. Trotz jahrelanger Forschung in 
diesem Gebiet, sind diese Mythen heutzutage noch weit verbreitet. 
Sie beruhen auf falschen Annahmen der Verantwortung und deutet eine gewisse Einflussnahme 
des Opfers an.

Diese Studie untersucht die Aggressivität des Beobachters von häuslicher Gewalt und 
dessen Akzeptanz von Gewaltmythen. Zu Beginn werden die einzelnen Konstrukte häusliche 
Gewalt, Gewaltmythen und Aggressivität erleutert und der theoretische Hintergrund, so wie 
die daraus abgeleiteten Hypothesen widergespiegelt. In den darauffolgenden Kapiteln wird 
auf die Methoden der Studiendurchführung, wie auch auf die Ergebnisse eingegangen. In der 
abschließenden Diskussion erfolgt eine umfassende Bewertung der Ergebnisse sowie ein 
Ausblick auf den weiteren Forschungsbedarf in diesen Themengebieten.

Aus Gründen der besseren Lesbarkeit wird in der nachfolgenden Arbeit die Sprachform des
generischen Maskulinums verwendet. An dieser Stelle wird darauf hingewiesen, dass die 
ausschließliche Verwendung der männlichen Form geschlechtsunabhängig verstanden werden 
soll.