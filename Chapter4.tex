\chapter{Diskussion}   \label{ch_4}
Interpretation der Ergebnisse und Reflexion der Arbeit

\section{Zitationsbeispiele}

\textbf{Werk einer Person:} sieht dann im Text so aus. \textcite{Hussy2010} und in 
der Klamer immer so \parencite{Hussy2010}.

\textbf{Körperschaftsautoren:} Beim ersten Auftreten sieht \textcite{6.13t} so aus.
Und in der Klammer sieht eine Körperschaft so aus: \parencite{6.13p}. Bei einer 
wiederholten Erwähnung sieht es dann jeweils so aus: \textcite{6.13t} im Text und 
\parencite{6.13p} in der Klammer.


\subsection{Werk von zwei oder mehr Personen}
\textbf{Werk von zwei Autoren:} Innerhalb des Textes wird die Quelle immer so
verfasst: \textcite{Amelang2006}. In der Klammer so: \parencite{Amelang2006}.

\textbf{Werk von mehr als zwei aber weniger als sechs Autoren:} Das ist der Beginn 
für die erste Erwähnung von vier Autoren, um sie später nochmal zu erwähnen 
\textcite{b0424f3eebf64b03a01a8841d7e3bf8d}. Das hier ist die Art und Weise in 
der Klammer \parencite{b0424f3eebf64b03a01a8841d7e3bf8d}. Und hier werden sie 
wieder erwähntv\textcite{b0424f3eebf64b03a01a8841d7e3bf8d}.

\textbf{Werk von sechs oder mehr Autoren aber unter acht:} So sieht es aus, wenn
sechs Autoren erwähnt werden \textcite{doi:10.1026/1616-3443/a000099}. In der
Klammer steht es wie folgt \parencite{doi:10.1026/1616-3443/a000099}.

\textbf{Werk von mindestens acht Autoren:} Hier werden im Literaturverzeichns nur
die ersten sechs Autoren genannt werden, gefolgt von drei Auslassungspunkten
und am Ende noch der letzte Name. So sieht \textcite{ao1} im Text aus. Und in 
der Klamer dann so: \parencite{ao1}.

\textbf{Werk von einer Person mit zwei Nachnamen:} Etwas gesagt von 
\textcite{AQUINOJARQUIN2021}. Und dann zitieren wir ihn nochmal,aber dieses mal in 
Klammern \parencite{AQUINOJARQUIN2021}.

\subsection{Blockzitat}
\noindent Ein Blockzitat ist ein wörtliches Zitat ab 40 Wörtern.
\begin{Blockzitat}
    CRISPR–Cas systems are recently discovered, RNA-based immune systems that 
    control invasions of viruses and plasmids in archaea and bacteria. Prokaryotes 
    with CRISPR–Cas immune systems capture short invader sequences within the 
    CRISPR loci in their genomes, and small RNAs produced from the CRISPR loci 
    (CRISPR (cr)RNAs) guide Cas proteins to recognize and degrade (or otherwise 
    silence) the invading nucleic acids. There are multiple variations of the 
    pathway found among prokaryotes, each mediated by largely distinct components 
    and mechanisms that we are only beginning to delineate. Here we will review our 
    current understanding of the remarkable CRISPR–Cas pathways with particular 
    attention to studies relevant to systems found in the archaea. 
    \parencite{TERNS2011321}
\end{Blockzitat}