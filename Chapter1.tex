
%{\chapterfont
\chapter{Theorie}   \label{ch_1}
Einleitung und prüfende konzeptuelle Hypothese(n)

\section{Um etwas in Anführungsstrichen zu schreiben}    \label{sec_1.1}
Hier sind die unterschiedlichen Anzführungszeichen für sowohl DE wie EN.

\subsection{Deutsche Art}    \label{subsubsec_1.1.1}
\glqq Das ist die deutsche Schreibweise\grqq

\glq So haben wir einfache Anführungsstriche im Deutschen\grq

\subsection{Englische Art}   \label{subsubsec_1.1.2}
``Das ist die englische Schreibweise''

`So haben wir einfache Anführungsstriche im Englischen'



\section{Underlined, fett oder kursiv}   \label{subsec_1.2}
So erscheint das Wort \underline{unterschtrichen}.
So erscheint das Wort \textbf{fett}.
So erscheint das Wort \textit{kursiv}.

\subsection{Zu kursiv schreibende Elemente}  \label{subsubsec_1.2.1}
Folgende Elemente werden \textit{kursiv} geschrieben:
\begin{itemize}[leftmargin=1.25cm]
    \item statistische Symbole mit lateinischen Buchstaben
    \item biologische Begriffe
    \item linguistische Beispiele
    \item Skalenbezeichnungen von Fragebögen
    \item zum ersten Mal verwendete Fachbegriffe oder missverständliche Bezeichnung
    \item im Literaturverseichnis für Angaben von Titeln und Bänden
\end{itemize}
