\begin{table}[htb]
    \caption[Itemanalyse für Skala 1 Schlaf]{\textit {Itemanalyse für Skala 1 Schlaf}} 
    \label{Itemanalyse für Skala 1 Schlaf}
    \centering
    \begin{adjustbox}{width=\textwidth}
    %\small
    \begin{tabular}{rlrrrr}
      \hline
    Nr.      & Itemtext & \( M \) & \( SD \) & \( p_i \) & \( r_{it-i} \) \\
      \hline
    1.      & Ich denke, dass ich ausreichend schlafe.
      & 3.48     & 1.24    & .62        & .53     \\
    2.      & Ich achte darauf, wie lange ich täglich schlafe.
      & 3.59     & 1.15    & .65        & .51     \\
    3.      & Wenn ich ausreichend geschlafen habe, fühle ich mich gesund.
      & 4.02     & 0.89    & .76        & .31     \\
    4.      & Wenn ich morgens aufstehe, fühle ich mich erholt und ausgeruht.
      & 3.05     & 1.18    & .51        & .56     \\
    5.      & Ich schlafe täglich 7-8 Stunden.
      & 3.72     & 1.18    & .63        & .61     \\
    6.      & (-) Ich habe abends Probleme einzuschlafen.
      & 3.12     & 1.13    & .53        & .26     \\
       \hline
    \end{tabular}
    \end{adjustbox}
    
    \begin{tablenotes}
        \item \textit{Anmerkungen.} \( N \) = 135. Codierung der Items: 1 = stimme
        nicht zu, 2 = stimme eher nicht zu, 3 = stimme teilweise zu, 4 = stimme eher 
        zu, 5 = stimme vollständig zu.\linebreak(-) = recodiertes Item. \( M \) Item-
        Mittelwert, \( SD \) Item-Streuung, \( p_i \) Item-Schwierigkeit, 
        \linebreak\( r_{it-i} \) Korregierte Item-Trennschärfe. Cronbachs-\textalpha \  
        (Skala 1) = .72.
      \end{tablenotes}
    \end{table}