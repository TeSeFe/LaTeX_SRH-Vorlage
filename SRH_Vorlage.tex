%% Diese Vorlage wurde am 24.08.2021 von Teresa Seidl Fernández erstellt 
%% und wird den Studenten/-innen der Fakultät für Angewandte Psychologie
%% der SRH Hochschule zur Verfügung gestellt und kann auch von ihnen
%% angepasst werden.






%%%%%%%%%%%%%%%%%%%%%%%%%%%% VORSPANN %%%%%%%%%%%%%%%%%%%%%%%%%

\documentclass[12pt, a4paper]{report}

\usepackage{SRH_Vorlage}

\usepackage[ngerman]{babel}        % n steht für neu und richtet sich an den
% neuen Rechtschreibregeln. Weitere Sprachen können verwendet werden. Um eine
% dieser Sprachen als "Hauptsprache" zu determinieren: main=[gewünschte Sprache]
\usepackage{ifthen}     % The package’s basic command is \ifthenelse, which 
% can use a wide array of tests. Also provided is a simple loop command 
% \whiledo. Ifthen is a separate package within the LaTeX distribution; 
% while it will always be present in a LaTeX distribution, a \usepackage 
% command is always needed to load it. 
\usepackage[retainmissing]{MnSymbol}    % MnSymbol is a symbol font family.
%  Almost all of LaTeX and AMS mathematical symbols are provided
\usepackage{titletoc}     % Alternative headings for toc/lof/lot
% Anhang:
\usepackage[titletoc,title]{appendix} 
% für Tabellen:
\usepackage[flushleft]{threeparttable}  % um Anmerkungen unter Tabellen und 
% Abbildungen setzen zu können -> needed for notes
\usepackage{booktabs}
\usepackage{multirow}
\usepackage{adjustbox}
\usepackage{tabu}
\usepackage{longtable}
\usepackage{tabularx}
% Abbildungen:
\usepackage{graphicx}
\usepackage{subcaption}   % environment allows you to place multiple images at a 
% certain location next to each other

% is für "Abbildung" im Verzeichnis
\makeatletter
\def\@chapter[#1]#2{\ifnum \c@secnumdepth >\m@ne
                         \refstepcounter{chapter}%
                         \typeout{\@chapapp\space\thechapter.}%
                         \addcontentsline{toc}{chapter}%
                                   {\protect\numberline{\thechapter}#1}%
                    \else
                      \addcontentsline{toc}{chapter}{#1}%
                    \fi
                    \chaptermark{#1}%
                    \if@twocolumn
                      \@topnewpage[\@makechapterhead{#2}]%
                    \else
                      \@makechapterhead{#2}%
                      \@afterheading
                    \fi}
\makeatother


%%%%%%%%%%%%%%%%%%%%%%%%%%%% END VORSPANN %%%%%%%%%%%%%%%%%%%%%%%%%


%%%%%%%%%%%%%%%%%%%%%%%%%%%% TEXT-TEIL %%%%%%%%%%%%%%%%%%%%%%%%%


\begin{document}
\begin{titlepage}
\begin{center}

\thispagestyle{empty} 

    SRH Hochschule Heidelberg\\
    Fakultät für Angewandte Psychologie\\
    Staatlich anerkannte Hochschule\\[3cm]
    % Dieser Teil bleibt fix
    
    Bachelor-Thesis\\   
    zur Erlangung des akademischen Grades\\
    Bachelor of Science Gesundheitspsychologie\\[1.5cm]
    % Bei einer Studienarbeit steht hier stattdessen: 
    % Studienarbeit im Modul XY

    Thema:\\
    {\bf Eine empirische Untersuchung der Aggressivität des Beobachters häuslicher Gewalt
    und dessen Akzeptanz von Gewaltmythen} \\[2cm]  
    % Thema in geschweifte Klammer einfügen

    Eingereicht von: Teresa Seidl Fern\'andez\\    % eigenen Namen einfügen
    Matrikelnummer: 11013160\\                     % eigene Matrikelnr einfügen
    Gruppennummer: 61751901\\[2.5cm]               % eigene Gruppennr einfügen

    Studiengangsleiter: Dipl-Psych. Willi Neuthinger\\
    % Das bleibt auf weiteres fix
    Betreuender Dozent: Prof. Dr. Haß\\[1cm]       % Name der betreuenden Person
    % in der Thesis oder im Modul

    Heidelberg, den \today

\end{center}
\end{titlepage}
\newcounter{abstractpage}
\setcounter{abstractpage}{\value{page}}

\begin{abstract}
 \thispagestyle{plain}
 \setcounter{page}{2}

%%%%%%%%% START ABSTRACT %%%%%%%%%

  \noindent Diese Arbeit beinhaltet eine LaTeX Vorlage für die Fakultät für Angewandte
  Psychologie der SRH Hochschule Heidelber. Es umfasst sowohl eine Gliederung 
  für eine übliche Arbeit als auch kommentierte Beispiele und generelle Regeln.
  
%%%%%%%%%% END ABSTRACT %%%%%%%%%%

 \setcounter{abstractpage}{\value{page}}
\end{abstract}

\setcounter{page}{\value{abstractpage}}
\stepcounter{page}


\tableofcontents
\clearpage
\listoffigures
\clearpage
\listoftables


\chapter{Einleitung}   \label{ch_1}
% Aggressivität
Prof. Dr. Friedrich Hacker sagte \enquote{Jede Aggression sucht sich zu rechtfertigen. Angefangen hat doch immer der andere}. \parencite{Friedrich_Hacker} Aggressivität zeigt sich auf unterschieliche Art und Weisen. Jeder Mensch hat ein gewisses Niveau von Aggressivität. Einige zeigen es mehr als andere. 

%Aggression an sich eher spontan: 1st time is Gewalt vielleicht mal ausgerutscht und das
%entwickelt sich dann in langanhaltende häusliche Gewalt. 

% häusliche Gewalt
Im Falle von häuslicher Gewalt tut dies der Partner, der man am meisten vertraut und 
bei der man nie damit gerechnet hätte, dass sie es einem gegenüber zeigt. In so einem 
Fall weiß die betroffene Person oftmals nicht, wie sie dieser Situation entkommen kann. 
Das Zuhause sollte für alle ein sich Ort sein, an dem sie sich entspannen können. 
Dennoch gibt es viele Fälle wo dies nicht der Fall ist. Im Jahr 2020 gab es 148.031 
gemeldete Fälle von häuslicher Gewalt \parencite{häusliche_Gewalt}. 
Die Dunkelziffer ist weit höher, da viele Betroffene keine Anzeige erstatten. Bei Männern
ist ein solches Verhalten möglicherweise schambehaftet.

% victim blaming
Wie Hacker \parencite{Friedrich_Hacker} 
in seinem Zitat beschrieben hat, sucht der Aggressor sich zu rechtfertigen, er fällt
dem \textit{victim blaming} zum Opfer. Das bedeutet, dass eine Person die Verantwortung
dem Geschädigten zuschreibt. 

% Gewaltmythen
Victim blaming ist ein großer Aspekt von Gewaltmythen. Trotz jahrelanger Forschung in 
diesem Gebiet, sind diese Mythen heutzutage noch weit verbreitet. 
Sie beruhen auf falschen Annahmen der Verantwortung und deutet eine gewisse Einflussnahme 
des Opfers an.

Diese Studie untersucht die Aggressivität des Beobachters von häuslicher Gewalt und 
dessen Akzeptanz von Gewaltmythen. Zu Beginn werden die einzelnen Konstrukte häusliche 
Gewalt, Gewaltmythen und Aggressivität erleutert und der theoretische Hintergrund, so wie 
die daraus abgeleiteten Hypothesen widergespiegelt. In den darauffolgenden Kapiteln wird 
auf die Methoden der Studiendurchführung, wie auch auf die Ergebnisse eingegangen. In der 
abschließenden Diskussion erfolgt eine umfassende Bewertung der Ergebnisse sowie ein 
Ausblick auf den weiteren Forschungsbedarf in diesen Themengebieten.

Aus Gründen der besseren Lesbarkeit wird in der nachfolgenden Arbeit die Sprachform des
generischen Maskulinums verwendet. An dieser Stelle wird darauf hingewiesen, dass die 
ausschließliche Verwendung der männlichen Form geschlechtsunabhängig verstanden werden 
soll.


\chapter{Theorie}   \label{ch_2}
In diesem Kapitel werden zunächst die verwendeten Konstrukte Aggressivität und Aggression, häusliche Gewalt und abschließend Gewaltmythen erkläutert. Anschließend folgt eine theoretische Herleitung der einzelnen Hypothesen.

% 2.1 Konstrukte:
\section{Konstrukte}    \label{sec_2.1}
Im Anschluss erfolgt in den Unterkapiteln ,  und  eine Darbietung der einzelnen Konstrukte. Beginnend werden in ~\ref{subsec_2.1.1} Aggressivität und Aggression die Aggressivität und Aggression voneinander unterschieden und der Unterschied zur Gewalt verdeutlicht. Anschließend erfolgt eine Darbietung verschiedener Arten von Aggression und Theorien, die die Entstehung von Aggression versuchen zu erklären. In Kapitel ~\ref{subsec_2.1.2} Häusliche Gewalt wird auf Muster und Arten, auf mögliche Ursachen und Aufrechterhaltung und abschließend auf die Folgen von Gewalt eingegangen. Inhalt dieses Kapitels ist auch eine kurze Erläuterung bezüglich der Erfüllung gewisses Artikel der Istambul Konvention. Im abschließenden Kapitel ~\ref{subsec_2.1.3} Gewaltmythen wird dieser Begriff aufgefasst, sowie das victim blaming. Zum Schluss diesem Unterkapitel erfolgen theoretische Ansätze, die versuchen das Phänomen des vitim blamings zu erklären.

\subsection{Aggressivität und Aggression}    \label{subsec_2.1.1}

%Aggressivität vs Aggression
Aggressivität ist nicht gleichzusetzen mit Aggression. Ersteres bezieht sich auf eine überdauernde Disposition eines Individuums zu aggressivem Verhalten. Diese Bereitschaft wird nich immer offen ausgeführt und ist unterschiedlich ausgeprägt
\parencite{Def_unterscheidung_Springer, Def_Aggressivität_Duden, Def_Aggressivität_Spektrum}. Aggressivität entspricht demzufolge einer Verhaltenstendez, einer übergeordnete Charaktereigenschaft, die sich in Form von Aggression oder aggressivem Verhalten zeigt. Personen, die Aggressivität als Teil ihrer Persönlichkeit haben, können beispielsweise die folgenden Charakteristika aufweisen \parencite{Def_Aggressivität_eng1}:
\begin{itemize} [leftmargin=1.25cm]
      \item Problematik die Emotionen und Gedanken anderer zu verstehen und nachzuempfinden
      \item externe Attribution
      \item Soziale Manipulation, um das Bedürfnis von Kontrolle über andere 
            Personen zu befriedigen
      \item Emotionale und affektive Defizite zeigen sich durch Aggressivität auf Grund einer fehlerhaften Wahnehmung von fehlender Wertschätzung anderer
      \item Aggressive Personen sind der Meinung, dass sie für ihre Verwandten oder nahestehenden Menschen nicht wichtig sind
\end{itemize}

%Aggression
Aggression hingegen ist als vorübergehende Handlungsart zu verstehen, die es zum Ziel hat eine Person oder einen Gegenstand zu verletzen oder zu schädigen
\parencite{Def_Aggression_1939, Def_unterscheidung_Springer, Def_Aggression_Duden}.
Ursprünglich kommt das Wort Aggression aus dem Lateinischen und bedeutet 
\enquote{an eine Sache heran gehen} oder \enquote{etwas in Angriff nehmen} \parencite{was_Aggression} und ist weder positiv noch negativ. Im normalen Sprachgebrauch besitzt dieses Wort jedoch häufig eine negative Konnotation und wird von großen Teilen der Bevölkerung missbilligt. Aggressive Handlungen reichen von negativen Äußerungen über Mitmenschen sowie das Schreien oder Fluchen bis hin zu beabsichtigter Schädigung fremden Eigentums. 

Negative Aggression gilt aufgrund der negativen Emotionen, die durch sie ausgelöst werden, als ungesund. Dauerhaftes Bestehen solcher Emotionen kann  schädlich für den Menschen sein \parencite{Aggression}.

Wenn Aggression aber das eigene Überleben, den eigenen Schutz oder auch die Bewahrung von Beziehungen fördert, dann bezeichnet \textcite{positive_aggression} 
es als positives und gesundes Verhalten. Wie \textcite{Aggression} zusammenfasst, ist es, im Sinne der positiven Aggression, während der Entwicklungsjahre eines 
Kindes und Jugendlichen notwendig ein gewisses Maß an Aggressivität zu besitzen. Dies hilf dem Heranwachsenden beim Ausbau von Autonomie und der eigenen Identität. Des Weiteren wird ein gewisses Grad an Aggression im Zusammenhang mit Wettkämpfen oder anderen Arten von Konkurrenz meist sogar erwünscht. Wenn die Aggression in die richtige Richtung gelenkt wird, ist sie die nötige Kraft, um ein gesundes Maß an Selbstbewusstsein, Dominanz und Unabhängigkeit zu erlangen \parencite{Aggression}. Positive Aggression hat viele Formen und Facetten. Ergänzend zu \textcite{positive_aggression} zählt \textcite{jack1999behind}
das Streben nach neuen Möglichkeiten und die Verteidigung gegen Schaden als Ausdruck positiver Aggression.

%Aggression nicht gleich Gewalt
Laut dem Duden ist Gewalt die \enquote{gegen jemanden, etwas [rücksichtslos] angewendete physische oder psychische Kraft, mit der etwas erreicht werden soll} \parencite{Gewalt_Duden}. Diese Definition ähnelt der der Aggression. Oftmals werden diese Wörter im Sprachgebrauch gleichdeutend verwendet. Sie sind jedoch nicht als Synonyme zu gebrauchen. Die Trennung beider Begriffe ist dennoch nicht einfach. \textcite{Def_Aggressivität_vs_violence} trennt diese beiden Begriffe wie folgt. Aggression ist ein natürlicher und angeborener Instinkt, der nicht ausschließlich dem Menschen zuzuschreiben ist. Gewalt hingegen ist ein von der Kultur bestimmtes Element und Teil der menschlichen Zivilisation. Wie bereits näher gebracht ist die Aggression, wie ihre höhere Instanz die Charaktereigenschaft Aggressivität, von biologischem Ursprung, dessen Ziel und Zweck das Überleben ist \parencite{Def_Aggressivität_vs_violence, Aggression}.
Die positiven Aggression von Liu ähnelt der Auffassung von Clark. Letztere weist aber auch darauf, dass durch die Beziehungen zu Gewalt die Aggression zu einem soziokulutrellen Aspekt geworden ist \parencite{Def_Aggressivität_vs_violence}.
Dies kann ein möglicher Grund für die erschwerte Abgrenzung zwischen diesen beiden
Begriffen sein. 

Die negative Aggression von Liu lässt sich zu Teilen mit der Gewalt von Clark vergleichen \parencite{Def_Aggressivität_vs_violence, Aggression}. Sie sieht Gewalt als erlerntes Verhalten, dass durch kulturelle Ideologien und Werte geprägt ist,  geplant und absichtlich ausgeführt wird. Der unterschied zwischen Aggression und Gewalt liegt darin, dass Gewalt versucht Macht und Kontrolle zu erhalten, während Aggression dem Eigenschutz dient \parencite{Def_Aggressivität_vs_violence}.

Zusammenfassend ist es nicht unbedingt ratsam zu versuchen die Begriffe Aggression, Aggressivität und Gewalt so klar abzutrennen. Die Nutzung und sprachliche Bedeutung der Worte haben sich im Wandel der Zeit verändert, wodurch sich die Bedeutungen der einzelnen Begriffe näher gekommen sind. Des Weiteren ist die klare Abtrennung durch die Verwobenheit der Konstrukte erschwert. Zusätzlich zu den hier aufgeführten Begriffen gibt es noch weitere, die mit dieser Thematik verwandt sind, auf die in dem Umfang dieser Arbeit jedoch nicht eingegangen werden. Des Weiteren kann Gewalt, so wie die Aggression, in unterschiedliche Arten unterteilt werden. Eine solche Unterteilung wird in Kapitel ~\ref{subsec_2.1.2} vorgenommen.


%sexuelle Aggro 9.5.2 im sozio buch

\subsubsection{Aggressionsarten}    \label{subsubsec_2.1.3.1}
%Arten von aggressivem Verhalten/ Aggressivität/ Aggression
In den vorangegangenen Paragraphen wurde bereits auf unterschiedliche Arten von Aggression eingegangen, auch wenn sie nicht explizit genannt wurden. In dieser Arbeit werden auf die folgenden Typen von Aggression eingegangen: impulsive, instrumentelle, physische und abschließenden verbale Aggression.

% impulsive Aggression
Die \textit{impulsive}, oder auch affektive Aggression ist die unvorhersehbare und automatische Darbietung von Gewalt. Oftmals entsteht sie aus dem momentan erlebten Emotionen ohne über die eigendliche Handlung oder ihre Folgen nachzudenken. Diese Reaktion auf eine reale, oder auch eingebildete Provokation, kann unkontrolliert oder unverhaltnismäßig erscheinen \parencite{impulsive_instrumental_aggro_healtline, impulsive_aggro}. Impulsive Aggression ist bei einigen psychischen Störungen wie beispielsweise ADHS, Zwangsstörungen, oder bipolare Störungen zu beobachten \parencite{impulsive_aggro_psych_Störung}.

% instrumentelle Aggression
Wie die Bezeichnung diese Art von Aggression nahelegt, handelt es sich bei der 
\textit{instrumentellen} oder kognitiven Aggression um ein Hilfsmittel um ein größeres Ziel zu erreichen. Hierbei besteht keine zwangläufige Absicht einer Person Schaden zuzuführen \parencite{instrumental_aggro, instrumental_dictionary}.
Ein Beispiel instrumenteller Gewalt sind Auftragskiller und zu gewissem Grad auch Soldaten, die für die Zielerreichung des Geldes Personenschaden als Nebeneffekt annehmen. Diese Darbietung von Aggression ist kalkulierter und zielgerichteter ohne die Kontrolle zu verlieren \parencite{impulsive_instrumental_aggro_healtline}.

% physische und verbale Aggression
Aggression wird letzendlich auf zwei verschiedene Art und Weisen ausgedrückt. Wenn sie in Form von Schlägen, Tritten, oder jeglicher weiter Handlungen, die dazu führen, dass eine Person physisch verletzt wird, ausgedrückt wird, dann handelt es sich um \textit{physische} Aggression \parencite{impulsive_instrumental_aggro_healtline, physische_verbale_aggro, physische_verbale_aggro_2}. Bei der \textit{verbale} Aggression wiederum handelt es sich um Worte, die einen schädigenden Effekt haben. Es handelt sich dabei um Beschimpfungen, Drohungen oder Mobbing, um Einige zu nennen \parencite{physische_verbale_aggro, physische_verbale_aggro_2, impulsive_instrumental_aggro_healtline}. Obwohl der Schaden physicher Aggression einfacher zu erkennen ist, sind die Kosten verbaler Aggression hoch. Mobbingopfer wießen im vergleich zu anderen Kindern gehäuft Depression, Angstzustände, Einsamkeit und Ablehnung durch Gleichaltrige auf \parencite{ausmaß_verbale_aggro}.



\subsubsection{Aggressionstheorie}    \label{subsubsec_2.1.3.2}
In der Forschung gibt es mehrere Modelle und Theorien, die sich  mit der Entstehung und Aufrechterhaltung von Aggression und aggressivem Verhalten befassen. Im Rahmen dieser Arbeit wird ein näherer Blick auf den lerntheoretischen Ansatz geworfen. Lernerfahrungen haben zweifellos eine wichtige Rolle in der Entstehung und Aufrechterhaltung aggressivem Verhaltens \parencite{Aggro_Theorie}. Dabei sind die \textit{direkte Verstärkung} und das \textit{Modelllernen} von Bedeutung. 

Bei der Verstärkung wird aggressives Verhalten belohnt, wodurch das Kind lernt, dass solches Benehmen angebracht ist. Die Belohnung tritt in Kraft, durch die Erreichung eines zuvor festgelegten Ziels oder durch die Erfahrung sozialer Annerkennung als Folge des aggressiven Verhaltens. Unter direkter Verstärkung ist damzufolge der Effekt positiver Konsequenzen auf aggressives Verhalten zu verstehen \parencite{Aggro_Theorie_Buch}.

Das Modelllernen geht davon aus, dass die Etablierung von aggressivem Verhalten keine eigene motorische Erfahrung benötigt. Laut diesem Mechanismus lernt das Individuum durch Beobachtung aggressiven Verhaltens, dieses anzuwenden. Durch die Belohnung oder Bestrafung der beobachteten Person lernt das Individuum welche Formen von Aggression in welchen Umgebungen und zu welchen Ausmaßen toleriert werden \parencite{Aggro_Theorie_Buch}.
\subsection{Häusliche Gewalt}    \label{subsec_2.1.2}
% Definition häusliche Gewalt
Wie in Kapitel ~\ref{subsec_2.1.1} Aggressivität und Aggression bereits erwähnt ist der Zweck von Gewalt die Macht und Kontrolle zu erhalten \parencite{Def_Aggressivität_vs_violence}. Wenn in einer Beziehung oder innerhalb der Familie eine Person versucht Macht oder Kontrolle über ein anderes Mitglied zu haben, zählt dies zur häuslichen Gewalt. Sowohl in bestehenden, wie auch in aufgelösten Beziehungen familiären, ehelichen oder eheähnlichen Ursprungs kann häusliche Gewalt auftreten \parencite{Def_haus_Gewalt, Def_haus_Gewalt_2}. Diese kommmt nicht nur in der häuslichen Umgebung, einem als sicher gedachten Ort vor, sondern kann auch im öffentlichen Raum stattfinden \parencite{Gewaltarten_WHO}. 

Es lassen sich dabei zwei Muster identifizieren. Ein \textit{spontanes Konfliktverhalten}, oder auch \textit{situative Gewalt} gennant, kann einmalig, aber auch regelmäßig stattfinden und hat die Funktion einer negativen Stressbewältigung. Durch fehlende Ressourcen sehen die gewalttätigen Personen nur die Gewalt als einzige Lösung, um ein Konflikt zu lösen. Ein solches Gewaltmuster ist sowohl bei Männern, wie auch bei Frauen zu finden. Das Motiv von Macht und Kontrolle ist in diesen Fällen nicht ausschlaggebend. Dieses Verhaltensmuster kann sich jedoch in langanhaltendes \textit{systematisches Gewalt- und Kontrollverhalten} verwandeln. Hier haben Macht und Kontrolle eine große Rolle. In diesen Fällen existiert die Absicht den Gegenüber zu kontrollieren und ein langanhaltendes Gefühl von Macht zu verspühren. Dieses Verhalten ist vermehrt bei Männer vorzufinden, die sich in einer ungleichen Beziehung befinden. Um dieses Gefühl von Macht und Kontrolle zu verspüren greifen sie auf entwürdigendes und machtmissbrauchendes Verhalten zurück \parencite{Def_Form_Folge_Gewalt}.

Gewalt lässt sich in drei Kategorien aufteilen, orientiert an der gewaltausübenden Person. Diese Unterteilungen lassen sich in weitere Subgruppierungen teilen. Die \textit{selbstgerichtete Gewalt} beinhaltet suizidales Verhalten, wie Gedanken und Versuche an Suizid, sowie die erfolgreiche Vollendung solcher Versuche. Der zweite Bestandteil dieser Gewalt ist die Selbstmisshandlung \parencite{Gewaltarten_WHO}. Auch die \textit{zwischenmenschliche Gewalt} lässt sich in zwei Unterkategorien aufteilen. Die Gesellschaftsgewalt erfolgt zwischen Personen, die nicht miteinander verwandt sind und die sich möglicherweise nicht kennen und erfolgt demzufolge meist in der Öffentlichkeit. Ein weit bekanntes Beispiel für solhe Gewalt ist die Vergewaltigung oder die sexuelle Nötigung durch Fremde. Sie umfasst aber auch zufällige Gewaltausübungen oder die Gewalt innerhalb Institutionen wie der Schule, Arbeit, Gefängnissen oder Alten- und Pflegeheimen. Die zweite Untergruppe, und die im Rahmen dieser Arbeit wichtigere, ist die Familien- und speziell die Partnerschaftsgewalt \parencite{Gewaltarten_WHO}. Diese Unterkategorie der interpersonellen Gewalt entspricht der bereits zuvor dargeboten häuslichen Gewalt. Die dabei involvierten Personenkonstellationen können Kinder-Eltern, Eltern-Kinder, Geschwister und Partnerschaften sein \parencite{Def_Form_Folge_Gewalt}. Diese Studie fokusiert sich auf die Gewalt innerhalb der Partnerschaft. Die dritte und letzte Kategorie ist die \textit{kollektive Gewalt}, welche sich in soziale, politische und wirtschaftliche Gewalt unterteilen lässt. Diese Unterkategorien zeigen im Gegensatz zu den vorherigen Kategorien mögliche Gründe für die Gewalt durch Staaten oder großen Gruppen an Einzelpersonen. Terroranschläge oder auch Hassangriffe sind Beispiele für eine sozial motivierte kollektive Gewalt. Für politische Gewalt ist Krieg ein sehr prominentes Beispiel. Gewalthandlungen größerer Gruppen mit dem Ziel eines wirtschaftlichen Gewinns sind Bestandteil der wirtschaftlichen Gewalt. Gewalthandlungen durch größere Gruppen können stehts mehrere Motive haben \parencite{Gewaltarten_WHO}.


% GEWALTARTEN
\subsubsection{Gewaltarten}     \label{2.1.2.1}
Wenn man das Wort Gewalt hört, denken die meisten erst an \textit{physische Gewalt}. Doch so wie die Aggression kann sich die Gewalt auf unterschiedliche Arten manifestiert. Eine einheitliche Unterteilung der verschiedenen Gewaltarten ist in der Forschung bislang nicht gegeben. Je nach Schwerpunkt wird auf bis zu fünf Gewaltformen unterschieden. Zu der zu Beginn genannten physischen Gewalt gibt es noch die \textit{psychische, sexualisiert, soziale und ökonomische Gewalt} \parencite{Def_Form_Folge_Gewalt}.

% physische Gewalt
In Kapitel ~\ref{subsec_2.1.1} Aggressivität und Aggression wurde bereits auf die physische Aggression eingegangen und darauf, dass die Abgrenzung von Gewalt und Aggression nicht einfach ist. Denn wie auch die physische Aggression, ist Bestandteil der körperlichen Gewalt jede Form von physischen Angriffen \parencite{ph_G_wie_aggro}. Darunter zählen beispielsweise Tritte, Bisse, Würgen oder Gewaltausübungen mithilfe von Gegenständen. Im Extremfall kann es zur Tötung kommen \parencite{Gewaltart, Def_haus_Gewalt, physische_Gewalt_wie_aggro, Def_Form_Folge_Gewalt}.

% psychische Gewalt
Psychische Gewalt kann durch Worte oder auch durch Gesten und Gesichtsausdrücke erfolgen \parencite{Def_haus_Gewalt_2}. Einige Beispiele für psychische Gewalthandlungen sind Beleidigungen, wie Beschimpfungen und die damit einhergehende Demütigung der betroffenen Person. Oftmals beschädigt die gewalttätige Person Eigentum des Opfers oder behandelt dessen Haustiere nicht artgerecht. Es kann auch dazu kommen, dass die eigenen Kinder genutzt werden, um bei der betroffenen Person Druck auszuüben. Unter psychische Gewalt fallen auch  eifersüchtige Verhaltensweisen, die sich bei der Beendigung einer Beziehung in Stalking umwandeln können \parencite{Def_Form_Folge_Gewalt, Gewaltart, Def_haus_Gewalt_2}. Diese Taten werden verwendet, um den Gegenüber zu manipulieren oder um den eigenen Interssen und Absichten nachzugehen. Sie können auch Verwendung finden, um in böswilliger Absicht das Selbstbewusstsein des Partner zu senken, so das dieser widerstandslosen Gehorsam zeigt \parencite{Def_haus_Gewalt_2}. Diese Art von Gewalt ist nicht so wie die physische Gewalt zwangläufig am Körper sichtbar, ihre negativen Folgen können dennoch tiefer liegen und länger andauern \parencite{psych_Gewalt}. Oftmals ist die psychische Gewalt bloß ein Vorreiter für spätere sexuelle Misshandlungen \parencite{psych_Gewalt_2}. Des Weiteren werden Opfer einer solchen Gewalt von ihrem sozialen Umfeld nicht immer erkannt. Dies kann dazu führen, dass sie an ihrer persönlichen Wahrnehmung zweifeln. Die Reichweite und Intensität der Folgen ist von Person zu Person unterschiedlich und hängt von dessen Vorerfahrungen ab. Laut dem \textcite{Def_Form_Folge_Gewalt} zählt die Forschung die soziale und ökonomische Gewalt zu der psychischen dazu. Im Umfang dieser Arbeit werden diese drei Arten jedoch getrennt betrachtet. 

% sexualisiert Gewalt
Wie im zuvorigen Absatz bereits dargeboten, können psychische Gewalthandlungen, wie Drohungen oder oportune bloßstellende Kommentare, als Vorbote sexualisierter Gewalt genutzt werden \parencite{Übergang_psy_zu_sex_Gewalt}. Des Öfteren geben die Betroffenen auf, sich zu wehren und gehen den Anforderungen und Wünschen des Täters nach. Ein solches Verhalten wird von einigen als möglicher Konsens gedeutet \parencite{Def_haus_Gewalt_2}. Diese Art von Gewalt kann sich in unterschiedlichen sexualisierten Handlungen zeigen, wie die unerwünschte Nähe eines Partners, die Belästigung durch sexuelle Sprüche und Berührungen oder das Darbieten von pornografischen Bildern und Videos. Unter Vergewaltigung wird auch die Nötigung zu sexuellen Handlungen verstanden. Selbst der Versuch fällt unter die sexualisierte Gewalt \parencite{Def_haus_Gewalt_2, Gewaltart, Def_Form_Folge_Gewalt}. Die Zwangsgedanken und das Bedürfnis der Macht sind bei Vergewaltigungen stärker vorhanden, als bei anderen sexuellen Gewalthandlungen. Die ausübende Person ist der Auffassung, dass sie solche Handlungen ausüben muss, damit die Geschlechterhierarchie erhalten bleibt. Sie sind sich demzufolge oftmals ihrer Schuld nicht bewusst \parencite{Def_haus_Gewalt_2}.

% soziale Gewalt
Die soziale Gewalt wird vom \textcite{Def_Form_Folge_Gewalt} als Bestandteil der psychischen Gewalt angesehen. Auch sie ist körperlich nicht erkennbar. Es erfolgt eine soziale Abkoppelung des Opfers von dessn Umfeld. Der Kontakt zu Freunden oder Familien wird untersagt und auch das Treffen bekannter Personen ist sowohl außerhalb wie auch innerhalb des eigenen Zuhauses untersagt. In manchen Fällen sozialer Gewalt, werden die Telefonate durch den Täter mitgehört. Opfer haben oft nicht die Möglichkeit alleine das Haus zu verlassen. Sie werden von ihrem Partner zur und von der Arbeit gebracht \parencite{Def_haus_Gewalt_2, Def_Form_Folge_Gewalt}. Manche Opfer distanzieren sich, aufgrund der psychischen Belastungen, selbst von ihrem sozialen Umfeld. Durch den mangelden sozialen Kontakt ist die Hilfeleistung durch das Umfeld kaum, wenn nicht garnicht möglich \parencite{Def_haus_Gewalt_2}.

% ökonomische Gewalt
Wie auch die soziale Gewalt, wird die ökonomische Gewalt vom \textcite{Def_Form_Folge_Gewalt} als Unterkategorie der psychischen Gewalt gezählt. Müssen sich die Opfer wiederholt Vorwürfe der Fehlerhaften bzw. nicht ausreichenden Fähigkeiten ihren Beruf, den Haushalt oder die Erziehung der eigenen Kinder auszuüben, erzeugt dies große psychische Belastungen \parencite{Def_haus_Gewalt_2}. Es wird ihnen verboten einen Beruf auszuüben, oder sie müssen die Erwerbe an den Partner abgeben, so dass er der einzige ist, der Macht und Kontrolle über die Finanzen hat \parencite{Def_haus_Gewalt_2, Def_Form_Folge_Gewalt}. Dadurch sind die Betroffenen finanziel von ihren Partnern abhängig \parencite{physische_Gewalt_wie_aggro}.Damit die Gewalt von der Öffentlichkeit nicht erkennbar ist, werden den Betroffenen teure Geschenke überreicht. Dies stärkt die Unsicherheit und Bedenken den Partner zu verlassen, weil schwere finanzielle Folgen befürchtet werden \parencite{Übergang_psy_zu_sex_Gewalt}.

In Anlehung and das \textit{Rad der Gewalt} von \textcite{Rad_der_Gewalt} sind in Anhang A in Abbildung~\ref{Rad der Gewalt} die fünf Gewaltarten nochmal zusammenfassend bildlich dargestellt. %how to reference Anhang?

% URSACHE + KREISLAUF DER GEWALT
\subsubsection{Ursachen und Aufrechterhaltung von Gewalt}     \label{2.1.2.2}
Wie bei viele Situationen und Krankheiten, besteht bei der häusliche Gewalt, oder im Rahmen dieser Arbeit die Partnerschaftsgewalt eine Multikausalität. Dies bedeutet, dass nicht ein einziger Faktor als Ursache zu determinieren ist. Es ist vielmehr die Interaktion und Verwobenheit vieler unterschiedlicher Ursachen auf unterschiedlichen Ebenen \parencite{Ursache_hG}. Ein ökologisches Modell versucht anhand von vier Ebenen die Entstehung von Partnerschaftsgewalt zu systematisieren. Die unterschiedlichen Faktoren auf den jeweiligen Ebenen stehen im Zusammenhang miteinander und untereinander und bedingen sich somit gegenseitig. Durch die gemeinsame Interaktion können sie die Auftretenswahrscheinlichkeit von häuslicher Gewalt bevorzugen \parencite{Ursache_hG_2, Ursache_hG, Gewaltart}.
Auf der kleinsten Ebene steht das Individuum, dessen Verhalten durch persönliche, biologische und entwicklungsbedingte Faktoren beeinflusst wird. Ausschlaggebend sind hierbei eigene Erfahrungen mit Misshandlungen, sowie exessiver Konsum von Suchtmitteln \parencite{Ursache_hG_2, Ursache_hG, Gewaltart}. Das Verhalten der Personen ist laut \textcite{Gewaltart} sowie \textcite{Ursache_hG_2} ebenfalls durch psychische Störungen und Störungen der Persönlichkeit geprägt. Aber auch nicht klinische Eigenschaften, wie das Selbstwertgefühl oder die Stressregulationsfähigkeiten, haben einen bedeutsamen Einfluss auf menschliches Verhalten auf der individuellen Ebene \parencite{Ursache_hG}.
Eine Instanz höher, auf der Beziehungsebene, beschäftigt sich die Forschung mit der zwischenmenschlichen Interaktion naher Beziehungen. Hierbei wird die Art und Weise betrachtet, wie Partner im Austausch zueinander stehen, wie die Macht zwischen den Partnern verteilt ist und wie sie mit Konflikten unterschiedlichen Ausmaßes innerhalb der Beziehung umgehen \parencite{Ursache_hG_2, Ursache_hG, Gewaltart}.
Auf der Ebene der Gesellschaft wird das soziale und räumliche Umfeld, wie Verwandte, Freunde, Nachbarn, oder der Arbeitsplatz und Vereine betrachtet. Die Forschung betrachtet Aspekte wie die soziale Isolation oder ein gewalttolerierendes und unterstützendes Mileau \parencite{Ursache_hG_2, Ursache_hG, Gewaltart}. Die Nachbarschaft hat auch mit ihrer Rate an nicht Erwerbstätigen und durch mögliche Drogengeschäfte einen großen Einfluss auf das Verhalten der Beziehungspartner \parencite{Ursache_hG, Gewaltart}.
Die höchste Ebene ist die der Gesellschaft. Durch ihre sozialen und kulturellen Normen schafft sie ein gewaltförderndes oder -hinterndes Umfeld \parencite{Ursache_hG_2, Ursache_hG, Gewaltart}. Die positive bzw. negative Auffassung von Gewalt in der Gesellschaft generell, aber vor allem in der Politik, im Justizsystem und in den Medien beeinflusst das menschliche Verhalten \parencite{Ursache_hG_2, Ursache_hG}.  

Häufig entwickelt sich häusliche Gewalt  zu einem immer wiederkehrenden Kreislauf. Dieser beinhaltet die folgenden drei zusammenhängenden Zyklen:
\begin{itemize}
    \item Spannungsaufbau
    \item Gewaltausbruch
    \item Reue, Entschuldigungs- und Entlastungsversuche
\end{itemize}
Bei wiederholter Durchführung nimmt die zweite Phase an Intensität zu und tritt vermehrt auf. Da sich die entlastende Phase der Entschuldigung und Reue dabei verringert, fällt es den Opfern häuslicher Gewalt schwer, sich aus einer solchen Beziehung zu entfernen \parencite{Def_haus_Gewalt}.


\subsubsection{Folgen von Gewalt}     \label{2.1.2.3}
physische, psychische, soziale, finanzielle, gesellschaftliche (1) 
\parencite{Gewaltart}
Folgen in all ihren Dimensionen erfassen ist kaum möglich 
direkt, indirekt, kurzfristig, langfristig, chronifiziert
häusliche Gewalt = einer der gräßten Risikofaktoren für Gesundheit
je stärker Gewalterfahrung, desto häufiger gesundheitliche Beschwerden
physisch:
    Blaue Flecken, offene Wunden, Knochenbrüche, Bewusstlosigkeit, Komplikationen während Schwangerschaft (= direkte Folgen)
    körperliche Folgen haben Einfluss auf Lebensbereiche, wie Arbeit
    Selbstmedikation
    chronisch, wenn keine zeitnahe (ärztliche) Behandlung wahrgenommen
    (indirekte Spätfolgen) Schmerzsyndrome, Herzkreislaufbeschwerden, Beschwerden Bewegungsapparats und Atemwege
    Spätfolgen erklärbar durch andauernden Stressfaktor und somit negativ auf Gesundheit auswirken 
    schlimmster Ausmaß: Mod und Tötungsdelikte
psychisch:
    Angstgefühle, geringes Selbstwertgefühl, Niedergeschlagenheit, Depression, Scham- oder Schuldgefühle, Lustlosigkeit, Schlafstörungen, Konzentrationsschwierigkeiten Essstörungen, Selbstverletzun, Schwierigkeiten in der Beziehung zu Männern und in der Arbeit
    können zur chronifizierten PTBS oder Persönlichkeitsstörung führen
    Gewaltbetroffene Frauen weitaus häufiger Depression, Agststörung und Phobien
    Andere Strategien: Alkohol, Nikotin, Substanzmittel -> Selbstmedikation als Art innere Flucht
    Konsum hat wiederum negative Auswirkung auf körperliche und seelische Befinden
    Suizidalität höher: 10.7 Prozent sagen, haben versucht Suizid zu begehen \parencite{psy_Folgen_hG}
soziale:
    Opfer schämen oftmals und trauen sich nicht mit Bezugsperson über ihre Gewalterfahrungen zu sprechen oder Hilfe zu suchen
    Folge: von (helfenden) Umfeld zurückziehen, soziale Isolation
    Zerstörung familiären uns sozialen Strukturen
    durch Trennung ganze Familien (mit Kindern) auseinandergerissen
    Folge: Umzug, Wohnungslosigkeit, aus bestehenden sozialen Netzwerk herausgerissen
    Einfluss auf zukünftige Beziehungen
    oftmals Mühe, neue Partnerschaft einzugehen und wieder Vertrauen zu fassen
    Schwierigkeiten im Umgang mit Männern oder eigenen Sexualität
finanziel:
    niedriges Selbstwertgefühl -> Arbeit nicht mehr ausführen können
    oftmals Gefühl anfallende Erfordernisse nicht meistern können
    Verlust Arbeitsstelle bzw. längerfristige Erwerbslösigkeit -> psych./ phys. Probleme und soziale Isolation
    häusliche Gewalt = Armutsrisiko und sozialer Abstieg

gesundheitliche Folgen und Kosten (2)
\parencite{Def_haus_Gewalt}
zeitnah, langfristig
physisch, psychisch, psychosomatische Beschwerden, Tod
Auswirkungen können länger andeuern, auch wenn Misshandlung bereits beendet
Kosten erwartungsgemäß schwierig zu ermitteln
gesellschaftliche Kosten: in sozialen, juristischen Bereich, Gesundheits- und Bildungssektor
Erwerbsleben, Arbeitsunfähigkeit, Arbeitslosigkeit und Frühberentung
Folgekosten durch frühes Erkennen und adäquates Behandeln langfristig senken

gesundheitliche, sozioökonomische, psychosoziale (3)
\parencite{Def_haus_Gewalt_2}
gesundheitlich:
    kurz-, mittel, langfristige Folgen
    körperliche Folgen meist schneller sichtbar, jedoch psychische weitaus schwerwiegender
psychische:
    Angststörungen oder Depression
    mangeldes Selbstwertgefühl, Probleme im Umgang mit anderen Menschen insbesondere männliche Personen, Scham- sowie Schuldgefühle, Suchterkrankungen, PTBS, Suizidgedanken
    Selbstwert geschwächt -> sind Auffassung: durch Gewalt sind als Person weniger wert
    Folgen schlimmer je stärker Isolation
physisch:
    direkte körperliche Verletzungen, physische sowie psychosomatische Symptome (Magen-Darm-Probleme, nervöses Zucken)
    Substanzmittelkonsum, weil negative Gefühle von Angst und Hilflosigkeit mindern -> Selbstmedikation als Bewältigungsstrategie
sozioökonomisch:
    Schwierigkeiten am Arbeitsplatz
    durch steigende Stresssituation -> Anforderungen von Arbeitsalltag nicht mehr erfüllen
    Probleme sichtbar durch: Unpünktlichkeit, Krankenstände, Abwesenheit, geringe Arbeitsbelastung
    häufig Arbeitsplatzverlust, ständiger Arbeitsplatzwechsel
    häufig sozialer Abstieg
psychosozial:
    Kontakt zu engsten Vertrauten verlieren und somit kein Unterstützungssystem
    Probleme Hinblick auf neue Freundschaften und Beziehungen
    schwer jemandem gegenüber zu öffnen und Vertrauen aufzubauen
    schämen häufig und haben große Ängste und Wutgefühle, welche nicht mehr steuern können  -> machnmal ziehen selbst aus Netzwerk zurück

Folgen für Betroffene (4)
\parencite{Def_Form_Folge_Gewalt}
gesundheitlich:
    nicht nur direkt betroffene Opfer
    traumatisierend, versetzen in extreme Angst und Hilflosigkeit und überfordern normalen Anpassungs- und Bewältigungsstrategie
unmittelbar:
    körperlich: Prellungen, Verstauchungen, Hirnschütterung, Frakturen, innere Verletzungen
    auch unmittelbar mit psychischen Folgeproblemen: Leistungs- und Konzentrationsschwierigkeiten, erhöhte Medikamenten- und Alkoholkonsum
mittel- und langfristig:
    breites Sprektrum somatischer, psychosomatischer und psychischer Gesundheitsbelastung
    gynäkologische Beschwerden, Herz-Kreislauf-Beschwerden
    Depression, Stresssymptome, Angststörungen, PTBS, Essstörungen, Suizidalität
    gesundheitsgefährdende Bewältigungsstrategie
Art, Tragweite und Merkmale:
    Auswirkung von verschiedenen Faktoren beeinflusst: individuelle Vorraussetzungen, Form erlebter Gewalt, Verhältnis zur Tatperson
    psychische Gewalt längerfristig weit gravierendere Belastung als körperliche    -> psychische Langzeitfolgen belasten mehr
sozialer Bereich und Erwerbsleben
    Trennung, Scheidung, Auszug, Wegzug, Wechsel der Arbeitsplatzes, Schulwechsel   -> erhebliche Neuorientierung
    unmittelbar oder längerfristig kann auf Erwerbsleben auswirken
    Arbeitsunfähigkeit, Krankheitsabsenenz, Leistungsbußen


%- Folgen: körperlich; gesundheitsgefährdende (Überlebens-) Strategien; (Psycho-)somatische; reproduktive Gesundheit; psychische 
%- Istambul Konvention: was ist das; welche Aspekte; Artikel 11 und 13 (Partnerschaft Gewalt2020)



\subsection{Gewaltmythen}   \label{subsec_2.1.3}
hier auch vicitim blaming eingehen

% 2.2 Theoretischer Hintergrund: Aktueller Forschungsstand
\section{Aktueller Forschungsstand}   \label{sec_2.2}
In dieser Studie wird folgende Frage untersucht: Welchen Zusammenhang gibt es zwischen Aggressivität und der Akzeptanz von Gewaltmythen, sowie der Tendez zum victim blaming? (mögliche Quelle, die das unterschützt) 

In den darauffolgenden Unterkapiteln ~\ref{subsec_2.2.1} Hypothese 1, ~\ref{subsec_2.2.2} Hypothese 2 und ~\ref{subsec_2.2.3} Hypothese 3 erfolgt die Herleitung der zu untersuchenden Hypothesen auf Basis bereits bestehender Befunde.

\subsection{Hypothese 1}  \label{subsec_2.2.1}
In einer Studie mit $N$~=~1177 Universitätsstudenten, hypothetisiertern \textcite{H1_1993}, dass Männer, die in ihrer Vergangenheit sexuelle Aggression zeigten, dazu neigen, Opfern die Veranwortung zuzuschreiben. Mythen über häusliche Gewalt schließen Gewalt sexueller Form ein \parencite{H1_Poli_2022}.Im Rahmen dieser Arbeit wird sexualisierte Gewalt untersucht. Aggressives Agieren in einem Umfeld erhöht die Wahrscheinlichkeit, Aggression und das damit verbundene aggressive Verhalten in anderen Situationen darzustellen. Die Frage ist, ob ein höheres Maß an Aggression allgemein mit Victim Blaming in Verbindung steht.

Die in 2012 durchgeführt Studie von Kassim untersuchte unter anderem die Korrelation zwischen Victim Blaming und dem Anteil an aggressivem Verhalten bei malaysischen Jugendlichen. Sie kam mit einem $r$~=~.29 auf ein hoch signifikantes Ergebnis ($p$~=~.01) \parencite{H1_malasia_2012}. In ihrer theoretischen Aufarbeitung ihrer untersuchten Konstrukte eklärte Kassim, dass das Erleben von häuslicher Gewalt zur Akzeptanz der Mythen häuslicher Gewalt führt. Sie kam zur Erkenntnis, dass nicht nur die Verantwortungszuschreibung auf das Opfer, aber auch das Erleben von häuslicher Gewalt aggressives Verhalten aufklären \parencite{H1_malasia_2012}.

\textcite{H1_moderation_2020} untersuchten in ihrer Studie Aggression, häusliche Gewalt und Toleranz bei pakistanischen arbeitstätigen und verheirateten Frauen. Häusliche Gewalt wurde, wie in dieser vorliegenden Studie, mithilfe der Domestic Violence Myth Acceptance Scale (DVMAS; \textcite{Peters2003}) untersucht. \textcite{H1_moderation_2020} testeten zudem auch die Korrelation der Victim Blaming behandelnden DVMAS$-$Subskalen mit Aggression. Konträr zu \textcite{H1_malasia_2012} kamen sie jedoch zu den Ergebnissen, dass Aggression nicht mit Victim Blaming, im Sinne der Opferbeschuldigung aufgrund des Charakters und des Verhaltens und die Entschuldigung der Täter, korreliert ($r$~=~.04, $r$~=~.13, $r$~=~.11). Im Umfang einer Moderationsanalyse dieser Variablen mit einer Subskala der Toleranz kamen \textcite{H1_moderation_2020} zu dem signifikanten Ergebnis, dass die Interaktion zwischen dieser Subskala und der Opferbeschuldigung aufgrund des Charakters Aggression negativ vorhersagen (\textbeta~=~$-$.225, $p<$ .01). Toreranz scheint das Ausmaß des Victim Blaming bei gegebener Aggression zu mindern.

Die Studien von \textcite{H1_malasia_2012, H1_moderation_2020} beziehen sich auf den nahöstlichen und asiatischen Raum. Demzufolge können sich die Ergebnisse aus diesen Studien von der Diesigen unterscheiden. Des Weiteren bilden bei der Studie von \textcite{H1_malasia_2012} Jugendliche die Stichprobe. In der vorliegenden Studie werden ausschließlich Probanden herangezogen, die die Volljährigkeit erreicht haben. Nichtsdestotrotz wurde diese Studie als Beispiel des aktuellen Forschungsstandes berücksichtig. Durch die Reifung des präfrontalen Kortex nimmt während des Jugendalters die Intelligenz und das logische Denken zu. Auch die Frage der eigenen Identität steht in den späteren Jugendjahren im Vordergrund \parencite{H1_Entwicklung}. In diesen Jahren formen sich große Teile des späteren Selbst und die hier erhobene Stichprobe zeigt ihre größte Anhäufung an Probanden bei einem Alter von 21 Jahren (vgl. Abbildung~\ref{Histogramm Altersverteilung}). Durch diese Nähe an das Jugendalter wird die Studie von \textcite{H1_malasia_2012} dennoch für die Grundlage meiner Hypothese berücksichtig.

Es lässt sich vermuten, dass, wie zu Beginn dieses Kapitels angedeutet, das Ausmaß an Aggression einer Person mit dessen Tendenz zur Verantwortungszuschreibung gegenüber des Opfers zusammenhängen. Daraus bildet sich die ungerichtete Zusammenhangshypothese: Der Aggressionsscore korreliert mit Victim Blaming.
\subsection{Hypothese 2}    \label{subsec_2.2.2}
Bislang gibt es wenig Studien, die die Relation von Aggression und der Akzeptanz der Mythen häuslicher Gewalt untersucht. Es lässt sich jedoch vermuten, dass sie zusammenhängen, zumal der DVMAS mit patriachalen und sexistischen Einstellungen verbunden ist. Glaubende dieser Sichtweisen scheuen von der Anwendung von aggressiven Handlungen und Gewalt zur Behebung von Konflikten nicht zurück \parencite{DVMAS_Peters}. Obwohl Gewalt nicht gleichzusetzen ist mit Aggression, ist es auch nicht möglich zu sagen, dass diese beiden Begriffe nicht mit einander in Verbindung stehen.
Erst in jüngeren Jahren scheint die Korrelation von Aggression und Akzeptanz der Mythen häuslicher Gewalt an Interesse gewonnen zu haben. 

Die Studie von \textcite{H2_u_3_Bhogal_2016} mit $N$~=~121 Probanden untersuchte ob physische und verbale Aggression wie auch Ärger und Misstrauen die Akzeptanz der Vergewaltigungsmythen vorhersagen können. Die Ergebnisse zeigen, dass physische Aggression die Akzeptanz signifikant vorhersagt.

In einer Studie mit 100 verheirateten und berufstätigen Frauen untersuchten \textcite{H1_moderation_2020} unter anderem eine mögliche moderierenden Rolle der Toleranz$-$Subskala Neuheit zwischen häuslicher Gewalt und Aggression. Sie kamen zu der Erkenntnis, dass die Akzeptanz der Mythen häuslicher Gewalt ein signifikanter Prediktor von Aggression ist (\textbeta~=~.734, $p<$ .01).

Aufgrund des theoretischen Hintergrundes von \textcite{DVMAS_Peters} und den Ergebnissen der Studien von \textcite{H1_moderation_2020, H2_u_3_Bhogal_2016} bildet sich der Gedanke, dass Aggression und die Akzeptanz der Mythen häuslicher Gewalt häuslicher Gewalt positiv korrelieren. Daraus ergibt sich folgende gerichtete Zusammenhangshypothese: Der Aggressionsscore korreliert positiv mit der Akzeptanz der Mythen häuslicher Gewalt



\subsection{Hypothese 3}    \label{subsec_2.2.3}
\textcite{H2_u_3_Bhogal_2016} untersuchten in ihrer Studie 121 Studierende auf ihre Akzeptanz von Vergewaltigungsmythen und ihre Aggression in Form von physischer und verbaler Aggression, Ärger und Misstrauen. Sie kamen zum Entschluss, dass Männer eine höhere Akzeptanz von Vergewaltigungsmythen haben. Diese Mythen stellen die, meist weiblichen, als Mitschuldie da. Diese sichtweise des Opfers ist im Einklang mit den Mythen von häuslicher Gewalt. Ein weiteres Ergebnis der Untersuchung von \textcite{H2_u_3_Bhogal_2016} war, dass die physische Aggression bei Männern signifikant höher ist, verglichen mit den Frauen.

Laut \textcite{H3_MFUnterschied} sind Feministen der Meinung, dass Frauen ihre Aggressivität unterdrücken, gleichzeitig sind biologisch Positionierte der Auffassung, dass Frauen nicht die gleichen Fähigkeiten bzw. nicht das gleiche Bedürfniss haben, zu reagieren, so wie Männer es haben. Eine weitere Erklärung geht davon aus, dass Frauen das selbe biologische Potenzial für Aggressivität aufweisen, aber die Gesellschaft solch ein Verhalten ausschließlich bei Männern fördert. Die Verhaltensbiologie betrachtend, sind Männchen auf Grund von Testostron aggressiver als Weibchen. Obwohl Artenübergreifende Vergleiche mit Obhut zu genießen sind, bietet diese Tatsache einen starken Anhaltspunkt für einen Geschlechterunterschied menschlicher Aggressivität.

Im Rahmen einer Studie von $N$~=~329 Probanden zwischen 15 und 19 Jahren untersuchten \textcite{H3_2020} die Existenz sexualisierter Gewalt in romantischen Beziehungen, mögliche Zusammenhänge zwischen Mythen über sexualisierte Aggression und sexualisiertes Durchsetzungsvermögen und ihre möglichen geschlechterspezifischen Unterschiede. Ihre Ergebnisse zeigen, dass die männlichen Jugendlichen häufiger Täter sexualisierter Gewalt waren, und dass die männlichen Probanden vermehrt Mythen über sexualisierte Aggression Glauben schenkten. Das behandelte Thema des verwendeten Fragebogens ähnelt denen des DVMAS.

Aufgrund von \textcite{H2_u_3_Bhogal_2016, H3_MFUnterschied, H3_2020} liegt die Vermutung nahe, dass das Geschecht der Probanden eine moderierenden Rolle im Zusammenhang zwischen Akzeptanz von Gewaltmythen und Aggression hat. Aus diesem Gedanke ergibt sich die folgende Hypothese: Der Zusammenhang zwischen Akzeptanz von Gewaltmythen und Aggression wird durch das Geschlecht moderiert.

% Durch die in Kapitel 1.4 jeweiligen Studien von Haj-Yahia (2003), McElligott (2011), George und Martínez (2002) und Acosved und Long (2006) liegt der Gedanken nahe, dass der kulturelle Hintergrund der Betroffenen Auswirkungen auf die Wahrnehmung und daraus resultierende Verantwortungszuschreibung der Gesellschaft hat. Daraus ergibt sich die gerichtete Unterschiedshypothese:

\chapter{Methoden}   \label{ch_3}
Empirische Prüfung der empirischen Hypthese(n). Die Informationen werden in der Vergangenheit geschrieben.

\section{Stichprobenbeschreibung} \label{sec_3.1}
Rekrutierung und Eigenschaften der Stichprobe


\section{Untersuchungsdesign}  \label{sec_3.2}
Präregistrierung \\ %im Rahmen dessen gibt es weitere 4 Studien und NUR nennen was die erhoben
Feld-Laborstudie\\
genutzte Methode \\
Design \\
warum Online gemacht \\
effekte\\
randomiesiert und warum


\section{Operationalisierung der Konstrukte}    \label{sec_3.3}
anzahl FB + Vignette\\ % wieviel Items 2 Bsp. Skala, was erheben die; Reliabilität nennen (bei DVMAS da englische version nehmen für Validität)
Deutscher Aggressionsfragebogen: \textbf{Reliabilität} (Cronbachs-Alpha: .62-.82) Retest-Reliabilität (C-A: .73), \textbf{Validität} (Die differentielle Validität wurde über die Korrelationen der Antworten zu den vier Subskalen mit denen zu weiteren ausgewählten Konstrukten bestimmt. Die Subskalen wurden dabei nach dem 4-Faktor Modell von Buss und Perry gebildet und nicht nach den berichteten faktorenanalytischen Ergebnissen für die deutsche Version. Dieses Vorgehen wurde primär gewählt, um einen Vergleich mit anderen Untersuchungen zu erleichtern, aber auch, weil Daten zur Validierung hauptsächlich nur aus Stichprobe 1 vorliegen, der Item 29 nicht vorgegeben wurde.

In Stichprobe 1 wurden dazu jeweils einer Teilgruppe der Befragten Itembatterien zur Erfassung der folgenden Konstrukte ebenfalls vorgelegt:
\begin{itemize}
    \item generalisierter Selbstwert (Rosenberg-Skala; siehe v. Collani \& Herzberg, 2003)
    \item Skala Aggressivität aus dem FPI-R (Fahrenberg, Hampel \& Selg, 1994)
    \item drei Teilskalen des STAXI (Ärger-In, Ärger-Out, Ärger-Kontrolle; Schwenkenberger et al., 1992)
    \item NEO-FFI (Borkenau \& Ostendorf, 1991)
\end{itemize}

Stichprobe 2 beantwortete neben dem Aggressionsfragebogen noch die Kurzform einer Narzissmusskala (Raskin \& Terry, 1987) mit 10 Items.

Mit der Aggressionsskala des FPI (Tabelle 5) ergeben sich erwartungsgemäß hohe korrelative Zusammenhänge zu allen Subskalen des deutschen Aggressionsfragebogens, mit Ausnahme der Subskala Misstrauen. Diese korreliert jedoch hoch negativ mit dem generalisierten Selbstwert (Tabelle 5) nach Rosenberg und hoch positiv mit der nach innen gerichteten Ärgerkomponente aus dem STAXI. Der nach außen gerichtete Ärger ist mit allen drei Komponenten des deutschen Aggressionsfragebogens assoziiert, außer mit der Komponente Misstrauen. Die Subskala Misstrauen erfasst also offenbar im Unterschied zu den drei anderen Subskalen noch weitere Komponenten einer Aggressionsbereitschaft. Die differentielle Validität der Subskalen des deutschen Aggressionsfragebogens wird ferner durch einen hohen negativen Zusammenhang der Subskala Ärger mit Ärgerkontrolle nach dem STAXI (Tabelle 5) belegt und durch Zusammenhänge zwischen Misstrauen sowie Ärger mit den Werten für die Neurotizismusskala des NEO-FFI.)
was wurde noch alles für die andren Arbeit erhoben \\ 
% 2 Bsp-Vignetten am besten so viele Faktoren
die 3 Gütekriterien
Manipulationscheck


\section{Untersuchungsdurchführung}   \label{sec_3.4}
Zeitraum der Befragung \\
verteilung von FB \\
Bearbeitungszeit und VPN \\
Was wichtig für einleitungstext \\
sozio erhoben \\
potentielle störvariablen (wegen online nicht kontrollierbar)


\section{Auswertungsmethode}    \label{sec_3.5}
SPSS ausgewertet \\ %was umgepolt wurde: Variablen wurden so umkodiert, dass der betroffenen Person stets der Zahlenwert 101 sprich der vollen Verantwortung zugeschrieben wurde (GP13,14,17    GS2,5,8) + Aggro 14,22
% Vorbereitung der Daten: wieviele wurden rausgeschmissen
deskriptive \\ % Lage, Streuung, Median, Modus, Mittelwert, SD
hypothesen \\ % das aus präregistrierung: gerichtet, unterschied, tests
% H3, weil Moderatorviariable hat wurde auf SPSS mit dem plug-in PROCESS berechnet
erwähnen, dass Vorraussetzung gibt und werden später geprüft

auf 1-2 items eingehen wenn über FB geschrieben wird
\chapter{Ergebnisse}   \label{ch_4}
Im folgenden Kapitel werden die durchgeführten statistischen Berechnungen dargestellt. Zu Beginn werden die deskriptiven Ergebnisse beschrieben und mit den Normwerten verglichen, woraufhin die statistische Überprüfung der Manipulation folgt. Im inferenzstatistischen Unterkapitel wird mit der Prüfung der Vorraussetzung der einzelnen Test mathematisch dargestellt und abschließend werden die Ergebnisse der drei Hypothesen präsentiert. Die ausführliche Interpretation der Ergebnisse geschieht im nachkommenden Kapitel ~\ref{ch_5}.

\section{Deskriptive Ergebnisse}    \label{sec_4.1}
Die Fallvignetten wießen, bei einem $Min$~=~1 und einem $Max$~=~101, einen Mittelwert von $M$~=~26.46 ($SD$~=~27.98) auf. Der Median lag bei $Mdn$~=~19. Die am häufigsten verteilten Werte liegen an den beiden Extremen bei 1.00 und 101.00 und bilden eine bimodale Verteilung. Bei einer Schiefe von 1.17 und Kurtosis von 0.53, liegt eine rechstschiefe bzw. linkssteile und steilgipflige Verteilung vor. Abbildung~\ref{Histogramm VicBlame} stellt die unimodale Verteilung der Verantwortungszuschreibung bildlich da. Die vorhandenen Ausreißer befinden sich alle unter der Ausscheidungsgrenze.
% Modus: 1.00 und 101.00
\begin{figure}[htb]
    \centering
        \includegraphics[width=0.8\linewidth]{Histogramm - VicBlame.png}
        \caption[Histogramm Altersverteilung]{Verteilung der Verantwortungszuschreibung.}
        \label{Histogramm VicBlame}
\end{figure}



Der DVMAS zeigte einen Mittelwert von $M$~=~2.60 ($SD$~=~0.80) und einen Median von $Mdn$~=~2.5. Der geringste Wert war dabei $Min$~=~1 und der höchsete $Max$~=~5.56. Die häufigsten Werte lagen bei 2.11 und 2.67 und bilden eine bimodale Verteilung. Eine Schiefe von 0.60 bildet eine leicht rechstschiefe Verteilung. Die Kurtosis von 0.09 bildet eine leicht steilere Verteilung, als die Normalverteilung. Beide Werte weichen demzufolge leicht von einer Normalverteilung ab ($M$~=~2.30, $SD$~=~0.85 und Schiefe~=~0.63). 
Die Abbildung~\ref{Histogramm DVMAS} bildet die beschriebenen Werte dieser Studie bildlich ab. Die vorhandenen Ausreißer befinden sich alle unter der Ausscheidungsgrenze.
% Modus: 2.11, 2.67
\begin{figure}[htb]
    \centering
        \includegraphics[width=0.8\linewidth]{Histogramm - DVMAS.png}
        \caption[Histogramm DVMAS]{Verteilung der Akzeptanz von Gewaltmythen.}
        \label{Histogramm DVMAS}
\end{figure}



\begin{figure}[htb!]
    \centering
        \includegraphics[width=0.8\linewidth]{Histogramm - AggroFB.png}
        \caption[Histogramm Aggressionsfragebogen]{Verteilung der Angaben des Aggressionsfragebogens.}
        \label{Histogramm AggroFB}
\end{figure}


Beim Deutschen Aggressionsfragebogen wurde ein Mittelwert von $M$~=~1.88 ($SD$~=~0.43) und ein Median von $Mdn$~=~1.79 berechnet. Der geringste angegebene Wert betrug $Min$~=~1.10 und der höchste $Max$~=~3.52. Die Verteilung wies einen Schiefe von 0.94 und eine Kurtosis von 0.88 auf. Demzufolge ist die Verteilung rechstschiefe und hat eine breitgipflige Form, sowie eine Multimodalität mit den Modiwerten bei 1.59, 1.62, 2.52 und 2.55. Die Lageparameter weichen stark von den Normwerten ab. Diese sind wie folgt: $M$~=~2.66 ($SD$~=~0.88), Schiefe~=~3.81 und eine Kurtosis von -0.06. In Abbildung~\ref{Histogramm AggroFB} ist eine bildliche Darstellung der erhobenen Verteilung zu sehen. Die vorhandenen Ausreißer befinden sich alle unter der Ausscheidungsgrenze.
% Modus: 1.59, 1.62, 2.52, 2.55


\section{Manipulationscheck}    \label{sec_4.2}
Nachfolgend werden die Ergebnisse der vier Manipulationschecks in der folgenden Reihenfolge aufgeführt: Gewaltart, Geschlecht des Opfers, soziökonomischer Status der betroffenen Person und abschließend der Kulturelle Hintergrund.
Wie zuvor erklärt, musste der Datensatz verdoppelt werden, um eine berechnung mit den Daten gewährleisten zu können. Aus diesem Grund wird in diesem und den nachfolgenden Kapiteln von einer Stichprobengröße von $N_{neu}$~=~864 ausgegangen.

\begin{table}[htb]
    \caption[Kreuztabelle Manipulationscheck Gewaltart]{\textit {Kreuztabelle des Manipulationschecks der Gewaltart}} 
    \label{KT_G}
    \centering
    \begin{adjustbox}{width=10cm} %{width=\textwidth}
    \small
    \begin{tabular}{lrrr}
      \hline
        &   & psychische Gewalt & sexualisierte Gewalt \\
      \hline
    Ja   & Anzahl  & 32      & 354     \\
         & Prozent & 8.30\%  & 91.70\% \\
    Nein & Anzahl  & 400     & 78      \\
         & Prozent & 83.70\% & 16.30\% \\
       \hline
    \end{tabular}
    \end{adjustbox}
    
    \begin{tablenotes}
        \item \textit{Anmerkungen.} \( N \) = 864. Prüffrage: Ging es um sexualisierte Gewalt?
      \end{tablenotes}
    \end{table}
\begin{table}[htb]
    \caption[Kreuztabelle Manipulationscheck Opfergeschlecht]{\textit {Kreuztabelle des Manipulationschecks des Geschlecht der betroffenen Person}} 
    \label{KT_sex}
    \centering
    \begin{adjustbox}{width=10cm} %{width=\textwidth}
    \small
    \begin{tabular}{lrrr}
      \hline
        &   & weibliches Opfer & männliches Opfer \\
      \hline
    Ja   & Anzahl  & 411      & 14      \\
    & Prozent & 96.70\%  & 3.30\%  \\
    Nein & Anzahl  & 16       & 423     \\
    & Prozent & 3.60\%   & 96.40\% \\
       \hline
    \end{tabular}
    \end{adjustbox}
    
    \begin{tablenotes}
        \item \textit{Anmerkungen.} \( N_{neu} \)~=~864. Prüffrage: War das Opfer eine Frau?
      \end{tablenotes}
    \end{table}



Bei der Überprüfung der Gewaltart gaben bei Vignetten sexualisierter Gewalt 354 Personen (91.70\%) an eine solche Gewaltart behandelt zu haben. Handelte es sich um eine psychische Gewaltart an, reichten nur 400 Personen (83.70\%) die richtige Antwort ein. Der durchgeführte Chi$^2$-Test viel mit einem Wert von 485.53 signifikant aus ($p<$.001). 
In Tabelle~\ref{KT_G} sind die Werte dieser Manipulationsprüfung nochmals zusammengefasst.

War die Nachfrage auf das Geschlecht des Opfers gerichtet gaben die 411 (96.70\%) bzw. 423 (96.40\%) Probanden an, es handelte sich um ein weibliches bzw. männliches Opfer. Auch dieser Chi$^2$-Test war mit einem Wert von 748.16 hoch signifikant ($p<$.001). 
In Tabelle~\ref{KT_sex} sind die Werte dieser Manipulationsprüfung nochmals zusammengefasst.

Die Überprüfung des niedrigen soziökonomischen Status des Opfers wurde von 389 (80.2\%) bei dessen Gegebenheit richtig beanwortet. Bei der gegenteiligen Situation stieg die Rate der richtigen Antworten auf 342 (90.20\%). Mit einem Wert von 422.37 fiel dieser Chi$^2$-Test ebenfalls signifikant aus ($p<$.001). 
In Tabelle~\ref{KT_SES} sind die Werte dieser Manipulationsprüfung nochmals zusammengefasst.
\begin{table}[htb]
    \caption[Kreuztabelle Manipulationscheck soziökonomischer Status des Opfers]{\textit {Kreuztabelle des Manipulationschecks des soziökonomischen Status der betroffenen Person}} 
    \label{KT_SES}
    \centering
    \begin{adjustbox}{width=10cm} %{width=\textwidth}
    \small
    \begin{tabular}{lrrr}
      \hline
        &   & niedriger SES Opfer & hoher SES Opfer \\
      \hline
    Ja   & Anzahl  & 389      & 96      \\
         & Prozent & 80.20\%  & 19.8\%  \\
    Nein & Anzahl  & 37       & 342     \\
         & Prozent & 9.80\%   & 90.20\% \\
       \hline
    \end{tabular}
    \end{adjustbox}
    
    \begin{tablenotes}
        \item \textit{Anmerkungen.} \( N_{neu} \)~=~864. SES = soziökonomischer Status. Prüffrage: \enquote{War die finanzielle Situation des Opfers schlechter als die finanzielle Situation des Täters?}.
      \end{tablenotes}
    \end{table}

\begin{table}[htb]
    \caption[Kreuztabelle Manipulationscheck kultureller Status]{\textit {Kreuztabelle des Manipulationschecks des kulturellen Status}} 
    \label{KT_kult}
    \centering
    \begin{adjustbox}{width=6.5cm} %{width=\textwidth}
    \small
    \begin{tabular}{lrrr}
      \hline
        &   & arabisch & deutsch \\
      \hline
    Ja   & Anzahl  & 10      & 422      \\
         & Prozent & 2.30\%  & 97.70\%  \\
    Nein & Anzahl  & 423     & 9        \\
         & Prozent & 97.90\% & 2.10\%   \\
       \hline
    \end{tabular}
    \end{adjustbox}
    
    \begin{tablenotes}
        \item \textit{Anmerkungen.} \( N_{neu} \)~=~864. Prüffrage: Hatten die Personen deutsche Namen?
      \end{tablenotes}
    \end{table}
Die höchste Rate an Richtigkeit gab es bei dem Manipulationscheck des kulturellen Status. Hier gaben 422 (97.70\%) Personen bzw. 423 (97.90\%) die richtige Antwort. Auch dieser Chi$^2$-Test ist mit einem Wert von 789.68 signifikant ($p<$.001). 
In Tabelle~\ref{KT_kult} sind die Werte dieser Manipulationsprüfung nochmals zusammengefasst.





\section{Inferenzstatistische Ergebnisse}    \label{sec_4.3}
Anschließend erfolgt eine inferenzstatistische Analyse der erhobenen Daten zur Fesstellung der Korrelationen zwischen den in Kapitel~\ref{ch_2} näher gebrachten Konstrukte Aggression, Akzeptanz der Gewaltmythen und die Verantwortungszuschreibung. 


\subsection{Hypothese 1}    \label{subsec_4.3.1}
Die Hypothese 1 geht von einer Korrelation des Aggressionsscores mit victim blaming aus. Diese ungerichtete Zusammenhangshypothese wurde mit der Spearman$-$Rang$-$Korrelation berechnet.
Die in Kapitel ~\ref{sec_3.5} erwähnten Voraussetzungen für diesen Test sind, dass das Skalenniveau mindestens einer Variable ordinalskaliert ist. Die Variable des victim blaming ist in diesem Fall ordinal skaliert und somit ist diese Voraussetzung erfüllt. Auch die weitere Voraussetzung der paarweisen Beobachtung ist gegeben, da in SPSS eine Zeile die Erhebung einer Person darstellt. 

Das in Tabelle~\ref{H1_Spearman} ersichtliche Ergebnis der Spearman$-$Rang$-$Korrelation fiel nicht signifikant aus ($p$~=~n.s.). Mit dem sehr kleinen Effekt \parencite{Cohen_1992} von $r_s $~=~.05, korreliert die Aggression nur sehr gering mit der Schuldzuweiseung auf das Opfer. 
\begin{table}[htb]
    \caption[Mittelwerte, Standardabweichung und Korrelation von Aggression und Victim Blaming]{\textit {Mittelwerte, Standardabweichung und Korrelation von Aggression und Victim Blaming}} 
    \label{H1_Spearman}
    \centering
    \begin{adjustbox}{width=6.5cm} %{maxwidth=\textwidth}
    \small
    \begin{tabular}{lrrr}
      \hline
        & $M$   & $SD$ & 1 \\
      \hline
    1 Aggression      & 1.88  & 0.43   & $-$      \\
    2 Victim Blaming  & 26.46 & 27.98  & .18      \\
       \hline
    \end{tabular}
    \end{adjustbox}
    
    \begin{tablenotes}
        \item \textit{Anmerkungen.} \( N_{neu} \)~=~864; Wertebereich der Variable Aggression 1 (\textit{trifft nicht zu}) bis 4 (\textit{trifft voll zu}); Spearman$-$Rang$-$Korrelation.
      \end{tablenotes}
    \end{table}



\subsection{Hypothese 2}    \label{subsec_4.3.2}
Die Voraussetzung einer Pearson$-$Produkt$-$Moment$-$Korrelation wurden in ~\ref{sec_3.5} bereits aufgeführt. In diesem Falle sind auch alle Voraussetzungen gegeben: Aggression und die Akzeptanz von Gewaltmythen sind metrisch, die vorhandenen Ausreißer sind im akzeptablen Bereich, es liegt ein linearer Zusamenhang zwischen den Variablen vor und eine bivariate Normalverteilung ist auch gegeben.

Die in Tabelle~\ref{H2_Pearson} ersichtliche Pearson$-$Produkt$-$Moment$-$Korrelation zeigte einen Zusammenhang von $r$~=~.38. Nach \textcite{Cohen_1992} entspricht dies einer mittelgroßen Korrelation, die bei einseitiger Testung statistisch signifikant ausfiel ($p<$ .001).
\begin{table}[htb]
    \caption[Mittelwerte, Standardabweichung und Korrelation von Aggression und der Akzeptanz der Mythen häuslicher Gewalt]{\textit {Mittelwerte, Standardabweichung und Korrelation von Aggression und der Akzeptanz der Mythen häuslicher Gewalt}} 
    \label{H2_Pearson}
    \centering
    \begin{adjustbox}{width=5.5cm} %{maxwidth=\textwidth}
    \small
    \begin{tabular}{lrrr}
      \hline
        & $M$   & $SD$ & 1 \\
      \hline
    1 Aggression      & 1.88 & 0.43  & $-$      \\
    2 DVMAS           & 2.60 & 0.80  & .38*      \\
       \hline
    \end{tabular}
    \end{adjustbox}
    
    \begin{tablenotes}
        \item \textit{Anmerkungen.} \( N_{neu} \)~=~864; DVMAS = Akzeptanz der Gewaltmythen; Wertebereich der Variable Aggression 1 (\textit{trifft nicht zu}) bis 4 (\textit{trifft voll zu}); Variable DVMAS 1 (stimme überhaupt nich zu) bis 7 (stimme völlig zu); Pearson$-$Produkt$-$Moment$-$Korrelation. \\ *$p<$~.001
      \end{tablenotes}
    \end{table}




\subsection{Hypothese 3}    \label{subsec_4.3.3}
hier Vorraussetzungsergebnisse statistisch darlegen
%https://statistikguru.de/spss/moderation/voraussetzungen-14.html 
Linearität
Normalverteilung der Residuen (unwichtig, weil SP groß)
Homoskedastizität -> Schätzfehler über alle vorhergesagten y-Werte relativ gleich (gibt Heteroskedastizität in Residuen weil $p$~=~.03 beim White Test dadurch sind die Ergebnisse der Moderation nicht mehr sinnvoll interpretierbar)
Unabhängigkeit

Moderation: 14.31\% der Varianz des DVMAS werden durch das Modell erklärt.
Modell erklärt tatsächlich etwas, weil p kleiner .01 ist.
Grüne Linie: Geschlecht sorgt für höheren DVMAS-Wert bei gleichbleibender Aggression. Bei gleichbleibendem Aggressions-Score sorgt das Geschlecht für einen höheren DVMAS-Wert.

einen haupteffekt der signifikant ist der andere nicht, so wie die interaktion. Regressionsmodell mit 3 Prediktoren. eins hat n sig. Gewicht, der andere nicht.

Da die obere Grenze einen größeren Wert als 0 aufweist, ist die Interaktion nicht signifikant.

Eine Moderationsanalyse wurde durchgeführt, um zu bestimmen, ob die Interaktion zwischen Alter und Freizeit die Nutzung von sozialen Medien signifikant vorhersagt. Die Ergebnisse konnten keinen Moderationseffekt von Alter auf die Beziehung zwischen Freizeit auf Social Media-Nutzung finden, $\Delta R^{2}$ = 16.47\%, F(1, 96) = 18.93, p = .241, 95\% CI[-0.047, -0.015].


\section{Explorative Ergebnisse}    \label{sec_4.4}
Zusätzlich zu den untersuchten Hypothesen wurden noch weitere explorative Untersuchungen getätigt. In Kapitel~\ref{subsec_4.4.1} wurde getestet, ob die Subskalen von Aggression positiv mit dem DVMAS und mit dem Geschlecht der Probanden korreliert. Des Weiteren wurde Aggression mit dem kultuerllen Hintergrund der Probanden (vgl. Kapitel~\ref{subsec_4.4.2}) korreliert. Der Deutsche Aggressionsfragebogen unterlief zwei Unterschiedtestungen. In Kapitel~\ref{subsec_4.4.3} wurde exploriert, ob sich die Aggressionsausprägung zwischen den kulturellen Hintergründen unterscheidet und in Kapitel~\ref{subsec_4.4.4} mit den biologischen Geschlechtern der Probanden.

\subsection{Korrelation Aggresion-Subskalen mit DVMAS und Geschlecht}  \label{subsec_4.4.1}
Für die Korrelation zwischen den vier Aggressions-Subskalen physiche Aggression, verbale Aggression, Ärger und Misstrauen und der Akzeptanz von Gewaltmythen wurde eine Pearson-Moment-Korrelation berechnet, denn alle Variablen sind metrisch skaliert. An den, in Tabelle XX, kursiv gesetzten $r-$Werte ist erkennbar, dass alle Subskalen unter sich signifikant sind und, dass sie jeweils mit dem DVMAS positiv korrelieren. 

Die Korrelation von physische Aggression mit der verbalen weist einen mittleren Effekt ($r$~=~.42, $p<$~.001) auf. Ärger und physiche Aggression zeigen eine starke Korrelieren mit $r$~=~.52 auf ($p<$~.001). Die physische Aggression korreliert mittelgradig mit Ärger und Misstrauen ($r$~=~.32 und $r$~=~.38 bei $p<$~.001).
Die verbale Aggression korrelierte mittelgradig mit Ärger ($r$~=~.45, $p<$~.001) und Misstrauen ($r$~=~.30, $p<$~.001).
Die Subskala Ärger wies eine mittelgradig Korrelation mit Misstrauen ($r$~=~.48, $p<$~.001) auf.

Der DVMAS wies bei der physichen Aggression ($r$~=~.38, $p<$~.001) eine mittelgradig Korrelation auf und bei den drei weiteren Subskalen verbale Aggression ($r$~=~.27, $p<$~.001), Ärger ($r$~=~.25, $p<$~.001) und Misstrauen ($r$~=~.24, $p<$~.001) jeweils eine schwache Korrelation auf.

Das Geschlecht zeigte eine mittelgradig negative Korrelation mit der physischen Aggression da ($r$~=~$-$.25, $p<$~.001). Mit der varbalen Aggression korreliert das Geschlecht schwach negativ ($r$~=~$-$.16, $p<$~.001). Mit den beiden Subskalen Ärger ($r$~=~.07, $p$~=~.020) und Misstrauen ($r$~=~.09, $p$~=~.006) zeigte sich jeweils eine schwache Korrelation mit dem biologischen Geschlecht der Probanden.


\subsection{Korrelation Aggression mit kulturellem Hintergrund der Probanden}   \label{subsec_4.4.2}
Die mittelgradig negative Korrelation zwischen dem kulturellen Hintergrund der Probanden und dem Aggressionsscore zeigt mit $r$~=~$-$.26 ein signifikantes Ergebnis ($p<$~.001). Die Werte dieser Korrelation sind in Tabelle XX zusammengefasst.


\subsection{Unterscheidung Aggression mit kulturellem Hintergrund der Probanden}   \label{subsec_4.4.3}
Für die Untersuchung der Aggressionsausprägung bei Probanden mit arabischem Hintergrund verglichen mit denen eines anderen Hintergrundes, wurde ein t-Test für unabhängige Stichproben gerechnet. Mit $F$~=~.058 viel der Levene Test nicht signifikant aus ($p$~=~n.s.) und entspricht einer Varianzgleichheit. 
Der Unterschied zwischen den Probanden mit arabischen Hintergründen und denen ohne war signifikant ($t$(858)~=~7.78, $p<$~.001). Die Aggression war bei den Probanden ohne arabischen Hintergrund durchschnittlich 0.77 Einhaeiten niedriger (95\%$-$CI[0.58, 0.96]).


\subsection{Unterscheidung Aggression mit Geschlecht der Probanden}   \label{subsec_4.4.4}
Auch die explorative Untersuchung der Differenz zwischen den biologischen Geschlechtern in Anbetracht der Aggression, wurde mithilfe des t-Tests für unabhängige Stichproben getestet. Der Levene Test fiel mit $F$~5.715 signifikant aus. Das bedeutet, dass durch die Varianzunterschiede es zu falsche Schlüsse aus dem t-Test kommen kann.
Der Unterschied zwischen Männern und Frauen war signifikant ($t$(433.64)~=~2.18, $p$~=~.0.30). Die Aggression war bei Männern durchschnittlich 0.07 Einhaeiten niedriger (95\%$-$CI[0.01, 0.14]).

\chapter{Diskussion}   \label{ch_5}
Der abschließende Diskussionskapitel beginnt mit einer Zusammenfassung der zentralen Ergebnisse. Daraufhin folgt die Interpretation dieser und die Einordnung in den aktuellen wissenschaftlichen Forschungsstand. Im folgenden Zug werden die verwendeten Methoden bewertet. Die Arbeit endet mit theoretischen und oder praktischen Implikationen für die zukünftige Forschung.


\section{Zusammenfassung der zentralen Ergebnisse}  \label{sec_5.1}
Diese Bachelorarbeit befasst sich maßgeblich mit den Themen häusliche Gewalt, Gewaltmythen und deren Zusammenhang mit der Aggressivität der Probanden. Im laufe dieses Kapitels werden die Ergebnisse von den zuvor aufgestellten Hypothesen zusammgefasst, sowie die Ergebnisse der Post$-$hoc Analysen.

An dem online erhobenen Fragebogen nahmen hauptsächlich junge Personen teil und mehr als die Hälfte der $N$~=~432 großen Stichprobe waren Frauen. 
Die Ergebnisse der Manipulationschecks zeigen, dass mehr als 80\% der Probanden die Vignetten richtig gelesen haben und diese auch verstanden haben. Auffalend war die hohe Rate an richtigen Antworten bei der Nachfrage des Geschlecht des Opfers wie den kulturellen Hintergrund.


% H1: Der Aggressionsscore korreliert mit victim blaming.
Die H1 diente der Überprüfung einer möglichen Korrelation zwischen der Aggression und der Verantwortungszuschreibung auf das Opfer. Obwohl die Verteilung der Variable der Verantwortungszuschreibung nicht einer Normalverteilung entspricht, kann sie wegen der, mit $N_{neu}$~=~864, großen Stichprobe und des zentralen Grenzwertsatzes als metrisch eingestuft werden. 
Bei der Testung der Voraussetzungen für eine Pearson$-$Produkt$-$Moment$-$Korrelation kam eine nicht bivariate Normalverteilung heraus. Aus diesem Grund wurde auf die Spearman$-$Rang$-$Korrelation ausgewichen. Für diesen Test waren alle Voraussetzungen erfüllt. Die Spearman$-$Korrelation ergibt ein nicht signifikantes Ergebnis und auch der vorhandene Effekt ist gering. demzufolge wird die Nullhypothese angenommen und die Aggression der Probanden korreliert nur bedingt und nicht signifikant mit der Verantwortungszuschreibung auf das Opfer.

% H2: Der Aggressionsscore korreliert mit victim blaming.
Eine positive Korrelation zwischen Aggression und der Akzeptanz von Mythen häuslicher Gewalt ist der Bestandteil der H2. Für dessen Überprüfung waren die Voraussetzungen der Pearson$-$Produkt$-$Moment$-$Korrelation alle gegeben und konnte somit gerechnet werden. Die Testung kam zu einem signifikanten Ergebnis und einem mittelgroßen Effekt. Die Nullhypothese mit demzufolge verworfen werden und die Alternativhypothese angenommen werden. Die Aggression und die Akzeptanz der Gewaltmythen weisen eine mittelsrarke Korrelation auf.

% H3: Der Zusammenhang zwischen Akzeptanz von Gewaltmythen und Aggression wird durch das Geschlecht moderiert.
Von einer moderierenden Rolle des biologischen Geschlechts des Probanden auf den Zusammenhang zwischen der Akzeptanz von Mythen häuslichen Gewalt und Aggression, wurde bei der H3 ausgegangen. Bei der Überprüfung der Voraussetzungen kam der White-Test zu einem signifikanten Ergebnis, was für eine Homoskedastizität spricht. Die weiteren Voraussetzungen waren gegeben. Das Gesamtmodell der Moderation war signifikant, jedoch zeigte die Interaktion der drei Variablen ein nicht signifikantes Ergebnis. Die Nullhypothese muss demzufolge angenommen werden. Das Geschlecht hat keine moderierende Auswirkung auf den Zusammenhang von Aggression und der Akzeptanz der Gewaltmythen. Des Weiteren wurde überprüft, ob die beiden Prädiktoren Einfluss auf das Kriterium haben. Die Regressionsanalyse zeigt für das Geschlecht und die Aggression ein signifikantes Ergebnis. Beide Variablen beeinflussen demzufolge die Akzeptanz der Mythen häuslicher Gewalt.

% explorative Ergebnisse
Die ersten beiden Post$-$hoc Analysen dienten der weiteren Untersuchung der Korrelation von Aggression und der Akzeptanz der Mythen häuslicher Gewalt. Die erste Analyse erforschte die Korrelation zwischen den Aggression$-$Subskalen verbale Aggression, Ärger und Misstrauen untereinander und jeweils mit dem DVMAS. Sie kam mit erfüllten Voraussetzungen zu signifikanten Ergebnissen. Die drei Subskalen korrelieren mit einem mittelgroßen Effekt untereinander und mit einem schwachen Effekt mit dem DVMAS. Die vierte Subskala, physische Aggression, musste aufgrund zu großer Ausreißer von den Berechnungen entzogen werden.

Die zweite explorative Anaylse untersuchte den Einfluss des biologischen Geschlechts auf die Akzeptanz der Mythen häuslicher Gewalt. Da der Test alle Voraussetzungen erfüllte konnte er durchgeführt werden und somit kann anschließend das Ergebnis zur Interpreation herangezogen werden. Der t$-$Test für unabhängige Stichproben zeigt ein signifikantes Ergebnis. Männer akzeptieren Mythen häuslicher Gewalt mehr als Frauen.

Zusätzlich zu den weiterführenden Analysen von Aggression und DVMAS wurde auch eine Zusammenhangsanalyse der Aggression$-$Subskalen mit der Verantwortungszuschreibung auf das Opfer unternommen. Aufgrund der Ausreißer der physischen Aggression Subskala wurde eine Spearman$-$Rang$-$Korrelation gerechnet. Es zeigt sich, dass ausschließlich die physische Aggression mit der Verantwortungszuschreibung korreliert.


\section{Einordnung und Diskussion der Befunde}     \label{sec_5.2}
Im folgenden Unterkapitel werden die Ergebnisse der Hypothesen, wie auch die der Post$-$hoc Analyse interpretiert, disskutiert und in den aktuellen Forschungsstand eingeordnet.

% H1:
Die H1 zeigt ein nicht signifikantes Ergebnis. Die Aggression korreliert nicht mit Victim Blaming. Dieses Ergebnis deckt sich mit dem Befund von \textcite{H1_moderation_2020}. Das vorliegende nicht signifikante Ergebnis kann möglicherweise auf die bimodale Verteilung zurückgeführt werden. Wie in Abbildung~\ref{Histogramm VicBlame} ersichtlich, gaben sehr viele Probanden dem Täter, auf der linken Seite, die Veranwortung. Auf der rechten Seite, beim Opfer, ist eine weitere vergrößerte Anhäufung der Verantwortungszuschreibung zu sehen. Gegebenenfalls waren die Vignetten zu polarisierend formuliert, oder beinhalteten zu wenige Informationen, die gemeinsam mit einer sozialen Erwünschtheit dazu führte, dass viele Teilnehmer die Veranwortung auf der Seite des Täters zuordneten und exessiv dem Täter die alleinige Veranwortung zuschrieben. Gegebenenfalls kam \textcite{H1_malasia_2012} zu einem signifikanten Ergebnis, da im Rahmen dieser Studie Jugendliche untersucht wurden. Konträr zur Vermutung, dass durch das junge Alter vieler Probanden sich die Ergebnisse mit denen von \textcite{H1_malasia_2012} ähneln werden, gleicht sich das Ergebnis mit dem von \textcite{H1_moderation_2020}. In Kapitel~\ref{subsec_2.1.1} wurde im Sinne der positiven Aggression von dessen Nutzen im Jugendalter berichtet \parencite{Aggression}. Gegebenenfalls weisen Personen im heranwachsenden Alter ein höheres Aggressionsniveau auf, das sich bei den jungen Erwachsenen bereits verringert hat. Die Stichprobe von \textcite{H1_moderation_2020} gleicht sich mehr mit der hier erhobenen und aus diesem Grund sind diese Ergebnisse besser vergleichbar, zumahl sie sich ähneln.

% H2:
Das signifikante Ergebnis der H2 bedeutet, dass Aggression positiv mit der Akzeptanz von Mythen häuslicher Gewalt korreliert. Obwohl bislang nicht viele Studien diese beien Konstrukte gemeinsam untersucht haben, zeigt diese Arbeit, wie auch die von \textcite{H2_u_3_Bhogal_2016, H1_moderation_2020} dass es einen Zusammenhang gibt. Demzufolge akzeptiert eine Person mit einem höheren Aggressionsniveau vermehrt die Mythen häuslicher Gewalt. Wie bereits in Kapitel~\ref{subsec_2.1.1} vermutet, kann es daran liegen, dass die Mythen Gewalt thematisieren, die oft, wenn nicht ausschließlich, ihren Ursprung in der Aggression haben. Wie \textcite{Def_Aggressivität_vs_violence} in Kapitel~\ref{subsec_2.1.1} berichtet wurde, ist Gewalt erlerntes Verhalten, geprägt duch kulturelle Ideologien. Die gesellschaftliche Wahrnehmung und Einstellung prägen auch die Mythen häuslicher Gewalt \parencite{Labelingtheory_plus, DVMAS_Peters}. Die gesellschaftlichen und kulturellen Einstellungen und Ideologien scheinen demzufolge sowohl die Mythen, wie auch die Gewalt zu bedingen. Aus dem vorliegenden signifikanten Ergebnis lässt sich vermuten, dass diese Gewalt Aggression beinhaltet.


% H3:
Die Moderationsanalyse der H3 zeigt eine nicht signifikante Interaktion des biologischen Geschlechts mit der Aggression und der Akzeptanz der Mythen häuslicher Gewalt. In Kapitel~\ref{subsec_2.2.3} wurde von Geschlechterunterschieden bezüglich der Aggressivität wie auch der Akzeptanz von Gewaltmythen berichtet \parencite{H2_u_3_Bhogal_2016, H3_MFUnterschied, H3_2020}. Obwohl diese Studien darlegen, dass das Geschlecht sowohl auf die Aggression, wie auch auf die Akzeptanz von Mythen Auswirkungen hat, konnte diese Moderationsanalyse keinen Beleg dafür darlegen. Der Einfluss von Aggression auf die Akzeptanz von Gewaltmythen ist Geschlechterunabhängig. Bei der Prüfung der Voraussetzungen einer Homoskedastizität fiel der Whit-Test signifikan aus, das für eine Heteroskedastizität spricht. Diese kann zu einem möglicherweise nicht signifikanten Ergebnis führen, obwohl beide Prädiktoren hoch signifikatne Haupteffekte aufweisen.
Der durch die Heteroskedastizität entstandene Bias des Standardfehlers kann zu fehlerhaften Interpretationen und Schlussfolgerungen des $p-$Wertes der Moderation führen \parencite{Voraussetzung_Moderation}. 

Aufgrund des dennoch nicht signifikanten Moderationseffekt, wurde eine Analyse der Haupteffekte durchgeführt. Sie zeigt einen Effekt des Geschlechts auf die Akzeptanz, wie einen Effekt der Aggression auf die Akzeptanz. Dies bedeutet, dass unabhängig voneinander das Geschlecht und die Aggression Einfluss auf die Akzeptanz von Mythen häuslicher Gewalt nehmen. Der Einfluss des Geschlechts stimmen mit den Ergebnissen von \textcite{H3_2020} überein.


%Explo
% coor subskalen dvmas
% corr subskalen vicblame
Die Spearman$-$Rang$-$Korrelationen der Post$-$hoc Analyse fiel größtenteils signifikant aus. Alle Aggression$-$Subskalen korrelieren positiv untereinander und mit dem DVMAS. Bei der Korrelation mit der Verantwortungszuschreibung auf das Opfer korreliert ausschließlich die physische Aggression positiv mit der Verantwortungszuschreibung. Zur Untersuchung der Zusammenhänge wurde die Spearman$-$Korrelation verwendet, da bei der Prüfung der Voraussetzungen einer Pearson$-$Produkt$-$Moment$-$Korrelation die Aggression$-$Subskala große Ausreißer aufwies. Die signifikanten Korrelation der Subskalen sind mit der Validierung des Deutschen Aggressionsfragebogens \parencite{Aggressionsfragebogen} übereinstimmig. Somit unterstützen sie die Validität des verwendeten Fragbogens. Die signifikante positive Korrelation der Subskalen mit dem DVMAS dienen einer genaueren Untersuchung der H2 Korrelation. Das Ergebnis der Korrelation der Subskalen mit Victim Blaming weicht von dem Ergebnis von \textcite{H1_moderation_2020} ab. 
% warum nur physisch signifikant?


\section{Bewertung der Methode}   \label{sec_5.3}
Dieses Unterkapitel betrachtet die verwendete Methode genauer. Es werden sowohl die Stärken, wie auch ihre Schwächen aufgewiesen.
Durch die erreichte Stichprobe von $N$~=~432 konnte die zuvor berechnete Mindeststichprobe von 395 Probanden erreicht werden. Dies ist ein Argument für eine valide Erhebung einer quantitativen Querschnittsstudie. Die Stichprobe wurde nichtrandomisiert erhoben, die sich durch den Schneeballeffekt vergrößerte.

Der Einsatz von Manipulationschecks zur Überprüfung der Verständlichkeit und bewusste Wahrnehmung der dargebotenen Fallvignetten, trägt ebenfalls zu einer validen Untersuchung bei. Die Überprüfung des sozioökonomischen Status der Personen innerhalb der Fallvignetten kann zu Verwirrungen geführt haben, da gegebenenfalls von einem gemeinsamen sozioökonomischen Status des Paares ausgegangen wurde. Dennoch gewährleisten die Überprüfungen eine gute Einschätzung der Verständlichkeit der Vignetten. Dies wurde ersichtlich durch die hohe Identifikation des Geschlechts und des kulturellen Status innerhalb der Fallvignetten. Das Geschlecht kann aufgrund der noch immer bestehenden Vorstellung eines überwiegend weiblichen Opfers zu dessen richtigen Identifikation geführt haben. Die ausländischen und somit, in einer deutschen Befragung, exotischeren Namen konnten der Grund für dessen hohe Rate an Identifikation sein.

Der verwendete Aggressionsfragebogen, wie auch der Fragebogen zur Erfassung der Akzeptanz von Mythen häuslicher Gewalt zeigen beide eine hohe Reliabilität und Validität auf \parencite{Peters2003, Aggressionsfragebogen}. Der Aggressionsfragebogen, wie auch der deutsche DVMAS eigneten sich demzufolge gut für die Erhebung der Variablen Aggression und Akzeptanz von Gewaltmythen. 
Die Entscheidung die Fallvignetten zu Beginn zu präsentieren, diente der unvoreingenommenen Verantwortungszuschreibung seitens der Probanden. Diese präventive Maßnahme birgt jedoch auch Nachteile. Die Fallvignetten forderten die Beurteilung eines sensiblen Themas und die Verantwortungszuschreibung der kritischen Situation. Durch diese Voreingenommenheit kann es zu sozial erwünschtem Antwortverhalten der späteren Fragbögen gekommen sein. Ebenfalls die Erhebung der, in den Fallvignetten befindlichen Variablen verleif nicht optimal. Jeder Proband erhielt ein psychisches und ein sexualisiertes Szenario. Die übrigen drei Variablen wurden randomisiert zugeteilt. Dies führte dazu, dass den Probanden gegebenenfalls die selbe Variable dargeboten wurde, nur ein Mal im sexualisierten und ein Mal im physischen Setting. 

Durch das Onlinemedium konnte kein Einfluss auf Störvariablen erfolgen. Es war den Teilnehmern selbst überlassen unter welchen Bedingungen sie an der Studie teilnehmen. Durch die abschließende Nachfrage der sinngemäßen Bearbeitung kann davon ausgegangen werden, dass es keine zu großen Störvariablen bei der Durchführung gab. Durch einen fehlenden Versuchsleiter entfielen dessen Effekte, wie zum Beispiel, dass sich ein Proband dadruch beobachtet fühlen kann. Dies führt zu objektiveren Datenerhebungen. Dennoch konnte keine aktive Kontrolle der Störvariablen erfolgen.

Zusammenfassend kann man die Ergebung der Daten durch die standardisierten Fragebögen und durch die Manipulationschecks als valide einstufen. Dennoch sollte bei einer erneuten Untersuchung auf eine verbesserte Randomisierung der Variablen, innerhalb der Fallvignetten, wie auch der Fragebögen geachtet werde. 


\section{Ausblick}    \label{sec_5.4}
In diesem abschließenden Kapitel werden die Konsequenzen und Folgen der gewonnenen Erkenntnisse genannt. Des Weiteren wird auf einen weiteren Forschungsbedarf hingewiesen.

Die zu Beginn aufgeworfene Frage, eines möglichen Zusammenfassungs zwischen der Aggression der Probanden und ihrer Akzeptanz von Mythen häuslicher Gewalt, sowie ihrer Tendenz zum Victim Blaming, kann abschließend teilwese zugestimmt werden. Das Ausmaß an Aggression hängt mit der Akzeptanz von Mythen häuslicher Gewalt zusammen. Weiter zeigt eine explorative Post$-$hoc Analyse, dass die vier Subskalen der Aggression ebenfalls mit dem DVMAS zusammenhängen. Jedoch korreliert, im Rahmen einer weiteren Post$-$hoc Analyse, nur eine Subskala der Aggression, die physische Aggression, mit der Verantwortungszuschreibung auf das Opfer. Auch der vermutete Interaktionseffekt des Geschlechts zwischen Aggression und Gewaltmythenakzeptanz ist nicht gegeben. 

%Fragebogen validieren
Der im Rahmen des Projektes verwendete Fragebogen zur Erhebung der Akzeptanz von Mythen häuslicher Gewalt ist zu dem Zeitpunkt dieser Verschriftlichung bislang noch nicht validiert worde. Da es sich lediglich um eine sinngemäße Übersetzung des Originals von \textcite{Peters2003} handelt, kann davon ausgegangen werden, dass sich die Validität, Reliabilität und Objektivität nicht ausschlaggebend davon unterscheiden. Aus diesem Grund wird die Validität dieser Arbeit in Bezug auf den DVMAS als gut geschätzt. Dennoch sollte eine Validierung der deutschen Übersetzung erfolgen und anschließend gegebenenfalls eine erneute Untersuchung der, im Rahmen dieser Bachelorthesis, aufgestellten Hypothesen bezüglich des DVMAS durchgeführt werden. 

%viel mehr infos und weiterbildung muss sein (Mythen)
Die Ergebnisse zeigen, dass in Bezug auf die Gewaltmythen die Gesellschaft noch weiter aufgeklärt werden muss. Der Originalfragebogen stammt aus dem Jahr 2003 und ist somit an die 20 Jahre als. Dennoch sind sie noch weit stark verbreitet, was durch unsere Stichprobe ersichtlich ist. Im Rahmen der Aufklärung sollte ein Fokus auf die männliche Bevölkerung gesetzs werden, das sie diese Mythen vermehrt akzeptieren. Zudem sollten auch verhaltensauffällige Kinder und Jugendliche über dieses Thema aufgeklärt werden, da vor allem in den Jugendjahren sich große Teile des Selbst entwickeln \parencite{H1_Entwicklung}.

%Alt mit Jung vergleichen, weil hauptsächlich Jung kann auch verzerren
% vielleicht auch mit kindern, um altersunterschied zu erforschen was aggro, aber auch dvmas angeht und ggf. vic blame
Daran anschließend sollten die Themen dieser Arbeit in einer repräsentativen Studie wiederholt werden. Durch die von der Volljährigkeit nicht weit entfernet Stichprobe, kann nicht auf ältere Generationen geschlossen werden. Zusätzlich währe eine Untersuchung der Aggressivität, der Gewaltmythenakzeptanz und des Victim Blamings bei Kindern, Jugendlichen und älteren Altersgruppen interessant, um auch Generationsunterschiede zu erforschen. In Kapitel~\ref{sec_5.2} wurde bereits vermutet, dass das Alter eventuell eine Auswirkung auf die Aggressionsausprägung hat. Durch eine altersunterscheidende Untersuchung könnte dies weiter erforscht werden. 

%komplette randomisierung: Variablen Vignetten und FB Reiehendolge
% meine explos können falsch positiv sein, deshalb nochmal prüfen
% h3 wegen hetero kein valides Ergebnis, deshalb nochmal prüfen, weil beide prädiktoren haben hoch signifikatne haupteffekte und wegen hetero vielleicht fälschlicherweise nicht sig
% themen von h1 nochmal prüfen, weil VicBlame net normalverteilt und dadurch vielleicht falsche schlüsse
Die im Rahmen dieser Bachelorthesis durchgeführten Analysen sollten zur Validierung erneut und gegebenenfalls mit einer größeren Stichprobe nochmals erhoben werden. Zum Einen dient dies der Überprüfung der möglicherweise falsch positiven Post$-$hoc Analysen, weiter einer potentiellen Verifikation der Hypothesenbefunde. Zum Anderen kann somit eine vollständige Randomisierung aller Variablen ermöglicht werden. Somit würden die Variablen innerhalb der Fallvignetten randomisiert werden. Zusätzlich kann somit die Reihenfolge der im Fragebogen enthaltenden vier Bausteine, Fallvignetten, DVMAS, Deutscher Aggressionsfragebogen und sozidemographische Daten, zufällig festgelegt werden. 


% abschließendes Fazit (i guess)
Abschließend kann festgehalten werden, dass eine erneute Durchführung, der hier untersuchten Konstrukte, ein Gewinn für die zukünftige Forschung darstellt. Nichts desto trotz sollte bereits jetzt mehr Aufklärung über häusliche Gewalt gewährleistet werden.



\printbibliography[heading=bibintoc,title={Literaturverzeichnis}]
%%%%%%%%% wenn mehrere Anhänge vorhanden %%%%%%%%%
\begin{appendices}
    \chapter*{Anhang A}
    \addcontentsline{toc}{chapter}{A}
    \noindent \textit{Titel von Anhang A}

Wenn nur ein Anhang vorhanden ist, dann sieht die Syntax so aus:

\noindent \texttt{\char`\\ chapter*\char`\{Anhang\char`\}} \\
\texttt{\char`\\addcontentsline\char`\{toc\char`\}\char`\{chapter\char`\}\char`\{
    Anhang\char`\}} \\
\texttt{\char`\\noindent \char`\\textit\char`\{Titel des Anhangs\char`\}}


    \chapter*{Anhang B}
    \addcontentsline{toc}{chapter}{B}
    \noindent \textit{Titel von Anhang B}

    Inhalt
\end{appendices}
%%%%%%%%% wenn mehrere Anhänge vorhanden %%%%%%%%%



%%%%%%%%% wenn nur 1 Anhang vorhanden %%%%%%%%%
%\chapter*{Anhang}
%\addcontentsline{toc}{chapter}{Anhang}
%\noindent \textit{Titel des Anhangs}

%Inhalt
%%%%%%%%% wenn nur 1 Anhang vorhanden %%%%%%%%%
\newpage
\fontsize{18pt}{0pt}\textbf{Ehrenwörtliche Erklärung}
\parskip=32pt

\noindent Gemäß Studien- und Prüfungsordnung erkläre ich, dass ich diese 
Bachelorthesis selbstständig angefertigt und wörtliche und sinngemäße 
Zitate kenntlich gemacht habe. Mit der Überprüfung auf etwaige 
Übereinstimmungen mit fremden Quellen mit Hilfe von Anti- Plagiatssoftware bin
ich einverstanden. Ich erkläre außerdem, dass diese Arbeit nicht im Rahmen 
eines anderen Prüfungsverfahrens bereits vorgelegt wurde. 
\parskip=32pt

\noindent Heidelberg, den \today
\parskip=32pt

\noindent Unterschrift: % Unterschriften

[Unterschrift 1]

% [Unterschrift 2]
\end{document}
