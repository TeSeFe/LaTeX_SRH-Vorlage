\subsection{Hypothese 2}    \label{subsec_2.2.2}
Bislang gibt es wenig Studien, die die Relation von Aggression und der Akzeptanz von Gewaltmythen untersucht. Es lässt sich jedoch vermuten, dass sie zusammenhängen, zumal der DVMAS mit patriachalen und sexistischen Einstellungen verbunden ist. Glaubende dieser Sichtweisen scheuen von der Anwendung von aggressiven Handlungen und Gewalt zur Behebung von Konflikten nicht zurück \parencite{DVMAS_Peters}. Obwohl Gewalt nicht gleichzusetzen ist mit Aggression, ist es auch nicht möglich zu sagen, dass diese beiden Begriffe nicht mit einander in Verbindung stehen.
Erst in jüngeren Jahren scheint die Korrelation von Aggression und Akzeptanz von Gewaltmythen an Interesse gewonnen zu haben. 

Die Studie von \textcite{H2_u_3_Bhogal_2016} mit $N$~=~121 Probanden untersuchte ob physische und verbale Aggression wie auch Ärger und Misstrauen die Akzeptanz von Vergewaltigungsmythen vorhersagen können. Die Ergebnisse zeigen, dass physische Aggression die Akzeptanz signifikant vorhersagt.

In einer Studie mit 100 verheirateten und berufstätigen Frauen untersuchten \textcite{H1_moderation_2020} unter anderem eine mögliche moderierenden Rolle der Toleranz$-$Subskala Neuheit zwischen häuslicher Gewalt und Aggression. Sie kamen zu der Erkenntnis, dass die Akzeptanz von Gewaltmythen ein signifikanter Prediktor von Aggression ist (\textbeta~=~.734, $p<$ .01).

Aufgrund des theoretischen Hintergrundes von \textcite{DVMAS_Peters} und den Ergebnissen der Studien von \textcite{H1_moderation_2020, H2_u_3_Bhogal_2016} bildet sich der Gedanke, dass Aggression und die Akzeptanz von Gewaltmythen häuslicher Gewalt positiv korrelieren. Daraus ergibt sich folgende gerichtete Zusammenhangshypothese: Der Aggressionsscore korreliert positiv mit der Akzeptanz von Gewaltmythen


