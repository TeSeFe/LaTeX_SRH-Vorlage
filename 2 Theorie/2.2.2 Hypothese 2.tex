\subsection{Hypothese 2}    \label{subsec_2.2.2}
Die Studie von \textcite{H2_u_3_Bhogal_2016} zeigt, dass körperliche Aggression einen signifikanten Einfluss auf die Akzeptanz von Vergewaltigungsmythen hat. Diese Studie wird untersuchen, ob Aggression im Allgemeinen auch ein gültiger Prädiktor für die Akzeptanz von Gewaltmythen im Allgemeinen ist.

Bei der Untersuchung der möglichen moderierenden Rolle der Toleranz$-$Subskala Neuheit zwischen häuslicher Gewalt und Aggression kamen \textcite{H1_moderation_2020} zur Erkenntnis, dass die Akzeptanz von Gewaltmythen ein signifikanter Prediktor von Aggression ist ($\beta$~=~.734, $p<$ .01).

Aufgrund von QUELLEN wird folgende Hypothese abgeleitet: Die Aggression korreliert positiv mit der Akzeptanz von Gewaltmythen.

% Durch die in Kapitel 1.4 jeweiligen Studien von Haj-Yahia (2003), McElligott (2011), George und Martínez (2002) und Acosved und Long (2006) liegt der Gedanken nahe, dass der kulturelle Hintergrund der Betroffenen Auswirkungen auf die Wahrnehmung und daraus resultierende Verantwortungszuschreibung der Gesellschaft hat. Daraus ergibt sich die gerichtete Unterschiedshypothese: