\subsection{Aggressivität und Aggression}    \label{subsec_2.1.1}

%Aggressivität vs Aggression
Aggressivität ist nicht gleichzusetzen mit Aggression. Ersteres bezieht sich auf eine 
überdauernde Disposition eines Individuums zu aggressivem Verhalten. Diese Bereitschaft wird 
nich immer offen ausgeführt und ist unterschiedlich ausgeprägt
\parencite{Def_unterscheidung_Springer, Def_Aggressivität_Duden, Def_Aggressivität_Spektrum}
Aggressivität entspricht demzufolge einer Verhaltenstendez, einer übergeordnete 
Charaktereigenschaft, die sich in Form von Aggression oder aggressivem Verhalten zeigt.
Personen, die Aggressivität als Teil ihrer Persönlichkeit haben, können beispielsweise die  
folgenden Charakteristika aufweisen \parencite{Def_Aggressivität_eng1}:
\begin{itemize} [leftmargin=1.25cm]
      \item Problematik die Emotionen und Gedanken anderer zu verstehen und nachzuempfinden
      \item externe Attribution
      \item Soziale Manipulation, um das Bedürfnis von Kontrolle über andere 
            Personen zu befriedigen
      \item Emotionale und affektive Defizite zeigen sich durch Aggressivität auf Grund 
            einer fehlerhaften Wahnehmung von fehlender Wertschätzung anderer
      \item Aggressive Personen sind der Meinung, dass sie für ihre Verwandten oder nahestehenden
            Menschen nicht wichtig sind
\end{itemize}

%Aggression
Aggression hingegen ist als vorübergehende Handlungsart zu verstehen, die es zum Ziel hat eine 
Person oder einen Gegenstand zu verletzen oder zu schädigen
\parencite{Def_Aggression_1939, Def_unterscheidung_Springer, Def_Aggression_Duden}.
Ursprünglich kommt das Wort Aggression aus dem Lateinischen und bedeutet 
\enquote{an eine Sache heran gehen} oder \enquote{etwas in Angriff nehmen} \parencite{was_Aggression}
 und ist weder positiv noch negativ. Im normalen Sprachgebrauch besitzt dieses Wort jedoch 
häufig eine negative Konnotation und wird von großen Teilen der Bevölkerung missbilligt. 
Aggressive Handlungen reichen von negativen Äußerungen über Mitmenschen sowie das Schreien oder 
Fluchen bis hin zu beabsichtigter Schädigung fremden Eigentums. 

Negative Aggression gilt 
aufgrund der negativen Emotionen, die durch sie ausgelöst werden, als ungesund. Dauerhaftes 
bestehen solcher Emotionen kann  schädlich für den Menschen sein \parencite{Aggression}.

Wenn Aggression aber das eigene Überleben, den eigenen Schutz oder auch die Bewahrung 
von Beziehungen fördert, dann bezeichnet \textcite{positive_aggression} 
es als positives und gesundes Verhalten. Wie \textcite{Aggression} 
zusammenfasst, ist es, im Sinne der positiven Aggression, während der Entwicklungsjahre eines 
Kindes und Jugendlichen notwendig ein gewisses Maß an Aggressivität zu besitzen. Dies hilf dem 
Heranwachsenden beim Ausbau von Autonomie und der eigenen Identität. Des Weiteren wird ein 
gewisses Grad an Aggression im Zusammenhang mit Wettkämpfen oder anderen Arten von Konkurrenz 
meist sogar erwünscht. Wenn die Aggression in die richtige Richtung gelenkt wird, ist sie die 
nötige Kraft, um ein gesundes Maß an Selbstbewusstsein, Dominanz und Unabhängigkeit zu erlangen \parencite{Aggression}.
Positive Aggression hat viele Formen und Facetten. Ergänzend zu \textcite{positive_aggression}
zählt \textcite{jack1999behind}
das Streben nach neuen Möglichkeiten und die Verteidigung gegen Schaden als Ausdruck positiver 
Aggression.

%Aggression nicht gleich Gewalt
Laut dem Duden ist Gewalt die \enquote{gegen jemanden, etwas [rücksichtslos] angewendete
physische oder psychische Kraft, mit der etwas erreicht werden soll} \parencite{Gewalt_Duden}.
Diese Definition ähnelt der der Aggression. Oftmals werden diese Wörter im Sprachgebrauch
gleichdeutend verwendet. Sie sind jedoch nicht als Synonyme zu gebrauchen. Die Trennung beider
Begriffe ist dennoch nicht einfach. \textcite{Def_Aggressivität_vs_violence}
trennt diese beiden Begriffe wie folgt. Aggression ist ein natürlicher und angeborener Instinkt, 
der nicht ausschließlich dem Menschen zuzuschreiben ist. Gewalt hingegen ist ein von der Kultur
bestimmtes Element und Teil der menschlichen Zivilisation. Wie bereits näher gebracht ist die 
Aggression wie ihre höhere Instanz, die Charaktereigenschaft Aggressivität von biologischem 
Ursprung, dessen Ziel und Zweck das Überleben ist \parencite{Def_Aggressivität_vs_violence, Aggression}.
Die positiven Aggression von Liu ähnelt der Auffassung von Clark. Letztere weist aber auch
darauf, dass durch die Beziehungen zu Gewalt die Aggression zu einem sozikulutrellen Aspekt
geworden ist \parencite{Def_Aggressivität_vs_violence}.
Dies kann ein möglicher Grund für die erschwerte Abgrenzung zwischen diesen beiden
Begriffen sein. 

Die negative Aggression von Liu lässt sich zu Teilen mit der Gewalt von Clark vergleichen \parencite{Def_Aggressivität_vs_violence, Aggression}.
Sie sieht Gewalt als erlerntes Verhalten, dass durch kulturelle Ideologien und Werte geprägt 
ist,  geplant und absichtlich ausgeführt wird. Der unterschied zwischen Aggression und Gewalt
liegt darin, dass Gewalt versucht Macht und Kontrolle zu erhalten, während Aggression dem 
Eigenschutz dient \parencite{Def_Aggressivität_vs_violence}.

Zusammenfassend ist es nicht unbedingt ratsam zu versuchen die Begriffe Aggression, 
Aggressivität und Gewalt so klar abzutrennen. Die Nutzung und sprachliche Bedeutung der Worte
haben sich im Wandel der Zeit verändert, wodurch sich die Bedeutungen der einzelnen Begriffe 
näher gekommen sind. Des Weiteren ist die klare Abtrennung durch die Verwobenheit der Konstrukte
erschwert. Zusätzlich zu den hier aufgeführten Begriffen git es noch weitere, die mit dieser
Thematik verwandt sind, auf die in dem Umfang dieser Arbeit jedoch nicht eingegangen werden.
Des Weiteren kann Gewalt, so wie die Aggression, in unterschiedliche Arten unterteilt werden.
Eine solche Unterteilung wird in Kapitel ~\ref{subsec_2.1.2} vorgenommen.


%sexuelle Aggro 9.5.2 im sozio buch

\subsubsection{Aggressionsarten}    \label{subsubsec_2.1.3.1}
%Arten von aggressivem Verhalten/ Aggressivität/ Aggression
In den vorangegangenen Absetzen wurde bereits auf unterschie Arten von Aggression eingegangen
auch wenn sie nicht explizit genannt wurden. In dieser Arbeit werden auf die folgenden Typen
von Aggression eingegangen: impulsive, instrumentelle, physische und abschließenden verbale 
Aggression.

% impulsive Aggression
Die \textit{impulsive}, oder auch affektive Aggression ist die unvorhersehbare und automatische 
Darbietung von Gewalt. Oftmals entsteht sie aus dem momentan erlebten Emotionen ohne über die 
eigendliche Handlung oder ihre Folgen nachzudenken. Diese Reaktion auf eine reale, oder auch 
eingebildete Provokation, kann unkontrolliert oder unverhaltnismäßig erscheinen \parencite{impulsive_instrumental_aggro_healtline, impulsive_aggro}
Impulsive Aggression ist bei einigen psychischen Störungen wie beispielsweise ADHS, 
Zwangsstörungen, oder bipolare Störungen zu beobachten \parencite{impulsive_aggro_psych_Störung}.

% instrumentelle Aggression
Wie die Bezeichnung diese Art von Aggression nahelegt, handelt es sich bei der 
\textit{instrumentellen} oder kognitiven Aggression um ein Hilfsmittel um ein größeres Ziel zu 
erreichen. Hierbei besteht keine zwangläufige Absicht einer Person Schaden zuzuführen \parencite{instrumental_aggro, instrumental_dictionary}.
Ein Beispiel instrumenteller Gewalt sind Auftragskiller und zu gewissem Grad auch Soldaten, die
für die Zielerreichung des Geldes Personenschaden als Nebeneffekt annehmen. Diese Darbietung
von Aggression ist kalkulierter und zielgerichteter ohne die Kontrolle zu verlieren \parencite{impulsive_instrumental_aggro_healtline}.

% physische und verbale Aggression
Aggression wird letzendlich auf zwei verschiedene Art und Weisen ausgedrückt. Wenn sie in Form
von Schlägen, Tritten, oder jeglicher weiter Handlungen, die dazu führen, dass eine Person 
physisch verletzt wird, dann handelt es sich um \textit{physische} Aggression \parencite{impulsive_instrumental_aggro_healtline, physische_verbale_aggro, physische_verbale_aggro_2}.
Bei der \textit{verbale} Aggression wiederum handelt es sich um Worte, die einen schädigenden 
Effekt haben. Es handelt sich dabei um Beschimpfungen, Drohungen oder Mobbing, um einige zu 
nennen \parencite{physische_verbale_aggro, physische_verbale_aggro_2, impulsive_instrumental_aggro_healtline}
Obwohl der Schaden physicher Aggression einfacher zu erkennen ist, sind die Kosten verbaler 
Aggression hoch. Mobbingopfer wießen im vergleich zu anderen Kindern gehäufte Depression, 
Angstzustände, Einsamkeit und Ablehnung durch Gleichaltrige auf \parencite{ausmaß_verbale_aggro}.



\subsubsection{Aggressionstheorie}    \label{subsubsec_2.1.3.2}
In der Forschung gibt es mehrere Modelle und Theorien, die sich  mit der Entstehung und 
Aufrechterhaltung von Aggression und aggressivem Verhalten befassen. Im Rahmen dieser Arbeit wird
ein näherer Blick auf den lerntheoretischen Ansatz geworfen. Lernerfahrungen haben zweifellos eine
wichtige Rolle in der Entstehung und Aufrechterhaltung aggressivem Verhaltens \parencite{Aggro_Theorie}.
Dabei sind die \textit{direkte Verstärkung} und das \textit{Modelllernen} von Bedeutung. 

Bei der Verstärkung wird aggressives Verhalten belohnt, wodurch das Kind lernt, dass solches 
Benehmen angebracht ist. Die Belohnung tritt in Kraft, durch die Erreichung eines zuvor festgelegten
Ziels oder durch die Erfahrung sozialer Annerkennung als Folge des aggressiven Verhaltens. 
Zusammenfassend kann man unter direkter Verstärkung den Effekt positiver Konsequenzen auf aggressives 
Verhalten verstehen \parencite{Aggro_Theorie_Buch}.

Das Modelllernen geht davon aus, dass die Etablierung von aggressivem Verhalten keine eigene
motorische Erfahrung benötigt. Laut diesem Mechanismus lernt das Individuum durch Beobachtung
aggressiven Verhaltens, dieses anzuwenden. Durch die Belohnung oder Bestrafung der beobachteten 
Person lernt das Individuum welche Formen von Aggression in welchen Umgebungen und zu welchen 
Ausmaßen toleriert werden \parencite{Aggro_Theorie_Buch}.