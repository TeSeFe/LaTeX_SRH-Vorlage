\subsection{Gewaltmythen}   \label{subsec_2.1.3}
Die Art und Weise, wie Opfer wahrgenommen werden, spielt eine große Rolle für die gesellschaftliche und die eigene Reaktion auf häusliche Gewalt. Ob und wie die Öffentlichkeit und das Opfer auf einen Fall von häuslicher Gewalt reagiert, hängt stark von der gesellschaftlichen Wahrnehmung und Einstellung diesem Thema gegenüber ab \parencite{Labelingtheory_plus}. Diese Einstellungen werden von stereotypischen Vorstellungen über häusliche Gewalt geprägt. \textcite{DVMAS_Peters} nennt diese falschen Einstellungen und Vorstellungen \textit{Mythen häuslicher Gewalt}, die die Schwere physischer Aggression versuchen zu verringern, abzustreiten oder versuchen, Rechtfertigungsgründe dafür vorzubringen. Mythen häuslicher Gewalt senken die soziale Hilfestellung. Durch sie verändert sich die gewaltbetroffene Person von einem unschuldigen Opfer zu einer Person, die es bewusst oder unbewusst wollte, misshandelt zu werden. Demzufolge ist das Opfer kein Opfer mehr, denn die Person hätte die Gewalt vermeiden können, oder hat den Partner willentlich provoziert \parencite{DVMAS_Peters}. Durch ein solches Bild der Opfer häuslicher Gewalt wird die Schwere dieser Tat minimiert, wird das Verhalten der Täter bagatellisiert. Dieser Bagatellisierung reduziert die Hemmschwelle zur Ausübung von Gewalt, sodass die Häufigkeit gewalttätigen Verhalten begünstigt wird. Des Weiteren prägen sie, wie zu Beginn des Kapitels bereits angedeutet, auch das Opfer. Ein falsches Unrechtempfinden sowohl beim Opfer selbst als auch bei weiten Teilen der Gesellschaft macht es dem Opfer schwer, Hilfe bei Familie, Nachbarn oder öffentlichen Institutionen zu suchen. Sie denken selbst, der Verursacher ihrer Lage zu sein, oder vermuten, dass außenstehende Personen sie nicht verstehen und somit auch die Hilfe unterlassen würde \parencite{Gewaltmythen}.

\subsubsection{Vergewaltigungsmythen}  \label{2.1.3.1}
Vergewaltigungmythen, sind wie die der häuslichen Gewalt stark von einem patriachalem System und einer solchen Denkweise geprägt. Sie umfassen Überzeugungen über die Ursachen und den damit verbundenen Kontext, wie auch die Folgen der Tat. Sie konstruieren ein Bild des Täters und des Opfers, durch das die sexualisierte Gewalt, häufig ausgehend von Männern und gerichtet auf Frauen, geleugnet und verharmlost wird \parencite{Vergewaltigung_Bohner_1996}. Die Absichten von Vergewaltigungmythen sind denen der häuslichen Gewalt sehr ähnlich \parencite{DVMAS_Peters}. Sie rechtfertigen die Tat, entschuldigen den Täter und schreiben dem Opfer Mitverantwortung zu \parencite{Vergewaltigung_Boris_2004}. Beispiele wie \enquote{Sie hat es durch ihr Aussehen oder Verhalten provoziert.} oder \enquote{Wenn die Frau es nicht will, kann sie sich doch wehren. Wenn sie es nicht tut, dann will sie es ja eigentlich.} \parencite{Vergewaltigung_Boris_2004}.

Diese Einstellungen und Überzeugungen klassifizierte \textcite{Vergewaltigung_Typen_Burt_1991} in vier Typen. Der erste Typ behauptet, \textit{es sei nichts passiert}. Die sexuelle Interaktion von Täter und Opfer wird dadurch abgestritten. Des Weiteren schließt dieser Typ nur gewisse Handlungen unter den Begriff der Vergewaltigung ein. 

Der zweite Typ deklariert, dass \textit{kein Schaden entstanden sei} und deutet damit hin, dass die Handlungen einer Vergewaltigung zu einer normalen sexuellen Interaktion gehört.

Unter dem dritten Typ werden Frauen so dargestellt, dass \textit{sie es eigentlich wolle}. Diese Überzeugungen sind darauf gerichtet, dass eine erwachsene Frau, wenn sie wolle, den Mann davon abhalten könne und mit unterlassenem Wiederstand einwilligt.

Der vierte und letzte Typ dient der Rechtfertigung der Tat, denn die betroffene Frau \textit{habe es verdient}, dass eine Mann solche Taten mit und an ihr ausübt. Somit würden nur Frauen vergewaltätig, die nicht in das Frauenbild einer patriachalen Denkweise passen \parencite{Vergewaltigung_Typen_Burt_1991}.


\subsubsection{Victim Blaming}     \label{2.1.3.2}
Ein großer Bestandteil dieser Gewaltmythen ist das \textit{Victim Blaming}. Unter diesem Begriff fällt die Überzeugung einer Person, egal ob Beobachter, Aggressor, oder Opfer, dass die betroffene Person die Verantwortung für die gegebene Lage trägt. In ihren Augen scheint das Opfer nicht schuldlos zu sein. Individuen, die Victim Blaming betreiben, rechtfertigen ihre Verantwortungszuschreibung mit Sätzen, wie: \textit{Sie will doch von ihm dominiert werden}. Oder Sätze wie: \textit{Wenn sie ihn so eifersüchtig macht, dann ist es normal, wenn er so reagiert} \parencite{Peters2003}. \textcite{victim_blaming} greift zwei Aspekte auf, die die Beurteilung eines Opfers beeinflussen können: Die Beziehung zu ihrem Aggressor und die schwierige Kooperation mit der Polizei. Die Zusammenarbeit mit dem Justizsystem ist erschwert, da die betroffene Person in vielen Fällen eine enge emotionale und finanzielle Bindung mit dem Täter hat. Ein weiterer Grund für die heruasfordernde Zusammenarbeit können die gemeinsamen Kinder des Opfers und Aggressors sein \parencite{victim_blaming}. Im nachfolgenden Kapitel ~\ref{2.1.3.2} werden weitere Gründe aufgegriffen, die zum Victim Blaming führen können.


\subsubsection{Theorien zur Erklärung von Victim Blaming}     \label{2.1.3.3}
Wie kann es dazu kommen, dass eine Person, die über eine lange Zeit hinweg regelmäßig von ihrem Partner misshandelt wird, dafür die Verantwortung zugeschrieben bekommt? Anhand von drei Theorien wird versucht darzustellen, wie persönliche Einstellungen und Weltbilder die Wahrnehmung eines jeden Menschen prägen. Diese geprägte Wahrnehmung führt bei manchen zum Victim Blaming.

Die \textit{Etikettierungstheorie} sieht die Ursache häuslicher als Resultat einer falschen Vorstellung des Opfers. Diese Theorie geht einen Schritt zurück und betrachtet die Gesellschaft und wie diese zu dem Schluss gelangt, dass gewisses Verhalten von der gesellschaftlichen Norm abweichend beziehungsweise kriminell sei. Dabei werden zwei Arten von Fehlverhalten unterschieden. \textit{Primäre Abweichung} tritt bei missbilligung sozialer Normen ein, ohne sich dessen bewusst zu sein. Wenn die Gesellschaft das Verhalten anschließend als abweichend definiert, kann dies zu Veränderungen der Selbstkonzeption sowie zur geminderten Identifikation weiterer Subgruppen abweichenden Verhaltens führen. \textit{Sekundäre Abweichung} betrifft Fehlverhalten, das als Resultat der Etikettierung und dessen negative Auswirkung auf die Selbstwahrnehmung entstanden ist. Vorallem wenn ein Opfer die missbrauchende Situation nicht verlässt, wird es oft dafür beschuldigt. Ursprünglich galt diese Theorie nur Kriminellen gegeüber. Im Laufe der Jahre wurde sie jedoch herangezogen, um die Betitelung des Opfers zu verstehen. \textcite{Labelingtheory_plus} berichten, dass laut bestehender Literatur Opfer basierend auf ihrem Verhalten vor, während und nach ihrer Viktimisierung als von der Norm abweichend betitelt würden. Opfer und ihr Verhalten werden demzufolge von einem großen Teil der Bevölkerung basierend auf gängigen Mythen über Missbrauchsbeziehungen bewertet, ohne den Erfahrungen der Betroffenen eine große Bedeutung beizumessen \parencite{Labelingtheory_plus}.

Der \textit{gerechte-Welt-Glaube} sieht die Akzeptanz der Gewaltmythen als Resultat nötigen Eigenschutzes. Er determiniert, dass jede Person das erhält, was ihr zusteht und was sie verdient hat, weil alles auf Basis eines universellen Prinzips der Gerechtigkeit geschieht. Menschen, die an eine solche Welt glauben, müssen demzufolge davon ausgehen, dass das Opfer Empfänger einer gerechten Handlung ist, weil sonst würde dies bedeuten, dass ihre Sicht der Welt inkorrekt sei \parencite{GM_Theorien}. 

Auch die \textit{defensive Attributionstheorie} vertritt einen selbstschützenden Ansatz. Für Frauen dient das Nutzen dieser Mythen um sich selbst vor der Vorstelluung zu schützen, Schaden zu erleiden \parencite{DVMAS, DVMAS_Peters}. Denn durch die Mythen bleibt der Anschein bestehen, dass die Wahrscheinlichkeit für solche Taten gering ist. Sie isolieren Partnerschaftsgewalt auf eine kleine Gruppe von Personen \parencite{DVMAS}. Auf diese Weise werden auch Männer geschützt, denn dadurch wird verhindert, dass Männer als potentielle Aggressoren angezweifelt werden können \parencite{DVMAS, DVMAS_Peters}. Laut der defensive Attributionstheorie schützen Mythen demzufolge potenzielle Opfer vor dem Bewusstwerden der Bedrohung und potenzielle Täter vor der Verantwortungszuweiseung \parencite{DVMAS_Peters}. Diese Schutzverhaltensweisen stehen im Einklang mit dem gerechte-Welt-Glaube.