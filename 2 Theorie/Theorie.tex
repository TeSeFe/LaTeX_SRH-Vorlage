

\chapter{Theorie}   \label{ch_2}
In diesem Kapitel werden zunächst die verwendeten Konstrukte Aggressivität und Aggression, häusliche Gewalt und abschließend Gewaltmythen erkläutert. Anschließend folgt eine theoretische Herleitung der einzelnen Hypothesen.

% 2.1 Konstrukte:
\section{Konstrukte}    \label{sec_2.1}
Im Anschluss erfolgt in den Unterkapiteln eine Einführung in die einzelnen Konstrukte. Beginnend werden in Kapitel~\ref{subsec_2.1.1} Aggressivität und Aggression erläutert und dessen Unterschied zur Gewalt verdeutlicht. Anschließend erfolgt eine Vorstellung verschiedener Arten von Aggression und Theorien, die versuchen, die Entstehung von Aggression zu erklären. In Kapitel ~\ref{subsec_2.1.2} wird häusliche Gewalt mit ihren Mustern und Arten präsentiert. Dieses Unterkapitel geht anschließend auf mögliche Ursachen und Aufrechterhaltung sowie auf die Folgen von Gewalt ein. Inhalt dieses Kapitels ist auch eine kurze Erläuterung bezüglich der Erfüllung des Artikel 13 der Istambul Konvention \parencite{Istambul_Konvention}. Im abschließenden Kapitel ~\ref{subsec_2.1.3} Gewaltmythen wird auf diesen Begriff und das Victim Blaming eingegangen.

\input{2 Theorie/2.1.1 Aggressivität.tex}
\input{2 Theorie/2.1.2 Häusliche Gewalt.tex}
\subsection{Gewaltmythen}   \label{subsec_2.1.3}
Die Art und Weise, wie Opfer wahrgenommen werden, spielt eine große Rolle für die gesellschaftliche und die eigene Reaktion auf häusliche Gewalt. Ob und wie die Öffentlichkeit und das Opfer auf einen Fall von häuslicher Gewalt reagiert, hängt stark von gesellschaftlichen Wahrnehmung und Einstellung diesem Thema gegenüber ab \parencite{Labelingtheory_plus}. Diese Einstellungen werden von stereotypischen Vorstellungen über häusliche Gewalt geprägt. \textcite{DVMAS_Peters} nennt diese falschen Einstellungen und Vorstellungen Mythen häuslicher Gewalt, die die Schwere physische Aggression versuchen zu verringern, abzustreiten oder versuchen Rechtfertigungsgründe dafür vorzubringen. Mythen häuslicher Gewalt senken die soziale Hilfestellung. Durch sie verändert sich die gewaltbetroffene Person von einem unschuldigen Opfer zu einer Person, die es bewusst oder unbewusst wollte, misshandelt zu werden. Demzufolge ist das Opfer kein Opfer mehr, denn die Person hätte die Gewalt vermeiden können, oder hat den Partner provoziert \parencite{DVMAS_Peters}.

%Vergewaltigungmythen (Exit-Mythen)

\subsubsection{Victim blaming}     \label{2.1.3.1}
Einer großer Bestandteil dieser Gewaltmythen ist das \textit{victim blaming}. Unter diesem Begriff fällt die Überzeugung einer Person, egal ob Beobachter, Aggressor, oder Opfer, dass die betroffene Person die Verantwortung für die gegebene Lage trägt. In ihren Augen scheint das Opfer nicht Schuldlos zu sein. Individuen, die victim blaming betreiben, rechtfertigen ihre Schuldzuweiseung mit Sätzen, wie: Sie will doch von ihm dominiert werden. Oder Sätze wie: Wenn sie ihn so eifersüchtig macht, dann ist es normal, wenn er so reagiert \parencite{DVMAS_deutsch}. \textcite{victim_blaming} greift zwei Aspekte auf, die die Beurteilung eines Opfers beeinflussen können: Die Beziehung zu ihrem Aggressor und die schwierige Kooperation mit der Polizei. Die Zusammenarbeit mit dem Justizsystem ist erschwert, da die betroffene Person in vielen Fällen eine enge emotionale und finanzielle Bindung mit dem Täter hat. Ein weiterer Grund für die heruasfordernde Zusammenarbeit können die gemeinsamen Kinder des Opfers und Aggressors sein \parencite{victim_blaming}. Im nachfolgenden Kapitel ~\ref{2.1.3.2} werden weitere Gründe dargeboten, die zum victim blaming führen können.


\subsubsection{Theorien zur Erklärung von victim blaming}     \label{2.1.3.2}
Wie kann es dazu kommen, dass eine Person, die über eine lange Zeit hinweg regelmäßig von ihrem Partner misshandelt wird, dafür die Verantwortung zugeschrieben bekommt? Anhand von drei Theorien wird versucht, darzustellen wie persönliche Einstellungen und Weltbilder die Wahrnehmung eines jeden Menschen prägen, die bei manchen zum victim blaming führt.

Die \textit{Etikettierungstheorie} sieht die Ursache als Resultat einer falschen Vorstellung des Opfers häuslicher Gewalt. Diese Theorie geht einen Schritt zurück und betrachtet die Gesellschaft und wie sich zu dem Schluss gelangt, dass gewisses Verhalten abweichend bzw. kriminell ist. Dabei werden zwei Arten von Fehlverhalten unterschieden. \textit{Primäre Abweichung} tritt ein bei missbilligung soziale Normen, ohne dessen bewusst zu sein. Wenn die Gesellschaft das Verhalten anschließend als abweichend definiert, kann dies zu Veränderungen der Selbstkonzeption, sowie zur geminderten Identifikation weiterer Subgruppen abweichenden Verhaltens. \textit{Sekundäre Abweichung} betrifft Fehlverhalten, dass als Resultat der Kennzeichnung und dessen negative Auswirkung auf die Selbstwahrnehmung entstanden ist. Vorallem wenn ein Opfer die missbrauchende Situation nicht verlässt, wird es oft dafür beschuldigt. Das macht es nur schwere für die Person zu entkommen, da sich ihr Leben um diese Betittelung konstruiert. Ursprünglich galt diese Theorie nur Kriminellen gegeüber. Im Laufe der Jahre wurde sie jedoch herangezogen, um die Betittelung des Opfers zu verstehen. \textcite{Labelingtheory_plus} berichten, dass laut bestehender Literatur Opfer basierend auf ihrem Verhalten vor, während und nach ihrer Viktimisierung als von der Norm abweichend betittelt werden. Opfer und ihr Verhalten werden demzufolge von einem großen Teil der Bevölkerung basierend auf gängigen Mythen über missbräuchliche Beziehungen bewertet, ohne die Erfahrungen der Betroffenen eine große Bedeutung beizumessen \parencite{Labelingtheory_plus}.

Der \textit{gerechte Welt-Glaube} sieht die Akzeptanz der Gewaltmythen als Resultat nötigen Eigenschutzes. Er determiniert, dass jede Person das erhält, was ihr zusteht und was sie verdient hat, weil alles auf Basis eines universellen Prinzips der Gerechtigkeit geschieht. Menschen, die an eine solche Welt glauben, müssen demzufolge davon ausgehen, dass das Opfer Empfänger einer gerechten Handlung ist, weil sonst würde dies bedeuten, dass ihre sicht der Welt inkorrekt ist \parencite{GM_Theorien}. 

Auch die \textit{defensive Attributionstheorie} vertritt einen selbstschützenden Ansatz. Für Frauen dient das Nutzen dieser Mythen um sich selbst vor der Vorstelluung, Schaden zu erleiden, zu schützen \parencite{DVMAS, DVMAS_Peters}. Denn durch die Mythen bleibt der Anschein bestehen, dass die Wahrscheinlichkeit für solche Taten gering ist. Sie isolieren Partnerschaftsgewalt auf eine kleine Gruppe von Personen \parencite{DVMAS}. Auf diese Weise werden auch Männer geschützt, denn dadurch wird verhindert, dass Männer als potentielle Aggressoren angezweifelt werden können \parencite{DVMAS, DVMAS_Peters}. Laut der defensive Attributionstheorie schützen Mythen demzufolge potenzielle Opfer vor dem Bewusstwerden der Bedrohung und potenzielle Täter vor der Schuldzuweiseung \parencite{DVMAS_Peters}. Diese Schutzverhaltensweisen stehen im Einklang mit dem gerechte Welt-Glaube.

% 2.2 Theoretischer Hintergrund: Aktueller Forschungsstand
\section{Aktueller Forschungsstand}   \label{sec_2.2}
In dieser Studie wird folgende Frage untersucht: Welchen Zusammenhang gibt es zwischen Aggression und der Akzeptanz der Mythen sowie der Tendez zum Victim Blaming? 

In den folgenden drei Unterkapiteln ~\ref{subsec_2.2.1} Hypothese 1, ~\ref{subsec_2.2.2} Hypothese 2 und ~\ref{subsec_2.2.3} Hypothese 3 erfolgt die Herleitung der zu untersuchenden Hypothesen auf Basis bereits bestehender Befunde.

\subsection{Hypothese 1}  \label{subsec_2.2.1}
d
\subsection{Hypothese 2}    \label{subsec_2.2.2}
Inhalt
\subsection{Hypothese 3}    \label{subsec_2.2.3}
\textcite{H2_u_3_Bhogal_2016} untersuchten in ihrer Studie 121 Studierende auf ihre Akzeptanz von Vergewaltigungsmythen und ihre Aggression in Form von physischer und verbaler Aggression, Ärger und Misstrauen. Sie kamen zum Entschluss, dass Männer eine höhere Akzeptanz von Vergewaltigungsmythen haben. Diese Mythen stellen die, meist weiblichen, als Mitschuldie da. Diese sichtweise des Opfers ist im Einklang mit den Mythen von häuslicher Gewalt. Ein weiteres Ergebnis der Untersuchung von \textcite{H2_u_3_Bhogal_2016} war, dass die physische Aggression bei Männern signifikant höher ist, verglichen mit den Frauen.

Laut \textcite{H3_MFUnterschied} sind Feministen der Meinung, dass Frauen ihre Aggressivität unterdrücken, gleichzeitig sind biologisch Positionierte der Auffassung, dass Frauen nicht die gleichen Fähigkeiten bzw. nicht das gleiche Bedürfniss haben, zu reagieren, so wie Männer es haben. Eine weitere Erklärung geht davon aus, dass Frauen das selbe biologische Potenzial für Aggressivität aufweisen, aber die Gesellschaft solch ein Verhalten ausschließlich bei Männern fördert. Die Verhaltensbiologie betrachtend, sind Männchen auf Grund von Testostron aggressiver als Weibchen. Obwohl Artenübergreifende Vergleiche mit Obhut zu genießen sind, bietet diese Tatsache einen starken Anhaltspunkt für einen Geschlechterunterschied menschlicher Aggressivität.

Im Rahmen einer Studie von $N$~=~329 Probanden zwischen 15 und 19 Jahren untersuchten \textcite{H3_2020} die Existenz sexualisierter Gewalt in romantischen Beziehungen, mögliche Zusammenhänge zwischen Mythen über sexualisierte Aggression und sexualisiertes Durchsetzungsvermögen und ihre möglichen geschlechterspezifischen Unterschiede. Ihre Ergebnisse zeigen, dass die männlichen Jugendlichen häufiger Täter sexualisierter Gewalt waren, und dass die männlichen Probanden vermehrt Mythen über sexualisierte Aggression Glauben schenkten. Das behandelte Thema des verwendeten Fragebogens ähnelt denen des DVMAS.

Aufgrund von \textcite{H2_u_3_Bhogal_2016, H3_MFUnterschied, H3_2020} liegt die Vermutung nahe, dass das Geschecht der Probanden eine moderierenden Rolle im Zusammenhang zwischen Akzeptanz von Gewaltmythen und Aggression hat. Aus diesem Gedanke ergibt sich die folgende Hypothese: Der Zusammenhang zwischen Akzeptanz von Gewaltmythen und Aggression wird durch das Geschlecht moderiert.

% Durch die in Kapitel 1.4 jeweiligen Studien von Haj-Yahia (2003), McElligott (2011), George und Martínez (2002) und Acosved und Long (2006) liegt der Gedanken nahe, dass der kulturelle Hintergrund der Betroffenen Auswirkungen auf die Wahrnehmung und daraus resultierende Verantwortungszuschreibung der Gesellschaft hat. Daraus ergibt sich die gerichtete Unterschiedshypothese:
