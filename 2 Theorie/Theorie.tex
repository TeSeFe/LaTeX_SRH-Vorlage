

\chapter{Theorie}   \label{ch_2}
In diesem Kapitel werden zunächst die verwendeten Konstrukte Aggressivität und Aggression, häusliche Gewalt und abschließend Gewaltmythen erkläutert. Anschließend folgt eine theoretische Herleitung der einzelnen Hypothesen.

% 2.1 Konstrukte:
\section{Konstrukte}    \label{sec_2.1}
Im Anschluss erfolgt in den Unterkapiteln ,  und  eine Darbietung der einzelnen Konstrukte. Beginnend werden in ~\ref{subsec_2.1.1} Aggressivität und Aggression die Aggressivität und Aggression voneinander unterschieden und der Unterschied zur Gewalt verdeutlicht. Anschließend erfolgt eine Darbietung verschiedener Arten von Aggression und Theorien, die die Entstehung von Aggression versuchen zu erklären. In Kapitel ~\ref{subsec_2.1.2} Häusliche Gewalt wird auf Muster und Arten, auf mögliche Ursachen und Aufrechterhaltung und abschließend auf die Folgen von Gewalt eingegangen. Inhalt dieses Kapitels ist auch eine kurze Erläuterung bezüglich der Erfüllung gewisses Artikel der Istambul Konvention. Im abschließenden Kapitel ~\ref{subsec_2.1.3} Gewaltmythen wird dieser Begriff aufgefasst, sowie das victim blaming. Zum Schluss diesem Unterkapitel erfolgen theoretische Ansätze, die versuchen das Phänomen des vitim blamings zu erklären.

\subsection{Aggressivität und Aggression}    \label{subsec_2.1.1}

%Aggressivität vs Aggression
Aggressivität ist nicht gleichzusetzen mit Aggression. Ersteres bezieht sich auf eine überdauernde Disposition eines Individuums zu aggressivem Verhalten. Diese Bereitschaft wird nich immer offen ausgeführt und ist unterschiedlich ausgeprägt
\parencite{Def_unterscheidung_Springer, Def_Aggressivität_Duden, Def_Aggressivität_Spektrum}. Aggressivität entspricht demzufolge einer Verhaltenstendez, einer übergeordnete Charaktereigenschaft, die sich in Form von Aggression oder aggressivem Verhalten zeigt. Personen, die Aggressivität als Teil ihrer Persönlichkeit haben, können beispielsweise die folgenden Charakteristika aufweisen \parencite{Def_Aggressivität_eng1}:
\begin{itemize} [leftmargin=1.25cm]
      \item Problematik die Emotionen und Gedanken anderer zu verstehen und nachzuempfinden
      \item externe Attribution
      \item Soziale Manipulation, um das Bedürfnis von Kontrolle über andere 
            Personen zu befriedigen
      \item Emotionale und affektive Defizite zeigen sich durch Aggressivität auf Grund einer fehlerhaften Wahnehmung von fehlender Wertschätzung anderer
      \item Aggressive Personen sind der Meinung, dass sie für ihre Verwandten oder nahestehenden Menschen nicht wichtig sind
\end{itemize}

%Aggression
Aggression hingegen ist als vorübergehende Handlungsart zu verstehen, die es zum Ziel hat eine Person oder einen Gegenstand zu verletzen oder zu schädigen
\parencite{Def_Aggression_1939, Def_unterscheidung_Springer, Def_Aggression_Duden}.
Ursprünglich kommt das Wort Aggression aus dem Lateinischen und bedeutet 
\enquote{an eine Sache heran gehen} oder \enquote{etwas in Angriff nehmen} \parencite{was_Aggression} und ist weder positiv noch negativ. Im normalen Sprachgebrauch besitzt dieses Wort jedoch häufig eine negative Konnotation und wird von großen Teilen der Bevölkerung missbilligt. Aggressive Handlungen reichen von negativen Äußerungen über Mitmenschen sowie das Schreien oder Fluchen bis hin zu beabsichtigter Schädigung fremden Eigentums. 

Negative Aggression gilt aufgrund der negativen Emotionen, die durch sie ausgelöst werden, als ungesund. Dauerhaftes Bestehen solcher Emotionen kann  schädlich für den Menschen sein \parencite{Aggression}.

Wenn Aggression aber das eigene Überleben, den eigenen Schutz oder auch die Bewahrung von Beziehungen fördert, dann bezeichnet \textcite{positive_aggression} 
es als positives und gesundes Verhalten. Wie \textcite{Aggression} zusammenfasst, ist es, im Sinne der positiven Aggression, während der Entwicklungsjahre eines 
Kindes und Jugendlichen notwendig ein gewisses Maß an Aggressivität zu besitzen. Dies hilf dem Heranwachsenden beim Ausbau von Autonomie und der eigenen Identität. Des Weiteren wird ein gewisses Grad an Aggression im Zusammenhang mit Wettkämpfen oder anderen Arten von Konkurrenz meist sogar erwünscht. Wenn die Aggression in die richtige Richtung gelenkt wird, ist sie die nötige Kraft, um ein gesundes Maß an Selbstbewusstsein, Dominanz und Unabhängigkeit zu erlangen \parencite{Aggression}. Positive Aggression hat viele Formen und Facetten. Ergänzend zu \textcite{positive_aggression} zählt \textcite{jack1999behind}
das Streben nach neuen Möglichkeiten und die Verteidigung gegen Schaden als Ausdruck positiver Aggression.

%Aggression nicht gleich Gewalt
Laut dem Duden ist Gewalt die \enquote{gegen jemanden, etwas [rücksichtslos] angewendete physische oder psychische Kraft, mit der etwas erreicht werden soll} \parencite{Gewalt_Duden}. Diese Definition ähnelt der der Aggression. Oftmals werden diese Wörter im Sprachgebrauch gleichdeutend verwendet. Sie sind jedoch nicht als Synonyme zu gebrauchen. Die Trennung beider Begriffe ist dennoch nicht einfach. \textcite{Def_Aggressivität_vs_violence} trennt diese beiden Begriffe wie folgt. Aggression ist ein natürlicher und angeborener Instinkt, der nicht ausschließlich dem Menschen zuzuschreiben ist. Gewalt hingegen ist ein von der Kultur bestimmtes Element und Teil der menschlichen Zivilisation. Wie bereits näher gebracht ist die Aggression, wie ihre höhere Instanz die Charaktereigenschaft Aggressivität, von biologischem Ursprung, dessen Ziel und Zweck das Überleben ist \parencite{Def_Aggressivität_vs_violence, Aggression}.
Die positiven Aggression von Liu ähnelt der Auffassung von Clark. Letztere weist aber auch darauf, dass durch die Beziehungen zu Gewalt die Aggression zu einem soziokulutrellen Aspekt geworden ist \parencite{Def_Aggressivität_vs_violence}.
Dies kann ein möglicher Grund für die erschwerte Abgrenzung zwischen diesen beiden
Begriffen sein. 

Die negative Aggression von Liu lässt sich zu Teilen mit der Gewalt von Clark vergleichen \parencite{Def_Aggressivität_vs_violence, Aggression}. Sie sieht Gewalt als erlerntes Verhalten, dass durch kulturelle Ideologien und Werte geprägt ist,  geplant und absichtlich ausgeführt wird. Der unterschied zwischen Aggression und Gewalt liegt darin, dass Gewalt versucht Macht und Kontrolle zu erhalten, während Aggression dem Eigenschutz dient \parencite{Def_Aggressivität_vs_violence}.

Zusammenfassend ist es nicht unbedingt ratsam zu versuchen die Begriffe Aggression, Aggressivität und Gewalt so klar abzutrennen. Die Nutzung und sprachliche Bedeutung der Worte haben sich im Wandel der Zeit verändert, wodurch sich die Bedeutungen der einzelnen Begriffe näher gekommen sind. Des Weiteren ist die klare Abtrennung durch die Verwobenheit der Konstrukte erschwert. Zusätzlich zu den hier aufgeführten Begriffen gibt es noch weitere, die mit dieser Thematik verwandt sind, auf die in dem Umfang dieser Arbeit jedoch nicht eingegangen werden. Des Weiteren kann Gewalt, so wie die Aggression, in unterschiedliche Arten unterteilt werden. Eine solche Unterteilung wird in Kapitel ~\ref{subsec_2.1.2} vorgenommen.


%sexuelle Aggro 9.5.2 im sozio buch

\subsubsection{Aggressionsarten}    \label{subsubsec_2.1.3.1}
%Arten von aggressivem Verhalten/ Aggressivität/ Aggression
In den vorangegangenen Paragraphen wurde bereits auf unterschiedliche Arten von Aggression eingegangen, auch wenn sie nicht explizit genannt wurden. In dieser Arbeit werden auf die folgenden Typen von Aggression eingegangen: impulsive, instrumentelle, physische und abschließenden verbale Aggression.

% impulsive Aggression
Die \textit{impulsive}, oder auch affektive Aggression ist die unvorhersehbare und automatische Darbietung von Gewalt. Oftmals entsteht sie aus dem momentan erlebten Emotionen ohne über die eigendliche Handlung oder ihre Folgen nachzudenken. Diese Reaktion auf eine reale, oder auch eingebildete Provokation, kann unkontrolliert oder unverhaltnismäßig erscheinen \parencite{impulsive_instrumental_aggro_healtline, impulsive_aggro}. Impulsive Aggression ist bei einigen psychischen Störungen wie beispielsweise ADHS, Zwangsstörungen, oder bipolare Störungen zu beobachten \parencite{impulsive_aggro_psych_Störung}.

% instrumentelle Aggression
Wie die Bezeichnung diese Art von Aggression nahelegt, handelt es sich bei der 
\textit{instrumentellen} oder kognitiven Aggression um ein Hilfsmittel um ein größeres Ziel zu erreichen. Hierbei besteht keine zwangläufige Absicht einer Person Schaden zuzuführen \parencite{instrumental_aggro, instrumental_dictionary}.
Ein Beispiel instrumenteller Gewalt sind Auftragskiller und zu gewissem Grad auch Soldaten, die für die Zielerreichung des Geldes Personenschaden als Nebeneffekt annehmen. Diese Darbietung von Aggression ist kalkulierter und zielgerichteter ohne die Kontrolle zu verlieren \parencite{impulsive_instrumental_aggro_healtline}.

% physische und verbale Aggression
Aggression wird letzendlich auf zwei verschiedene Art und Weisen ausgedrückt. Wenn sie in Form von Schlägen, Tritten, oder jeglicher weiter Handlungen, die dazu führen, dass eine Person physisch verletzt wird, ausgedrückt wird, dann handelt es sich um \textit{physische} Aggression \parencite{impulsive_instrumental_aggro_healtline, physische_verbale_aggro, physische_verbale_aggro_2}. Bei der \textit{verbale} Aggression wiederum handelt es sich um Worte, die einen schädigenden Effekt haben. Es handelt sich dabei um Beschimpfungen, Drohungen oder Mobbing, um Einige zu nennen \parencite{physische_verbale_aggro, physische_verbale_aggro_2, impulsive_instrumental_aggro_healtline}. Obwohl der Schaden physicher Aggression einfacher zu erkennen ist, sind die Kosten verbaler Aggression hoch. Mobbingopfer wießen im vergleich zu anderen Kindern gehäuft Depression, Angstzustände, Einsamkeit und Ablehnung durch Gleichaltrige auf \parencite{ausmaß_verbale_aggro}.



\subsubsection{Aggressionstheorie}    \label{subsubsec_2.1.3.2}
In der Forschung gibt es mehrere Modelle und Theorien, die sich  mit der Entstehung und Aufrechterhaltung von Aggression und aggressivem Verhalten befassen. Im Rahmen dieser Arbeit wird ein näherer Blick auf den lerntheoretischen Ansatz geworfen. Lernerfahrungen haben zweifellos eine wichtige Rolle in der Entstehung und Aufrechterhaltung aggressivem Verhaltens \parencite{Aggro_Theorie}. Dabei sind die \textit{direkte Verstärkung} und das \textit{Modelllernen} von Bedeutung. 

Bei der Verstärkung wird aggressives Verhalten belohnt, wodurch das Kind lernt, dass solches Benehmen angebracht ist. Die Belohnung tritt in Kraft, durch die Erreichung eines zuvor festgelegten Ziels oder durch die Erfahrung sozialer Annerkennung als Folge des aggressiven Verhaltens. Unter direkter Verstärkung ist damzufolge der Effekt positiver Konsequenzen auf aggressives Verhalten zu verstehen \parencite{Aggro_Theorie_Buch}.

Das Modelllernen geht davon aus, dass die Etablierung von aggressivem Verhalten keine eigene motorische Erfahrung benötigt. Laut diesem Mechanismus lernt das Individuum durch Beobachtung aggressiven Verhaltens, dieses anzuwenden. Durch die Belohnung oder Bestrafung der beobachteten Person lernt das Individuum welche Formen von Aggression in welchen Umgebungen und zu welchen Ausmaßen toleriert werden \parencite{Aggro_Theorie_Buch}.
\subsection{Häusliche Gewalt}    \label{subsec_2.1.2}
% Definition häusliche Gewalt
Wie in Kapitel ~\ref{subsec_2.1.1} Aggressivität und Aggression bereits erwähnt ist der Zweck von Gewalt die Macht und Kontrolle zu erhalten \parencite{Def_Aggressivität_vs_violence}. Wenn in einer Beziehung oder innerhalb der Familie eine Person versucht Macht oder Kontrolle über ein anderes Mitglied zu haben, zählt dies zur häuslichen Gewalt. Sowohl in bestehenden, wie auch in aufgelösten Beziehungen familiären, ehelichen oder eheähnlichen Ursprungs kann häusliche Gewalt auftreten \parencite{Def_haus_Gewalt, Def_haus_Gewalt_2}. Diese kommmt nicht nur in der häuslichen Umgebung, einem als sicher gedachten Ort vor, sondern kann auch im öffentlichen Raum stattfinden \parencite{Gewaltarten_WHO}. 

Es lassen sich dabei zwei Muster identifizieren. Ein \textit{spontanes Konfliktverhalten}, oder auch \textit{situative Gewalt} gennant, kann einmalig, aber auch regelmäßig stattfinden und hat die Funktion einer negativen Stressbewältigung. Durch fehlende Ressourcen sehen die gewalttätigen Personen nur die Gewalt als einzige Lösung, um ein Konflikt zu lösen. Ein solches Gewaltmuster ist sowohl bei Männern, wie auch bei Frauen zu finden. Das Motiv von Macht und Kontrolle ist in diesen Fällen nicht ausschlaggebend. Dieses Verhaltensmuster kann sich jedoch in langanhaltendes \textit{systematisches Gewalt- und Kontrollverhalten} verwandeln. Hier haben Macht und Kontrolle eine große Rolle. In diesen Fällen existiert die Absicht den Gegenüber zu kontrollieren und ein langanhaltendes Gefühl von Macht zu verspühren. Dieses Verhalten ist vermehrt bei Männer vorzufinden, die sich in einer ungleichen Beziehung befinden. Um dieses Gefühl von Macht und Kontrolle zu verspüren greifen sie auf entwürdigendes und machtmissbrauchendes Verhalten zurück \parencite{Def_Form_Folge_Gewalt}.

Gewalt lässt sich in drei Kategorien aufteilen, orientiert an der gewaltausübenden Person. Diese Unterteilungen lassen sich in weitere Subgruppierungen teilen. Die \textit{selbstgerichtete Gewalt} beinhaltet suizidales Verhalten, wie Gedanken und Versuche an Suizid, sowie die erfolgreiche Vollendung solcher Versuche. Der zweite Bestandteil dieser Gewalt ist die Selbstmisshandlung \parencite{Gewaltarten_WHO}. Auch die \textit{zwischenmenschliche Gewalt} lässt sich in zwei Unterkategorien aufteilen. Die Gesellschaftsgewalt erfolgt zwischen Personen, die nicht miteinander verwandt sind und die sich möglicherweise nicht kennen und erfolgt demzufolge meist in der Öffentlichkeit. Ein weit bekanntes Beispiel für solhe Gewalt ist die Vergewaltigung oder die sexuelle Nötigung durch Fremde. Sie umfasst aber auch zufällige Gewaltausübungen oder die Gewalt innerhalb Institutionen wie der Schule, Arbeit, Gefängnissen oder Alten- und Pflegeheimen. Die zweite Untergruppe, und die im Rahmen dieser Arbeit wichtigere, ist die Familien- und speziell die Partnerschaftsgewalt \parencite{Gewaltarten_WHO}. Diese Unterkategorie der interpersonellen Gewalt entspricht der bereits zuvor dargeboten häuslichen Gewalt. Die dabei involvierten Personenkonstellationen können Kinder-Eltern, Eltern-Kinder, Geschwister und Partnerschaften sein \parencite{Def_Form_Folge_Gewalt}. Diese Studie fokusiert sich auf die Gewalt innerhalb der Partnerschaft. Die dritte und letzte Kategorie ist die \textit{kollektive Gewalt}, welche sich in soziale, politische und wirtschaftliche Gewalt unterteilen lässt. Diese Unterkategorien zeigen im Gegensatz zu den vorherigen Kategorien mögliche Gründe für die Gewalt durch Staaten oder großen Gruppen an Einzelpersonen. Terroranschläge oder auch Hassangriffe sind Beispiele für eine sozial motivierte kollektive Gewalt. Für politische Gewalt ist Krieg ein sehr prominentes Beispiel. Gewalthandlungen größerer Gruppen mit dem Ziel eines wirtschaftlichen Gewinns sind Bestandteil der wirtschaftlichen Gewalt. Gewalthandlungen durch größere Gruppen können stehts mehrere Motive haben \parencite{Gewaltarten_WHO}.


% GEWALTARTEN
\subsubsection{Gewaltarten}     \label{2.1.2.1}
Wenn man das Wort Gewalt hört, denken die meisten erst an \textit{physische Gewalt}. Doch so wie die Aggression kann sich die Gewalt auf unterschiedliche Arten manifestiert. Eine einheitliche Unterteilung der verschiedenen Gewaltarten ist in der Forschung bislang nicht gegeben. Je nach Schwerpunkt wird auf bis zu fünf Gewaltformen unterschieden. Zu der zu Beginn genannten physischen Gewalt gibt es noch die \textit{psychische, sexualisiert, soziale und ökonomische Gewalt} \parencite{Def_Form_Folge_Gewalt}.

% physische Gewalt
In Kapitel ~\ref{subsec_2.1.1} Aggressivität und Aggression wurde bereits auf die physische Aggression eingegangen und darauf, dass die Abgrenzung von Gewalt und Aggression nicht einfach ist. Denn wie auch die physische Aggression, ist Bestandteil der körperlichen Gewalt jede Form von physischen Angriffen \parencite{ph_G_wie_aggro}. Darunter zählen beispielsweise Tritte, Bisse, Würgen oder Gewaltausübungen mithilfe von Gegenständen. Im Extremfall kann es zur Tötung kommen \parencite{Gewaltart, Def_haus_Gewalt, physische_Gewalt_wie_aggro, Def_Form_Folge_Gewalt}.

% psychische Gewalt
Psychische Gewalt kann durch Worte oder auch durch Gesten und Gesichtsausdrücke erfolgen \parencite{Def_haus_Gewalt_2}. Einige Beispiele für psychische Gewalthandlungen sind Beleidigungen, wie Beschimpfungen und die damit einhergehende Demütigung der betroffenen Person. Oftmals beschädigt die gewalttätige Person Eigentum des Opfers oder behandelt dessen Haustiere nicht artgerecht. Es kann auch dazu kommen, dass die eigenen Kinder genutzt werden, um bei der betroffenen Person Druck auszuüben. Unter psychische Gewalt fallen auch  eifersüchtige Verhaltensweisen, die sich bei der Beendigung einer Beziehung in Stalking umwandeln können \parencite{Def_Form_Folge_Gewalt, Gewaltart, Def_haus_Gewalt_2}. Diese Taten werden verwendet, um den Gegenüber zu manipulieren oder um den eigenen Interssen und Absichten nachzugehen. Sie können auch Verwendung finden, um in böswilliger Absicht das Selbstbewusstsein des Partner zu senken, so das dieser widerstandslosen Gehorsam zeigt \parencite{Def_haus_Gewalt_2}. Diese Art von Gewalt ist nicht so wie die physische Gewalt zwangläufig am Körper sichtbar, ihre negativen Folgen können dennoch tiefer liegen und länger andauern \parencite{psych_Gewalt}. Oftmals ist die psychische Gewalt bloß ein Vorreiter für spätere sexuelle Misshandlungen \parencite{psych_Gewalt_2}. Des Weiteren werden Opfer einer solchen Gewalt von ihrem sozialen Umfeld nicht immer erkannt. Dies kann dazu führen, dass sie an ihrer persönlichen Wahrnehmung zweifeln. Die Reichweite und Intensität der Folgen ist von Person zu Person unterschiedlich und hängt von dessen Vorerfahrungen ab. Laut dem \textcite{Def_Form_Folge_Gewalt} zählt die Forschung die soziale und ökonomische Gewalt zu der psychischen dazu. Im Umfang dieser Arbeit werden diese drei Arten jedoch getrennt betrachtet. 

% sexualisiert Gewalt
Wie im zuvorigen Absatz bereits dargeboten, können psychische Gewalthandlungen, wie Drohungen oder oportune bloßstellende Kommentare, als Vorbote sexualisierter Gewalt genutzt werden \parencite{Übergang_psy_zu_sex_Gewalt}. Des Öfteren geben die Betroffenen auf, sich zu wehren und gehen den Anforderungen und Wünschen des Täters nach. Ein solches Verhalten wird von einigen als möglicher Konsens gedeutet \parencite{Def_haus_Gewalt_2}. Diese Art von Gewalt kann sich in unterschiedlichen sexualisierten Handlungen zeigen, wie die unerwünschte Nähe eines Partners, die Belästigung durch sexuelle Sprüche und Berührungen oder das Darbieten von pornografischen Bildern und Videos. Unter Vergewaltigung wird auch die Nötigung zu sexuellen Handlungen verstanden. Selbst der Versuch fällt unter die sexualisierte Gewalt \parencite{Def_haus_Gewalt_2, Gewaltart, Def_Form_Folge_Gewalt}. Die Zwangsgedanken und das Bedürfnis der Macht sind bei Vergewaltigungen stärker vorhanden, als bei anderen sexuellen Gewalthandlungen. Die ausübende Person ist der Auffassung, dass sie solche Handlungen ausüben muss, damit die Geschlechterhierarchie erhalten bleibt. Sie sind sich demzufolge oftmals ihrer Schuld nicht bewusst \parencite{Def_haus_Gewalt_2}.

% soziale Gewalt
Die soziale Gewalt wird vom \textcite{Def_Form_Folge_Gewalt} als Bestandteil der psychischen Gewalt angesehen. Auch sie ist körperlich nicht erkennbar. Es erfolgt eine soziale Abkoppelung des Opfers von dessn Umfeld. Der Kontakt zu Freunden oder Familien wird untersagt und auch das Treffen bekannter Personen ist sowohl außerhalb wie auch innerhalb des eigenen Zuhauses untersagt. In manchen Fällen sozialer Gewalt, werden die Telefonate durch den Täter mitgehört. Opfer haben oft nicht die Möglichkeit alleine das Haus zu verlassen. Sie werden von ihrem Partner zur und von der Arbeit gebracht \parencite{Def_haus_Gewalt_2, Def_Form_Folge_Gewalt}. Manche Opfer distanzieren sich, aufgrund der psychischen Belastungen, selbst von ihrem sozialen Umfeld. Durch den mangelden sozialen Kontakt ist die Hilfeleistung durch das Umfeld kaum, wenn nicht garnicht möglich \parencite{Def_haus_Gewalt_2}.

% ökonomische Gewalt
Wie auch die soziale Gewalt, wird die ökonomische Gewalt vom \textcite{Def_Form_Folge_Gewalt} als Unterkategorie der psychischen Gewalt gezählt. Müssen sich die Opfer wiederholt Vorwürfe der Fehlerhaften bzw. nicht ausreichenden Fähigkeiten ihren Beruf, den Haushalt oder die Erziehung der eigenen Kinder auszuüben, erzeugt dies große psychische Belastungen \parencite{Def_haus_Gewalt_2}. Es wird ihnen verboten einen Beruf auszuüben, oder sie müssen die Erwerbe an den Partner abgeben, so dass er der einzige ist, der Macht und Kontrolle über die Finanzen hat \parencite{Def_haus_Gewalt_2, Def_Form_Folge_Gewalt}. Dadurch sind die Betroffenen finanziel von ihren Partnern abhängig \parencite{physische_Gewalt_wie_aggro}.Damit die Gewalt von der Öffentlichkeit nicht erkennbar ist, werden den Betroffenen teure Geschenke überreicht. Dies stärkt die Unsicherheit und Bedenken den Partner zu verlassen, weil schwere finanzielle Folgen befürchtet werden \parencite{Übergang_psy_zu_sex_Gewalt}.

In Anlehung and das \textit{Rad der Gewalt} von \textcite{Rad_der_Gewalt} sind in Anhang A in Abbildung~\ref{Rad der Gewalt} die fünf Gewaltarten nochmal zusammenfassend bildlich dargestellt. %how to reference Anhang?

% URSACHE + KREISLAUF DER GEWALT
\subsubsection{Ursachen und Aufrechterhaltung von Gewalt}     \label{2.1.2.2}
Wie bei viele Situationen und Krankheiten, besteht bei der häusliche Gewalt, oder im Rahmen dieser Arbeit die Partnerschaftsgewalt eine Multikausalität. Dies bedeutet, dass nicht ein einziger Faktor als Ursache zu determinieren ist. Es ist vielmehr die Interaktion und Verwobenheit vieler unterschiedlicher Ursachen auf unterschiedlichen Ebenen \parencite{Ursache_hG}. Ein ökologisches Modell versucht anhand von vier Ebenen die Entstehung von Partnerschaftsgewalt zu systematisieren. Die unterschiedlichen Faktoren auf den jeweiligen Ebenen stehen im Zusammenhang miteinander und untereinander und bedingen sich somit gegenseitig. Durch die gemeinsame Interaktion können sie die Auftretenswahrscheinlichkeit von häuslicher Gewalt bevorzugen \parencite{Ursache_hG_2, Ursache_hG, Gewaltart}.
Auf der kleinsten Ebene steht das Individuum, dessen Verhalten durch persönliche, biologische und entwicklungsbedingte Faktoren beeinflusst wird. Ausschlaggebend sind hierbei eigene Erfahrungen mit Misshandlungen, sowie exessiver Konsum von Suchtmitteln \parencite{Ursache_hG_2, Ursache_hG, Gewaltart}. Das Verhalten der Personen ist laut \textcite{Gewaltart} sowie \textcite{Ursache_hG_2} ebenfalls durch psychische Störungen und Störungen der Persönlichkeit geprägt. Aber auch nicht klinische Eigenschaften, wie das Selbstwertgefühl oder die Stressregulationsfähigkeiten, haben einen bedeutsamen Einfluss auf menschliches Verhalten auf der individuellen Ebene \parencite{Ursache_hG}.
Eine Instanz höher, auf der Beziehungsebene, beschäftigt sich die Forschung mit der zwischenmenschlichen Interaktion naher Beziehungen. Hierbei wird die Art und Weise betrachtet, wie Partner im Austausch zueinander stehen, wie die Macht zwischen den Partnern verteilt ist und wie sie mit Konflikten unterschiedlichen Ausmaßes innerhalb der Beziehung umgehen \parencite{Ursache_hG_2, Ursache_hG, Gewaltart}.
Auf der Ebene der Gesellschaft wird das soziale und räumliche Umfeld, wie Verwandte, Freunde, Nachbarn, oder der Arbeitsplatz und Vereine betrachtet. Die Forschung betrachtet Aspekte wie die soziale Isolation oder ein gewalttolerierendes und unterstützendes Mileau \parencite{Ursache_hG_2, Ursache_hG, Gewaltart}. Die Nachbarschaft hat auch mit ihrer Rate an nicht Erwerbstätigen und durch mögliche Drogengeschäfte einen großen Einfluss auf das Verhalten der Beziehungspartner \parencite{Ursache_hG, Gewaltart}.
Die höchste Ebene ist die der Gesellschaft. Durch ihre sozialen und kulturellen Normen schafft sie ein gewaltförderndes oder -hinterndes Umfeld \parencite{Ursache_hG_2, Ursache_hG, Gewaltart}. Die positive bzw. negative Auffassung von Gewalt in der Gesellschaft generell, aber vor allem in der Politik, im Justizsystem und in den Medien beeinflusst das menschliche Verhalten \parencite{Ursache_hG_2, Ursache_hG}.  

Häufig entwickelt sich häusliche Gewalt  zu einem immer wiederkehrenden Kreislauf. Dieser beinhaltet die folgenden drei zusammenhängenden Zyklen:
\begin{itemize}
    \item Spannungsaufbau
    \item Gewaltausbruch
    \item Reue, Entschuldigungs- und Entlastungsversuche
\end{itemize}
Bei wiederholter Durchführung nimmt die zweite Phase an Intensität zu und tritt vermehrt auf. Da sich die entlastende Phase der Entschuldigung und Reue dabei verringert, fällt es den Opfern häuslicher Gewalt schwer, sich aus einer solchen Beziehung zu entfernen \parencite{Def_haus_Gewalt}.


\subsubsection{Folgen von Gewalt}     \label{2.1.2.3}
physische, psychische, soziale, finanzielle, gesellschaftliche (1) 
\parencite{Gewaltart}
Folgen in all ihren Dimensionen erfassen ist kaum möglich 
direkt, indirekt, kurzfristig, langfristig, chronifiziert
häusliche Gewalt = einer der gräßten Risikofaktoren für Gesundheit
je stärker Gewalterfahrung, desto häufiger gesundheitliche Beschwerden
physisch:
    Blaue Flecken, offene Wunden, Knochenbrüche, Bewusstlosigkeit, Komplikationen während Schwangerschaft (= direkte Folgen)
    körperliche Folgen haben Einfluss auf Lebensbereiche, wie Arbeit
    Selbstmedikation
    chronisch, wenn keine zeitnahe (ärztliche) Behandlung wahrgenommen
    (indirekte Spätfolgen) Schmerzsyndrome, Herzkreislaufbeschwerden, Beschwerden Bewegungsapparats und Atemwege
    Spätfolgen erklärbar durch andauernden Stressfaktor und somit negativ auf Gesundheit auswirken 
    schlimmster Ausmaß: Mod und Tötungsdelikte
psychisch:
    Angstgefühle, geringes Selbstwertgefühl, Niedergeschlagenheit, Depression, Scham- oder Schuldgefühle, Lustlosigkeit, Schlafstörungen, Konzentrationsschwierigkeiten Essstörungen, Selbstverletzun, Schwierigkeiten in der Beziehung zu Männern und in der Arbeit
    können zur chronifizierten PTBS oder Persönlichkeitsstörung führen
    Gewaltbetroffene Frauen weitaus häufiger Depression, Agststörung und Phobien
    Andere Strategien: Alkohol, Nikotin, Substanzmittel -> Selbstmedikation als Art innere Flucht
    Konsum hat wiederum negative Auswirkung auf körperliche und seelische Befinden
    Suizidalität höher: 10.7 Prozent sagen, haben versucht Suizid zu begehen \parencite{psy_Folgen_hG}
soziale:
    Opfer schämen oftmals und trauen sich nicht mit Bezugsperson über ihre Gewalterfahrungen zu sprechen oder Hilfe zu suchen
    Folge: von (helfenden) Umfeld zurückziehen, soziale Isolation
    Zerstörung familiären uns sozialen Strukturen
    durch Trennung ganze Familien (mit Kindern) auseinandergerissen
    Folge: Umzug, Wohnungslosigkeit, aus bestehenden sozialen Netzwerk herausgerissen
    Einfluss auf zukünftige Beziehungen
    oftmals Mühe, neue Partnerschaft einzugehen und wieder Vertrauen zu fassen
    Schwierigkeiten im Umgang mit Männern oder eigenen Sexualität
finanziel:
    niedriges Selbstwertgefühl -> Arbeit nicht mehr ausführen können
    oftmals Gefühl anfallende Erfordernisse nicht meistern können
    Verlust Arbeitsstelle bzw. längerfristige Erwerbslösigkeit -> psych./ phys. Probleme und soziale Isolation
    häusliche Gewalt = Armutsrisiko und sozialer Abstieg

gesundheitliche Folgen und Kosten (2)
\parencite{Def_haus_Gewalt}
zeitnah, langfristig
physisch, psychisch, psychosomatische Beschwerden, Tod
Auswirkungen können länger andeuern, auch wenn Misshandlung bereits beendet
Kosten erwartungsgemäß schwierig zu ermitteln
gesellschaftliche Kosten: in sozialen, juristischen Bereich, Gesundheits- und Bildungssektor
Erwerbsleben, Arbeitsunfähigkeit, Arbeitslosigkeit und Frühberentung
Folgekosten durch frühes Erkennen und adäquates Behandeln langfristig senken

gesundheitliche, sozioökonomische, psychosoziale (3)
\parencite{Def_haus_Gewalt_2}
gesundheitlich:
    kurz-, mittel, langfristige Folgen
    körperliche Folgen meist schneller sichtbar, jedoch psychische weitaus schwerwiegender
psychische:
    Angststörungen oder Depression
    mangeldes Selbstwertgefühl, Probleme im Umgang mit anderen Menschen insbesondere männliche Personen, Scham- sowie Schuldgefühle, Suchterkrankungen, PTBS, Suizidgedanken
    Selbstwert geschwächt -> sind Auffassung: durch Gewalt sind als Person weniger wert
    Folgen schlimmer je stärker Isolation
physisch:
    direkte körperliche Verletzungen, physische sowie psychosomatische Symptome (Magen-Darm-Probleme, nervöses Zucken)
    Substanzmittelkonsum, weil negative Gefühle von Angst und Hilflosigkeit mindern -> Selbstmedikation als Bewältigungsstrategie
sozioökonomisch:
    Schwierigkeiten am Arbeitsplatz
    durch steigende Stresssituation -> Anforderungen von Arbeitsalltag nicht mehr erfüllen
    Probleme sichtbar durch: Unpünktlichkeit, Krankenstände, Abwesenheit, geringe Arbeitsbelastung
    häufig Arbeitsplatzverlust, ständiger Arbeitsplatzwechsel
    häufig sozialer Abstieg
psychosozial:
    Kontakt zu engsten Vertrauten verlieren und somit kein Unterstützungssystem
    Probleme Hinblick auf neue Freundschaften und Beziehungen
    schwer jemandem gegenüber zu öffnen und Vertrauen aufzubauen
    schämen häufig und haben große Ängste und Wutgefühle, welche nicht mehr steuern können  -> machnmal ziehen selbst aus Netzwerk zurück

Folgen für Betroffene (4)
\parencite{Def_Form_Folge_Gewalt}
gesundheitlich:
    nicht nur direkt betroffene Opfer
    traumatisierend, versetzen in extreme Angst und Hilflosigkeit und überfordern normalen Anpassungs- und Bewältigungsstrategie
unmittelbar:
    körperlich: Prellungen, Verstauchungen, Hirnschütterung, Frakturen, innere Verletzungen
    auch unmittelbar mit psychischen Folgeproblemen: Leistungs- und Konzentrationsschwierigkeiten, erhöhte Medikamenten- und Alkoholkonsum
mittel- und langfristig:
    breites Sprektrum somatischer, psychosomatischer und psychischer Gesundheitsbelastung
    gynäkologische Beschwerden, Herz-Kreislauf-Beschwerden
    Depression, Stresssymptome, Angststörungen, PTBS, Essstörungen, Suizidalität
    gesundheitsgefährdende Bewältigungsstrategie
Art, Tragweite und Merkmale:
    Auswirkung von verschiedenen Faktoren beeinflusst: individuelle Vorraussetzungen, Form erlebter Gewalt, Verhältnis zur Tatperson
    psychische Gewalt längerfristig weit gravierendere Belastung als körperliche    -> psychische Langzeitfolgen belasten mehr
sozialer Bereich und Erwerbsleben
    Trennung, Scheidung, Auszug, Wegzug, Wechsel der Arbeitsplatzes, Schulwechsel   -> erhebliche Neuorientierung
    unmittelbar oder längerfristig kann auf Erwerbsleben auswirken
    Arbeitsunfähigkeit, Krankheitsabsenenz, Leistungsbußen


%- Folgen: körperlich; gesundheitsgefährdende (Überlebens-) Strategien; (Psycho-)somatische; reproduktive Gesundheit; psychische 
%- Istambul Konvention: was ist das; welche Aspekte; Artikel 11 und 13 (Partnerschaft Gewalt2020)



\subsection{Gewaltmythen}   \label{subsec_2.1.3}
hier auch vicitim blaming eingehen

% 2.2 Theoretischer Hintergrund: Aktueller Forschungsstand
\section{Aktueller Forschungsstand}   \label{sec_2.2}
In dieser Studie wird folgende Frage untersucht: Welchen Zusammenhang gibt es zwischen Aggressivität und der Akzeptanz von Gewaltmythen, sowie der Tendez zum victim blaming? (mögliche Quelle, die das unterschützt) 

In den darauffolgenden Unterkapiteln ~\ref{subsec_2.2.1} Hypothese 1, ~\ref{subsec_2.2.2} Hypothese 2 und ~\ref{subsec_2.2.3} Hypothese 3 erfolgt die Herleitung der zu untersuchenden Hypothesen auf Basis bereits bestehender Befunde.

\subsection{Hypothese 1}  \label{subsec_2.2.1}
In einer Studie mit $N$~=~1177 Universitätsstudenten, hypothetisiertern \textcite{H1_1993}, dass Männer, die in ihrer Vergangenheit sexuelle Aggression zeigten, dazu neigen, Opfern die Veranwortung zuzuschreiben. Mythen über häusliche Gewalt schließen Gewalt sexueller Form ein \parencite{H1_Poli_2022}.Im Rahmen dieser Arbeit wird sexualisierte Gewalt untersucht. Aggressives Agieren in einem Umfeld erhöht die Wahrscheinlichkeit, Aggression und das damit verbundene aggressive Verhalten in anderen Situationen darzustellen. Die Frage ist, ob ein höheres Maß an Aggression allgemein mit Victim Blaming in Verbindung steht.

Die in 2012 durchgeführt Studie von Kassim untersuchte unter anderem die Korrelation zwischen Victim Blaming und dem Anteil an aggressivem Verhalten bei malaysischen Jugendlichen. Sie kam mit einem $r$~=~.29 auf ein hoch signifikantes Ergebnis ($p$~=~.01) \parencite{H1_malasia_2012}. In ihrer theoretischen Aufarbeitung ihrer untersuchten Konstrukte eklärte Kassim, dass das Erleben von häuslicher Gewalt zur Akzeptanz der Mythen häuslicher Gewalt führt. Sie kam zur Erkenntnis, dass nicht nur die Verantwortungszuschreibung auf das Opfer, aber auch das Erleben von häuslicher Gewalt aggressives Verhalten aufklären \parencite{H1_malasia_2012}.

\textcite{H1_moderation_2020} untersuchten in ihrer Studie Aggression, häusliche Gewalt und Toleranz bei pakistanischen arbeitstätigen und verheirateten Frauen. Häusliche Gewalt wurde, wie in dieser vorliegenden Studie, mithilfe der Domestic Violence Myth Acceptance Scale (DVMAS; \textcite{Peters2003}) untersucht. \textcite{H1_moderation_2020} testeten zudem auch die Korrelation der Victim Blaming behandelnden DVMAS$-$Subskalen mit Aggression. Konträr zu \textcite{H1_malasia_2012} kamen sie jedoch zu den Ergebnissen, dass Aggression nicht mit Victim Blaming, im Sinne der Opferbeschuldigung aufgrund des Charakters und des Verhaltens und die Entschuldigung der Täter, korreliert ($r$~=~.04, $r$~=~.13, $r$~=~.11). Im Umfang einer Moderationsanalyse dieser Variablen mit einer Subskala der Toleranz kamen \textcite{H1_moderation_2020} zu dem signifikanten Ergebnis, dass die Interaktion zwischen dieser Subskala und der Opferbeschuldigung aufgrund des Charakters Aggression negativ vorhersagen (\textbeta~=~$-$.225, $p<$ .01). Toreranz scheint das Ausmaß des Victim Blaming bei gegebener Aggression zu mindern.

Die Studien von \textcite{H1_malasia_2012, H1_moderation_2020} beziehen sich auf den nahöstlichen und asiatischen Raum. Demzufolge können sich die Ergebnisse aus diesen Studien von der Diesigen unterscheiden. Des Weiteren bilden bei der Studie von \textcite{H1_malasia_2012} Jugendliche die Stichprobe. In der vorliegenden Studie werden ausschließlich Probanden herangezogen, die die Volljährigkeit erreicht haben. Nichtsdestotrotz wurde diese Studie als Beispiel des aktuellen Forschungsstandes berücksichtig. Durch die Reifung des präfrontalen Kortex nimmt während des Jugendalters die Intelligenz und das logische Denken zu. Auch die Frage der eigenen Identität steht in den späteren Jugendjahren im Vordergrund \parencite{H1_Entwicklung}. In diesen Jahren formen sich große Teile des späteren Selbst und die hier erhobene Stichprobe zeigt ihre größte Anhäufung an Probanden bei einem Alter von 21 Jahren (vgl. Abbildung~\ref{Histogramm Altersverteilung}). Durch diese Nähe an das Jugendalter wird die Studie von \textcite{H1_malasia_2012} dennoch für die Grundlage meiner Hypothese berücksichtig.

Es lässt sich vermuten, dass, wie zu Beginn dieses Kapitels angedeutet, das Ausmaß an Aggression einer Person mit dessen Tendenz zur Verantwortungszuschreibung gegenüber des Opfers zusammenhängen. Daraus bildet sich die ungerichtete Zusammenhangshypothese: Der Aggressionsscore korreliert mit Victim Blaming.
\subsection{Hypothese 2}    \label{subsec_2.2.2}
Bislang gibt es wenig Studien, die die Relation von Aggression und der Akzeptanz der Mythen häuslicher Gewalt untersucht. Es lässt sich jedoch vermuten, dass sie zusammenhängen, zumal der DVMAS mit patriachalen und sexistischen Einstellungen verbunden ist. Glaubende dieser Sichtweisen scheuen von der Anwendung von aggressiven Handlungen und Gewalt zur Behebung von Konflikten nicht zurück \parencite{DVMAS_Peters}. Obwohl Gewalt nicht gleichzusetzen ist mit Aggression, ist es auch nicht möglich zu sagen, dass diese beiden Begriffe nicht mit einander in Verbindung stehen.
Erst in jüngeren Jahren scheint die Korrelation von Aggression und Akzeptanz der Mythen häuslicher Gewalt an Interesse gewonnen zu haben. 

Die Studie von \textcite{H2_u_3_Bhogal_2016} mit $N$~=~121 Probanden untersuchte ob physische und verbale Aggression wie auch Ärger und Misstrauen die Akzeptanz der Vergewaltigungsmythen vorhersagen können. Die Ergebnisse zeigen, dass physische Aggression die Akzeptanz signifikant vorhersagt.

In einer Studie mit 100 verheirateten und berufstätigen Frauen untersuchten \textcite{H1_moderation_2020} unter anderem eine mögliche moderierenden Rolle der Toleranz$-$Subskala Neuheit zwischen häuslicher Gewalt und Aggression. Sie kamen zu der Erkenntnis, dass die Akzeptanz der Mythen häuslicher Gewalt ein signifikanter Prediktor von Aggression ist (\textbeta~=~.734, $p<$ .01).

Aufgrund des theoretischen Hintergrundes von \textcite{DVMAS_Peters} und den Ergebnissen der Studien von \textcite{H1_moderation_2020, H2_u_3_Bhogal_2016} bildet sich der Gedanke, dass Aggression und die Akzeptanz der Mythen häuslicher Gewalt häuslicher Gewalt positiv korrelieren. Daraus ergibt sich folgende gerichtete Zusammenhangshypothese: Der Aggressionsscore korreliert positiv mit der Akzeptanz der Mythen häuslicher Gewalt



\subsection{Hypothese 3}    \label{subsec_2.2.3}
\textcite{H2_u_3_Bhogal_2016} untersuchten in ihrer Studie 121 Studierende auf ihre Akzeptanz von Vergewaltigungsmythen und ihre Aggression in Form von physischer und verbaler Aggression, Ärger und Misstrauen. Sie kamen zum Entschluss, dass Männer eine höhere Akzeptanz von Vergewaltigungsmythen haben. Diese Mythen stellen die, meist weiblichen, als Mitschuldie da. Diese sichtweise des Opfers ist im Einklang mit den Mythen von häuslicher Gewalt. Ein weiteres Ergebnis der Untersuchung von \textcite{H2_u_3_Bhogal_2016} war, dass die physische Aggression bei Männern signifikant höher ist, verglichen mit den Frauen.

Laut \textcite{H3_MFUnterschied} sind Feministen der Meinung, dass Frauen ihre Aggressivität unterdrücken, gleichzeitig sind biologisch Positionierte der Auffassung, dass Frauen nicht die gleichen Fähigkeiten bzw. nicht das gleiche Bedürfniss haben, zu reagieren, so wie Männer es haben. Eine weitere Erklärung geht davon aus, dass Frauen das selbe biologische Potenzial für Aggressivität aufweisen, aber die Gesellschaft solch ein Verhalten ausschließlich bei Männern fördert. Die Verhaltensbiologie betrachtend, sind Männchen auf Grund von Testostron aggressiver als Weibchen. Obwohl Artenübergreifende Vergleiche mit Obhut zu genießen sind, bietet diese Tatsache einen starken Anhaltspunkt für einen Geschlechterunterschied menschlicher Aggressivität.

Im Rahmen einer Studie von $N$~=~329 Probanden zwischen 15 und 19 Jahren untersuchten \textcite{H3_2020} die Existenz sexualisierter Gewalt in romantischen Beziehungen, mögliche Zusammenhänge zwischen Mythen über sexualisierte Aggression und sexualisiertes Durchsetzungsvermögen und ihre möglichen geschlechterspezifischen Unterschiede. Ihre Ergebnisse zeigen, dass die männlichen Jugendlichen häufiger Täter sexualisierter Gewalt waren, und dass die männlichen Probanden vermehrt Mythen über sexualisierte Aggression Glauben schenkten. Das behandelte Thema des verwendeten Fragebogens ähnelt denen des DVMAS.

Aufgrund von \textcite{H2_u_3_Bhogal_2016, H3_MFUnterschied, H3_2020} liegt die Vermutung nahe, dass das Geschecht der Probanden eine moderierenden Rolle im Zusammenhang zwischen Akzeptanz von Gewaltmythen und Aggression hat. Aus diesem Gedanke ergibt sich die folgende Hypothese: Der Zusammenhang zwischen Akzeptanz von Gewaltmythen und Aggression wird durch das Geschlecht moderiert.

% Durch die in Kapitel 1.4 jeweiligen Studien von Haj-Yahia (2003), McElligott (2011), George und Martínez (2002) und Acosved und Long (2006) liegt der Gedanken nahe, dass der kulturelle Hintergrund der Betroffenen Auswirkungen auf die Wahrnehmung und daraus resultierende Verantwortungszuschreibung der Gesellschaft hat. Daraus ergibt sich die gerichtete Unterschiedshypothese:
