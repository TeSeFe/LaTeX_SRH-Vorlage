\subsection{Hypothese 3}    \label{subsec_2.2.3}
Laut %\textcite{H3_MFUnterschied} 
sind Feministen der Meinung, dass Frauen ihre Aggressivität unterdrücken, 
gleichzeitig sind biologisch Positionierte der Auffassung, dass Frauen nicht die gleichen 
Fähigkeiten bzw. nicht das gleiche Bedürfniss haben, zu reagieren wie Männer. Eine weitere
Erklärung geht davon aus, dass Frauen das selbe biologische Potenzial für Aggressivität
aufweisen, aber die Gesellschaft solch ein Verhalten ausschließlich bei Männern fördert.
Die Verhaltensbiologie betrachtend, sind Männchen auf Grund von Testostron aggressiver als
Weibchen. Obwohl Artenübergreifende Vergleiche mit Obhut zu genießen sind, bietet diese 
Tatsache einen starken Anhaltspunkt für einen Geschlechterunterschied menschlicher
Aggressivität.
