%%%%%%%%% wenn mehrere Anhänge vorhanden %%%%%%%%%
\begin{appendices}
    \chapter*{Anhang A}
    \addcontentsline{toc}{chapter}{A}
    \noindent \textit{Titel von Anhang A}

Wenn nur ein Anhang vorhanden ist, dann sieht die Syntax so aus:

\noindent \texttt{\char`\\ chapter*\char`\{Anhang\char`\}} \\
\texttt{\char`\\addcontentsline\char`\{toc\char`\}\char`\{chapter\char`\}\char`\{
    Anhang\char`\}} \\
\texttt{\char`\\noindent \char`\\textit\char`\{Titel des Anhangs\char`\}}


    \chapter*{Anhang B}
    \addcontentsline{toc}{chapter}{B}
    \noindent \textit{Titel von Anhang B}

    Inhalt
\end{appendices}
%%%%%%%%% wenn mehrere Anhänge vorhanden %%%%%%%%%



%%%%%%%%% wenn nur 1 Anhang vorhanden %%%%%%%%%
%\chapter*{Anhang}
%\addcontentsline{toc}{chapter}{Anhang}
%\noindent \textit{Titel des Anhangs}

%Inhalt
%%%%%%%%% wenn nur 1 Anhang vorhanden %%%%%%%%%