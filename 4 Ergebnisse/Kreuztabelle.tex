\begin{table}[htb]
    \caption[Kreuztabelle Manipulationscheck]{\textit {Kreuztabelle der Manipulationschecks}} 
    \label{Kreuztabelle}
    \centering
    \begin{adjustbox}{width=\textwidth}
    \small
    \begin{tabular}{lrrrrrrrrr}
      \hline
        &   & psychisch & sexualisiert & weibliches Opfer & männliches Opfer & niedriger SES Opfer & hoher SES Opfer & arabisch & deutsch \\
      \hline
    Ja   & Anzahl  & 32      & \textbf{354}     & \textbf{411}      & 14      & \textbf{389}      & 96      & 10      & \textbf{422}      \\
         & Prozent & 8.30\%  & \textbf{91.70\%} & \textbf{96.70\%}  & 3.30\%  & \textbf{80.20\%}  & 19.8\%  & 2.30\%  & \textbf{97.70\%}  \\
    Nein & Anzahl  & \textbf{400}     & 78      & 16       & \textbf{423}     & 37       & \textbf{342}     & \textbf{423}     & 9        \\
         & Prozent & \textbf{83.70\%} & 16.30\% & 3.60\%   & \textbf{96.40\%} & 9.80\%   & \textbf{90.20\%} & \textbf{97.90\%} & 2.10\%   \\
       \hline
    \end{tabular}
    \end{adjustbox}
    
    \begin{tablenotes}
        \item \textit{Anmerkungen.} \( N_{neu} \) = 864. SES = soziökonomischer Status. Prüffragen: (Gewaltart) Ging es um sexualisierte Gewalt? (Geschlecht) War das Opfer eine Frau? (SES) War die finanzielle Situation des Opfers schlechter als die finanzielle Situation des Täters? (Kultur) Hatten die Personen deutsche Namen?
      \end{tablenotes}
    \end{table}

    