\begin{table}[htb]
    \caption[Mittelwerte, Standardabweichung und Korrelation der Aggression$-$Subskalen, DVMAS und Victim Blaming]{\textit {Mittelwerte, Standardabweichung und Korrelation der Aggression$-$Subskalen, DVMAS und Victim Blaming}} 
    \label{lala}
    \centering
    \begin{adjustbox}{width=14cm} %{width=\textwidth}
    \small
    \begin{tabular}{lrrrrrrrr}
      \hline
        Variablen             & $M$  & $SD$   & 1     & 2     & 3     & 4    & 5 & 6\\
       \hline
       1 physische Aggression & 1.60  & 0.55  &       &       &        &      & &\\
       2 verbale Aggression   & 2.22  & 0.53  & .36** &       &        &      & &\\
       3 Ärger                & 1.96  & 0.58  & .46** & .39** &        &      & &\\
       4 Misstrauen           & 1.93  & 0.60  & .31** & .27** & .45**  &      & &\\
       5 DVMAS                & 2.60  & 0.80  & .28** & .25** & .19**  & .24** & &\\
       6 Victim Blaming       & 26.58 & 27.99 & .07*  & .02   & .04    & .02   & & \\
    \end{tabular}
    \end{adjustbox}
    
    \begin{tablenotes}
        \item \textit{Anmerkungen.} \( N_{neu} \)~=~864; DVMAS = Akzeptanz der Gewaltmythen; Wertebereich der Aggression-Subskalen von 1 (\textit{trifft nicht zu}) bis 4 (\textit{trifft voll zu});  DVMAS von 1 (\textit{stimme überhaupt nicht zu}) bis 7 (\textit{stimme völlig zu}); Victim Blaming von 1 (\textit{Verantwortungszuschreibung auf den Täter}) bis 101 (\textit{Verantwortungszuschreibung auf das Opfer}); Spearman$-$Rang$-$Korrelation. *$p<$~.05, **$p<$~.001
      \end{tablenotes}
    \end{table}


