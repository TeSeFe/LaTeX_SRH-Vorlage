\begin{table}[htb]
    \caption[Kreuztabelle Manipulationscheck soziökonomischer Status des Opfers]{\textit {Kreuztabelle des Manipulationschecks des soziökonomischen Status der betroffenen Person}} 
    \label{KT_SES}
    \centering
    \begin{adjustbox}{width=10cm} %{width=\textwidth}
    \small
    \begin{tabular}{lrrr}
      \hline
        &   & niedriger SES Opfer & hoher SES Opfer \\
      \hline
    Ja   & Anzahl  & 389      & 96      \\
         & Prozent & 80.20\%  & 19.8\%  \\
    Nein & Anzahl  & 37       & 342     \\
         & Prozent & 9.80\%   & 90.20\% \\
       \hline
    \end{tabular}
    \end{adjustbox}
    
    \begin{tablenotes}
        \item \textit{Anmerkungen.} \( N_{neu} \)~=~864. SES = soziökonomischer Status. Prüffrage: War die finanzielle Situation des Opfers schlechter als die finanzielle Situation des Täters?
      \end{tablenotes}
    \end{table}