\begin{table}[htb]
    \caption[Kreuztabelle Manipulationscheck Opfergeschlecht]{\textit {Kreuztabelle des Manipulationschecks des Geschlecht der betroffenen Person}} 
    \label{KT_sex}
    \centering
    \begin{adjustbox}{width=10cm} %{width=\textwidth}
    \small
    \begin{tabular}{lrrr}
      \hline
        &   & weibliches Opfer & männliches Opfer \\
      \hline
    Ja   & Anzahl  & 411      & 14      \\
    & Prozent & 96.70\%  & 3.30\%  \\
    Nein & Anzahl  & 16       & 423     \\
    & Prozent & 3.60\%   & 96.40\% \\
       \hline
    \end{tabular}
    \end{adjustbox}
    
    \begin{tablenotes}
        \item \textit{Anmerkungen.} \( N_{neu} \)~=~864. Prüffrage: \enquote{War das Opfer eine Frau?}.
      \end{tablenotes}
    \end{table}


