\chapter{Methoden}   \label{ch_3}
Empirische Prüfung der empirischen Hypthese(n). Die Informationen werden in der Vergangenheit
geschrieben.

\section{Stichprobenbeschreibung} \label{sec_3.1}
Rekrutierung und Eigenschaften der Stichprobe



\section{Untersuchungsdesign}  \label{sec_3.2}
Präregistrierung \\ %im Rahmen dessen gibt es weitere 4 Studien und NUR nennen was die erhoben
Feld-Laborstudie\\
genutzte Methode \\
Design \\
warum Online gemacht \\
effekte\\
randomiesiert und warum



\section{Operationalisierung der Konstrukte}    \label{sec_3.3}
anzahl FB + Vignette\\ % wieviel Items 2 Bsp. Skala, was erheben die; Reliabilität nennen
was wurde noch alles für die andren Arbeit erhoben \\ % 2 Bsp-Vignetten am besten so viele Faktoren
die 3 Gütekriterien

\section{Untersuchungsdurchführung}   \label{sec_3.4}
Zeitraum der Befragung \\
verteilung von FB \\
Bearbeitungszeit und VPN \\
Was wichtig für einleitungstext \\
sozio erhoben \\
potentielle störvariablen (wegen online nicht kontrollierbar)

\section{Auswertungsmethode}    \label{sec_3.5}
SPSS ausgewertet \\
deskriptive \\ % Lage, Streuung, Median, Modus, Mittelwert, SD
hypothesen \\ % das aus präregistrierung: gerichtet, unterschied, tests
% H3, weil Moderatorviariable hat wurde auf SPSS mit dem plug-in PROCESS berechnet

auf 1-2 items eingehen wenn über FB geschrieben wird