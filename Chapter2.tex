\chapter{Methoden}   \label{ch_2}
Empirische Prüfung der empirischen Hypthese(n)

\section{Fußnote}
Dieser Satz ist zur Darstellung der Fußnote \footnote{yeah} geschrieben worden.

\section{Darstellung von Begriffen und stat. Kennwerten}
Die Besonderheit von Tabellen- bzw. Abbildungsbeschriftungen sind in Kapitel
~\ref{ch_3}.
\begin{itemize} [leftmargin=1.25cm]
    \item biol. und lingu. Begriffe, Skalenbezeichnungen, zum 1. Mal verwendete
    Fachbegriffe:   \textit{Kursifschrift}
    \item lat. Symbole: \textit{Kursifschrift}
    \item griech. Symbole:  normale Schrift
    \item Darstellung von stat. Kennwerten, die nicht kleiner als $-1.00$ oder größer
    als $+1.00$ werden können:  mit führender Null und 2 Nachkommastellen
    \item Nachkommastellen bei Prozent und Stichprobengröße: kenie
    \item Nachkommastellen bei exakten $p$-Werten:  immer 3. bei $p = .000$ immer
    $p < .001$ schreiben
    \item Zahlen mit Dezimalstellen: durch \glqq Punkt \grqq getrennt
    \item num. Werte innerhalb Tabellen:    rechtsbündig formatiert
\end{itemize}

\subsection{Bestimmte stat. Symbole}
\begin{itemize} [leftmargin=1.25cm]
\item $M =$ Mittelwert
\item $SD =$ Standardabweichung
\item $d =$ Effektgröße nach Cohen
\item $r =$ Korrelation
\item $df =$ Freiheitsgrade
\item $p =$ Signifikanzwert (Alpha-Fehler)
\item $N =$ Größe der Gesamtstichprobe
\item $n =$ Anzahl der definierten Teilstichprobe
\end{itemize}