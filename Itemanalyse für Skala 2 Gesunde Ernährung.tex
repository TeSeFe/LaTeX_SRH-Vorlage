\begin{table}[htb]
    \caption[Itemanalyse für Skala 2 Gesunde Ernährung]{\textit{Itemanalyse für Skala 2 Gesunde Ernährung}} 
    \label{Itemanalyse für Skala 2 Gesunde Ernährung}
    \centering
    \begin{adjustbox}{width=\textwidth}
    \small
    \begin{tabular}{rlrrrrr}
      \hline
    Nr.      & Itemtext & \( M \) & \( SD \) & \( p_i \) & \( r_{it-i} \) 
             & \( r_{it-i}* \) \\
      \hline
    7.      & Ich denke, dass ich mich gesund ernähre.
      & 3.50	 & 0.95	   & .63	    & .25	    & .46   \\
    8.      & Ich denke, dass mein Körpergewicht gesund ist.
      & 3.99	 & 1.00	   & .75	    & .33	    & -     \\
    9.      & Wenn ich mein Essen frisch zubereite, fühle ich mich wohl.
      & 4.44	 & 0.75	   &.86	        &.45	    & .38   \\
    10.     & Wenn ich mein Körpergewicht konstant halte, fühle ich mich wohl.
      & 3.87	 & 1.03	   & .72	    & .24	    & -     \\
    11.     & Ich frühstücke täglich.
      & 3.42	 & 1.50	   & .61	    & .16	    & -     \\
    12.     & Zwischen meinen Hauptmahlzeiten vermeide ich fett- und zuckerreiche 
    Snacks.
      & 2.92	 & 1.21	   & .48	    & .26	    & .44     \\
       \hline
    \end{tabular}
    \end{adjustbox}
    
    \begin{tablenotes}
        \item \textit{Anmerkungen.} \( N \) = 135. Codierung der Items: 1 = stimme
        nicht zu, 2 = stimme eher nicht zu, 3 = stimme teilweise zu, 4 = stimme eher 
        zu, 5 = stimme vollständig zu.\linebreak(-) = recodiertes Item. \( M \) 
        Item-Mittelwert, \( SD \) Item-Streuung, \( p_i \) Item-Schwierigkeit, 
        \linebreak\( r_{it-i} \) Korregierte Item-Trennschärfe, \( r_{it-i}* \)
        Korregierte Item-Trennschärfe (revidiert). Cronbachs-\textalpha \ (Skala 2) = .60.
      \end{tablenotes}
    \end{table}